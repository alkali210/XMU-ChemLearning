% 编译提示:使用 XeLaTeX 编译
% 需要 ctex 宏包支持中文

\documentclass[a4paper,9pt]{article}
\usepackage{amsmath}
\usepackage{amssymb}
\usepackage{geometry}
\usepackage{ctex}
\usepackage{xcolor}
\usepackage{multicol}
\usepackage{titlesec}
\usepackage{fancyhdr}
\usepackage{tcolorbox}
\usepackage{amsthm}

\geometry{left=1cm,right=1cm,top=1cm,bottom=1.8cm}

\setlength{\columnsep}{0.8cm}
\setlength{\columnseprule}{0.4pt}
\renewcommand{\columnseprulecolor}{\color{gray!50}}

% 定义颜色方案
\definecolor{sectioncolor}{RGB}{0,102,204}
\definecolor{subsectioncolor}{RGB}{0,153,153}
\definecolor{keycolor}{RGB}{204,102,0}
\definecolor{derivcolor}{RGB}{100,149,237}
\definecolor{boxbg}{RGB}{240,248,255}
\definecolor{keybg}{RGB}{255,248,240}

% 自定义章节标题格式
\titleformat{\section}
  {\normalfont\large\bfseries\color{sectioncolor}}
  {\thesection}{0.5em}{}
\titlespacing*{\section}{0pt}{1ex plus 0.5ex minus 0.2ex}{0.8ex plus 0.2ex}

\titleformat{\subsection}
  {\normalfont\normalsize\bfseries\color{subsectioncolor}}
  {\thesubsection}{0.5em}{}
\titlespacing*{\subsection}{0pt}{0.8ex plus 0.3ex minus 0.1ex}{0.5ex plus 0.1ex}

% 自定义定理环境
\newtheorem{theorem}{原理}

% 自定义关键点环境
\newenvironment{keypoint}
{\par\vspace{0.3em}\noindent\begin{tcolorbox}[
  colback=keybg,
  colframe=keycolor,
  boxrule=0.5pt,
  arc=2pt,
  left=2pt,
  right=2pt,
  top=2pt,
  bottom=2pt,
  boxsep=0pt]
\small\noindent\textcolor{keycolor}{\textbf{关键点:}}}
{\end{tcolorbox}\par\vspace{0.3em}}

% 自定义推导环境
\newenvironment{derivation}
{\par\vspace{0.3em}\noindent\begin{tcolorbox}[
  colback=boxbg,
  colframe=derivcolor,
  boxrule=0.5pt,
  arc=2pt,
  left=2pt,
  right=2pt,
  top=2pt,
  bottom=2pt,
  boxsep=0pt]
\small\noindent\textcolor{derivcolor}{\textbf{推导:}}\par\linespread{1.1}\selectfont}
{\end{tcolorbox}\par\vspace{0.3em}}

% 自定义例题环境
\newenvironment{exampleblock}[1]
{\par\vspace{0.3em}\noindent\begin{tcolorbox}[
  colback=boxbg,
  colframe=derivcolor,
  boxrule=0.5pt,
  arc=2pt,
  left=2pt,
  right=2pt,
  top=2pt,
  bottom=2pt,
  boxsep=0pt]
\small\noindent\textcolor{derivcolor}{\textbf{#1}}\par\linespread{1.1}\selectfont}
{\end{tcolorbox}\par\vspace{0.3em}}

% 页眉页脚设置
\pagestyle{fancy}
\fancyhf{}
\fancyfoot[C]{\scriptsize9-\thepage}
\renewcommand{\headrulewidth}{0pt}
\renewcommand{\footrulewidth}{0.4pt}
\renewcommand{\footrule}{\hbox to\headwidth{\color{gray!50}\leaders\hrule height \footrulewidth\hfill}}

% 紧凑标题
\title{\vspace{-2em}\Large\bfseries\color{sectioncolor} Chapter 9 Collections - 双原子分子\vspace{-1em}}
\author{}
\date{2025 年 11 月 1 日}

\begin{document}

\maketitle
\thispagestyle{fancy}

\begin{multicols}{2}

\section{双原子分子的哈密顿算符}

\subsection{玻恩-奥本海默近似}

双原子分子的哈密顿算符(SI单位):
\begin{equation}
\begin{split}
\hat{H} &= -\frac{\hbar^2}{2m_e}\sum_i\nabla_i^2 - \frac{\hbar^2}{2}\sum_A\frac{1}{M_A}\nabla_A^2 - \sum_{i,A}\frac{Z_Ae^2}{4\pi\varepsilon_0 r_{iA}} \\
&\quad + \sum_{i<j}\frac{e^2}{4\pi\varepsilon_0 r_{ij}} + \sum_{A<B}\frac{Z_AZ_Be^2}{4\pi\varepsilon_0 R_{AB}}
\end{split}
\end{equation}

\begin{keypoint}
\textbf{Born-Oppenheimer近似}:由于核质量远大于电子质量($M \gg m_e$),核运动比电子慢得多,可将核坐标视为参数,分离核运动和电子运动。
\end{keypoint}

电子哈密顿算符(固定核坐标$R$):
\begin{equation}
\hat{H}_{el}(R) = -\frac{\hbar^2}{2m_e}\sum_i\nabla_i^2 - \sum_{i,A}\frac{Z_Ae^2}{4\pi\varepsilon_0 r_{iA}} + \sum_{i<j}\frac{e^2}{4\pi\varepsilon_0 r_{ij}} + \frac{Z_AZ_Be^2}{4\pi\varepsilon_0 R}
\end{equation}

电子能量$E_{el}(R)$加上核排斥能构成势能曲线$U(R)$。

\subsection{核运动方程}

在势能$U(R)$中,核的薛定谔方程:
\begin{equation}
\left[-\frac{\hbar^2}{2\mu}\frac{d^2}{dR^2} + U(R)\right]\chi(R) = E\chi(R)
\end{equation}

其中$\mu = \frac{M_AM_B}{M_A+M_B}$为约化质量。

进一步分离振动和转动:
\begin{equation}
\chi(R) = \frac{\xi(R)}{R}Y_{JM}(\theta,\phi)
\end{equation}

\section{$\mathbf{H_2^+}$分子离子}

\subsection{精确解的存在性}

$\text{H}_2^+$是最简单的分子,包含2个质子和1个电子,是唯一可精确求解的分子体系。

电子哈密顿算符(原子单位):
\begin{equation}
\hat{H} = -\frac{1}{2}\nabla^2 - \frac{1}{r_A} - \frac{1}{r_B} + \frac{1}{R}
\end{equation}

采用椭球坐标系$(\lambda, \mu, \phi)$:
\begin{align}
\lambda &= \frac{r_A + r_B}{R}, \quad 1 \le \lambda < \infty \\
\mu &= \frac{r_A - r_B}{R}, \quad -1 \le \mu \le 1
\end{align}

可分离变量得到精确解。

\subsection{键级与稳定性}

\textbf{键级}定义:
\begin{equation}
\text{键级} = \frac{1}{2}(\text{成键电子数} - \text{反键电子数})
\end{equation}

\begin{exampleblock}{例题:$\text{H}_2^+$的键级}
$\text{H}_2^+$电子组态:$(\sigma_g 1s)^1$

键级$= \frac{1}{2}(1-0) = 0.5$

实验:$D_e = 2.79$ eV,$R_e = 1.06$ \AA

理论计算与实验符合很好,证明LCAO-MO方法的有效性。
\end{exampleblock}

\section{LCAO-MO方法}

\subsection{基本假设}

分子轨道表示为原子轨道的线性组合(LCAO-MO):
\begin{equation}
\psi_{MO} = \sum_i c_i \phi_{AO,i}
\end{equation}

对$\text{H}_2^+$:
\begin{equation}
\psi = c_1\phi_{1s_A} + c_2\phi_{1s_B}
\end{equation}

\subsection{成键与反键轨道}

由线性变分法,解久期方程得两个分子轨道:

\textbf{成键轨道}($\sigma_g$):
\begin{equation}
\psi_g = \frac{1}{\sqrt{2(1+S)}}(\phi_A + \phi_B)
\end{equation}

能量:$E_g = \frac{H_{AA} + H_{AB}}{1+S}$(能量降低)

\textbf{反键轨道}($\sigma_u^*$):
\begin{equation}
\psi_u = \frac{1}{\sqrt{2(1-S)}}(\phi_A - \phi_B)
\end{equation}

能量:$E_u = \frac{H_{AA} - H_{AB}}{1-S}$(能量升高)

\begin{derivation}
\textbf{库仑积分与交换积分}:

库仑积分:
\begin{equation*}
H_{AA} = \int \phi_A^* \hat{H} \phi_A d\tau = E_{1s} + \int \phi_A^* \left(-\frac{1}{r_B}\right) \phi_A d\tau
\end{equation*}

交换积分:
\begin{equation*}
H_{AB} = \int \phi_A^* \hat{H} \phi_B d\tau = S \cdot E_{1s} + \int \phi_A^* \left(-\frac{1}{r_A}\right) \phi_B d\tau
\end{equation*}

重叠积分:
\begin{equation*}
S = \int \phi_A^* \phi_B d\tau
\end{equation*}

这些积分都是核间距$R$的函数。
\end{derivation}

\section{分子轨道对称性}

\subsection{对称性标记}

\textbf{关于对称中心}(仅同核双原子分子) :
\begin{itemize}\setlength{\itemsep}{0pt}
\item $g$(gerade,偶):对中心反演不变
\item $u$(ungerade,奇):对中心反演改变符号
\end{itemize}

\textbf{关于分子轴}:
\begin{itemize}\setlength{\itemsep}{0pt}
\item $\sigma$:绕轴旋转对称($|m_l|=0$)
\item $\pi$:一个节面通过轴($|m_l|=1$)
\item $\delta$:两个节面通过轴($|m_l|=2$)
\end{itemize}

\begin{keypoint}
完整标记示例:$\sigma_g$、$\sigma_u^*$、$\pi_g$、$\pi_u^*$
\end{keypoint}

键能、键长与键级的关系:

\textbf{键级越大}:键能越大,键长越短,振动频率越高

\begin{exampleblock}{示例:N$_2$、O$_2$、F$_2$的比较}
\begin{center}
\begin{tabular}{cccc}
\hline
分子 & 键级 & $D_e$/eV & $R_e$/\AA \\
\hline
N$_2$ & 3 & 9.90 & 1.10 \\
O$_2$ & 2 & 5.21 & 1.21 \\
F$_2$ & 1 & 1.65 & 1.42 \\
\hline
\end{tabular}
\end{center}

键级越大,键能越大,键长越短。
\end{exampleblock}

\section{分子光谱项}

\subsection{分子的量子数}

\textbf{轨道角动量投影}:
\begin{equation}
\Lambda = |M_L| = \left|\sum_i m_{l_i}\right|
\end{equation}

用希腊字母表示:$\Lambda = 0, 1, 2, 3, \ldots$对应$\Sigma, \Pi, \Delta, \Phi, \ldots$

\textbf{自旋投影}(沿分子轴) :
\begin{equation}
\Sigma_s = M_S = \sum_i m_{s_i}
\end{equation}

注意:$\Sigma_s$是量子数,$\Sigma$是光谱项符号,两者不同。

\textbf{总角动量投影}:
\begin{equation}
\Omega = \Lambda + \Sigma_s
\end{equation}

\subsection{光谱项符号详解}

分子光谱项完整记号:$^{2S+1}\Lambda_{\Omega,g/u}^{\pm}$

\textbf{各符号含义}:
\begin{itemize}\setlength{\itemsep}{0pt}
\item $2S+1$:自旋多重度(左上角)
\item $\Lambda$:轨道角动量符号($\Sigma, \Pi, \Delta, \Phi, \ldots$)
\item $\Omega$:总角动量投影(右下角下标)
\item $g/u$:反演对称性(仅同核分子,右下角)
\item $\pm$:反射对称性(仅$\Sigma$态,右上角)
\end{itemize}

\textbf{$\Sigma$态的额外标记}:

对于$\Lambda = 0$的$\Sigma$态,需标记反射对称性:
\begin{itemize}\setlength{\itemsep}{0pt}
\item $\Sigma^+$:对包含分子轴的平面反射不变
\item $\Sigma^-$:反射后改变符号
\end{itemize}

\begin{exampleblock}{例题1:O$_2$的基态}
O$_2$电子组态:$\ldots(\pi_{2p_x}^*)^1(\pi_{2p_y}^*)^1$

两个$\pi^*$电子占据简并轨道:
\begin{itemize}\setlength{\itemsep}{0pt}
\item $m_{l_1} = +1$,$m_{l_2} = -1$,$\Lambda = |1-1| = 0$ ($\Sigma$态)
\item 自旋平行(Hund规则):$S = 1$,$2S+1 = 3$
\item 同核分子,反演对称:偶宇称 $g$
\item 反射对称性:$\Sigma^-$
\end{itemize}

基态光谱项:$^3\Sigma_g^-$
\end{exampleblock}

\begin{exampleblock}{例题2:NO分子的基态}
NO电子组态:$\ldots(\pi_{2p})^4(\pi_{2p}^*)^1$

单个$\pi^*$电子:
\begin{itemize}\setlength{\itemsep}{0pt}
\item $|m_l| = 1$,$\Lambda = 1$ ($\Pi$态)
\item $S = 1/2$,$2S+1 = 2$(双重态)
\item $\Omega = \Lambda + \Sigma_s = 1 \pm 1/2 = 3/2$ 或 $1/2$
\item 异核分子,无$g/u$标记
\end{itemize}

基态光谱项:$^2\Pi_{1/2}$ 或 $^2\Pi_{3/2}$

实际基态为 $^2\Pi_{1/2}$(由自旋-轨道耦合决定)
\end{exampleblock}

\begin{exampleblock}{例题3:N$_2$的基态}
N$_2$电子组态:$\ldots(\sigma_{2p})^2$

两个$\sigma$电子,自旋反平行:
\begin{itemize}\setlength{\itemsep}{0pt}
\item $\Lambda = 0$ ($\Sigma$态)
\item $S = 0$,$2S+1 = 1$(单重态)
\item 同核分子:$g$(偶宇称)
\item 反射对称性:$\Sigma^+$
\end{itemize}

基态光谱项:$^1\Sigma_g^+$
\end{exampleblock}

\subsection{光谱项简并度}

\textbf{原子光谱项简并度}:

对于原子光谱项 $^{2S+1}L_J$:
\begin{equation}
g = 2J + 1
\end{equation}

不考虑自旋-轨道耦合时(仅$L$和$S$) :
\begin{equation}
g = (2L+1)(2S+1)
\end{equation}

\textbf{分子光谱项简并度}:

对于$\Lambda \neq 0$的分子光谱项 $^{2S+1}\Lambda$:
\begin{equation}
g = 2(2S+1) = 4S+2
\end{equation}

因子2来自$\pm\Lambda$的简并。

对于$\Lambda = 0$的$\Sigma$态 $^{2S+1}\Sigma^{\pm}$:
\begin{equation}
g = 2S+1
\end{equation}

无轨道简并($\Sigma^+$和$\Sigma^-$不简并)。

考虑$\Omega$时(Hund's case a) :
\begin{equation}
g_{\Omega} = 
\begin{cases}
2 & \Omega \neq 0 \\
1 & \Omega = 0
\end{cases}
\end{equation}

\begin{exampleblock}{示例:简并度计算}
\begin{itemize}\setlength{\itemsep}{0pt}
\item $^3\Sigma_g^-$(O$_2$):$g = 2S+1 = 3$
\item $^2\Pi$(NO):$g = 2(2S+1) = 4$
\item $^1\Sigma_g^+$(N$_2$):$g = 2S+1 = 1$
\item $^3\Pi_g$:$g = 2(2S+1) = 6$(不考虑$\Omega$)
\end{itemize}
\end{exampleblock}

\section{光电子能谱}

\subsection{基本原理}

光电子能谱(PES)利用光电效应:
\begin{equation}
h\nu = IE + KE
\end{equation}

其中:
\begin{itemize}\setlength{\itemsep}{0pt}
\item $h\nu$:入射光子能量
\item $IE$:电离能
\item $KE$:出射电子动能
\end{itemize}

测量电子动能分布,得到各分子轨道的电离能。

\begin{theorem}[Koopmans定理]
分子轨道的电离能近似等于该轨道能量的负值:
\begin{equation}
IE_i \approx -\varepsilon_i
\end{equation}
\end{theorem}

假设:电离过程中其他轨道不变(冻结轨道近似)。

\begin{exampleblock}{应用:N$_2$的光电子能谱}
N$_2$的PES显示三组峰:

\begin{itemize}\setlength{\itemsep}{0pt}
\item $\sigma_{2p}$:$IE \approx 18$ eV
\item $\pi_{2p}$:$IE \approx 17$ eV
\item $\sigma_{2s}$:$IE \approx 19$ eV
\end{itemize}

谱线精细结构反映振动能级。
\end{exampleblock}

\section{分子振动}

\subsection{谐振子近似}

双原子分子的振动,在平衡位置附近可近似为谐振子:
\begin{equation}
V(x) = \frac{1}{2}k x^2, \quad x = R - R_e
\end{equation}

振动能级:
\begin{equation}
E_v = \hbar\omega\left(v + \frac{1}{2}\right), \quad v = 0,1,2,\ldots
\end{equation}

其中$\omega = \sqrt{k/\mu}$,$k$为力常数。

\subsection{非谐振子修正}

实际势能曲线偏离谐振子,采用Morse势:
\begin{equation}
V(R) = D_e[1 - e^{-\beta(R-R_e)}]^2
\end{equation}

能级(含非谐性修正):
\begin{equation}
E_v = \hbar\omega\left(v + \frac{1}{2}\right) - \hbar\omega x_e\left(v + \frac{1}{2}\right)^2
\end{equation}

$x_e$为非谐性常数,$x_e \ll 1$。

\begin{keypoint}
振动能级间隔随$v$增大而减小,最终趋于离解极限。
\end{keypoint}

\section{分子转动}

\subsection{刚性转子模型}

固定核间距$R_e$,分子转动的能级:
\begin{equation}
E_J = BJ(J+1), \quad J = 0,1,2,\ldots
\end{equation}

转动常数:
\begin{equation}
B = \frac{\hbar^2}{2I} = \frac{\hbar^2}{2\mu R_e^2}
\end{equation}

转动能级间隔:
\begin{equation}
\Delta E = E_{J+1} - E_J = 2B(J+1)
\end{equation}

\subsection{振-转耦合}

考虑振动时核间距变化,转动常数依赖于振动量子数:
\begin{equation}
B_v = B_e - \alpha_e\left(v + \frac{1}{2}\right)
\end{equation}

总能量:
\begin{equation}
E_{v,J} = \hbar\omega\left(v + \frac{1}{2}\right) + B_vJ(J+1)
\end{equation}

\section{常用公式速查}

\begin{scriptsize}
\begin{itemize}\setlength{\itemsep}{0pt}
\item 键级:$\frac{1}{2}($成键电子$ - $反键电子$)$
\item 约化质量:$\mu = \frac{M_A M_B}{M_A + M_B}$
\item 成键轨道能量:$E_g = \frac{H_{AA} + H_{AB}}{1+S}$
\item 反键轨道能量:$E_u = \frac{H_{AA} - H_{AB}}{1-S}$
\item 分子轨道角动量:$\Lambda = |\sum_i m_{l_i}|$
\item 光谱项符号:$^{2S+1}\Lambda_{\Omega,g/u}^{\pm}$
\item 原子简并度:$g = 2J+1$ 或 $(2L+1)(2S+1)$
\item 分子简并度:$g = 4S+2$ ($\Lambda \neq 0$) 或 $2S+1$ ($\Sigma$态)
\item 振动能级:$E_v = h\nu(v + \frac{1}{2})$
\item 转动能级:$E_J = BJ(J+1)$
\item 转动常数:$B = \frac{h}{8\pi^2 c I}$ (cm$^{-1}$)
\item 光电子能谱:$h\nu = IE + KE$
\item Morse势:$V = D_e[1-e^{-\beta(R-R_e)}]^2$
\item 波数换算:$1$ eV $= 8065.5$ cm$^{-1}$
\end{itemize}
\end{scriptsize}

\end{multicols}

\end{document}

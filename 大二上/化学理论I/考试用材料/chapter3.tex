% 编译提示:使用 XeLaTeX 编译
% 需要 ctex 宏包支持中文

\documentclass[a4paper,9pt]{article}
\usepackage{amsmath}
\usepackage{amssymb}
\usepackage{geometry}
\usepackage{ctex}
\usepackage{xcolor}
\usepackage{multicol}
\usepackage{titlesec}
\usepackage{fancyhdr}
\usepackage{tcolorbox}
\usepackage{amsthm}

\geometry{left=1cm,right=1cm,top=1cm,bottom=1.8cm}

\setlength{\columnsep}{0.8cm}
\setlength{\columnseprule}{0.4pt}
\renewcommand{\columnseprulecolor}{\color{gray!50}}

% 定义颜色方案
\definecolor{sectioncolor}{RGB}{0,102,204}
\definecolor{subsectioncolor}{RGB}{0,153,153}
\definecolor{keycolor}{RGB}{204,102,0}
\definecolor{derivcolor}{RGB}{100,149,237}
\definecolor{boxbg}{RGB}{240,248,255}
\definecolor{keybg}{RGB}{255,248,240}

% 自定义章节标题格式
\titleformat{\section}
  {\normalfont\large\bfseries\color{sectioncolor}}
  {\thesection}{0.5em}{}
\titlespacing*{\section}{0pt}{1ex plus 0.5ex minus 0.2ex}{0.8ex plus 0.2ex}

\titleformat{\subsection}
  {\normalfont\normalsize\bfseries\color{subsectioncolor}}
  {\thesubsection}{0.5em}{}
\titlespacing*{\subsection}{0pt}{0.8ex plus 0.3ex minus 0.1ex}{0.5ex plus 0.1ex}

% 自定义定理环境
\newtheorem{theorem}{原理}

% 自定义关键点环境
\newenvironment{keypoint}
{\par\vspace{0.3em}\noindent\begin{tcolorbox}[
  colback=keybg,
  colframe=keycolor,
  boxrule=0.5pt,
  arc=2pt,
  left=2pt,
  right=2pt,
  top=2pt,
  bottom=2pt,
  boxsep=0pt]
\small\noindent\textcolor{keycolor}{\textbf{关键点:}}}
{\end{tcolorbox}\par\vspace{0.3em}}

% 自定义推导环境
\newenvironment{derivation}
{\par\vspace{0.3em}\noindent\begin{tcolorbox}[
  colback=boxbg,
  colframe=derivcolor,
  boxrule=0.5pt,
  arc=2pt,
  left=2pt,
  right=2pt,
  top=2pt,
  bottom=2pt,
  boxsep=0pt]
\small\noindent\textcolor{derivcolor}{\textbf{推导:}}\par\linespread{1.1}\selectfont}
{\end{tcolorbox}\par\vspace{0.3em}}

% 自定义例题环境
\newenvironment{exampleblock}[1]
{\par\vspace{0.3em}\noindent\begin{tcolorbox}[
  colback=boxbg,
  colframe=derivcolor,
  boxrule=0.5pt,
  arc=2pt,
  left=2pt,
  right=2pt,
  top=2pt,
  bottom=2pt,
  boxsep=0pt]
\small\noindent\textcolor{derivcolor}{\textbf{#1}}\par\linespread{1.1}\selectfont}
{\end{tcolorbox}\par\vspace{0.3em}}

% 页眉页脚设置
\pagestyle{fancy}
\fancyhf{}
\fancyfoot[C]{\scriptsize3-\thepage}
\renewcommand{\headrulewidth}{0pt}
\renewcommand{\footrulewidth}{0.4pt}
\renewcommand{\footrule}{\hbox to\headwidth{\color{gray!50}\leaders\hrule height \footrulewidth\hfill}}

% 紧凑标题
\title{\vspace{-2em}\Large\bfseries\color{sectioncolor} Chapter 3 Collections - 薛定谔方程与箱中粒子\vspace{-1em}}
\author{}
\date{2025 年 11 月 1 日}

\begin{document}

\maketitle
\thispagestyle{fancy}

\begin{multicols}{2}

\section{薛定谔方程}

\subsection{含时薛定谔方程}

描述量子态随时间演化的基本方程:
\begin{equation}
i\hbar\frac{\partial\psi}{\partial t} = \hat{H}\psi
\end{equation}

其中哈密顿算符:
\begin{equation}
\hat{H} = -\frac{\hbar^2}{2m}\nabla^2 + V(\vec{r},t)
\end{equation}

\subsection{定态薛定谔方程}

当势能不显含时间 $V = V(\vec{r})$ 时,可分离变量:
\begin{equation}
\psi(\vec{r},t) = \psi(\vec{r})e^{-iEt/\hbar}
\end{equation}

得到定态薛定谔方程:
\begin{equation}
\hat{H}\psi = E\psi
\end{equation}

或展开形式:
\begin{equation}
-\frac{\hbar^2}{2m}\nabla^2\psi + V\psi = E\psi
\end{equation}

\begin{keypoint}
定态意味着所有可观测量的期望值不随时间变化。概率密度 $|\psi(\vec{r})|^2$ 与时间无关。
\end{keypoint}

\subsection{波函数的物理意义}

\textbf{Born概率诠释}:
\begin{equation}
P = \int_{V} |\psi(\vec{r})|^2 d\tau
\end{equation}

表示在体积 $V$ 内发现粒子的概率。

\textbf{归一化条件}:
\begin{equation}
\int_{-\infty}^{\infty} |\psi|^2 d\tau = 1
\end{equation}

\textbf{概率流密度}:
\begin{equation}
\vec{j} = \frac{\hbar}{2mi}\left(\psi^*\nabla\psi - \psi\nabla\psi^*\right) = \frac{\hbar}{m}\text{Im}(\psi^*\nabla\psi)
\end{equation}

概率守恒方程(连续性方程):
\begin{equation}
\frac{\partial|\psi|^2}{\partial t} + \nabla\cdot\vec{j} = 0
\end{equation}

\section{算符理论基础}

\subsection{算符的定义}

\textbf{线性算符}:满足 $\hat{A}(c_1f_1 + c_2f_2) = c_1\hat{A}f_1 + c_2\hat{A}f_2$

\textbf{厄米算符}:满足 $\int f^*\hat{A}g\,d\tau = \int g(\hat{A}f)^*\,d\tau$

\begin{keypoint}
量子力学中,可观测量对应厄米算符。厄米算符的本征值必为实数。
\end{keypoint}

\subsection{基本算符}

\textbf{位置算符}:$\hat{x} = x$(乘法算符)

\textbf{动量算符}:$\hat{p}_x = -i\hbar\frac{\partial}{\partial x}$

\textbf{能量算符}:$\hat{E} = i\hbar\frac{\partial}{\partial t}$

\textbf{哈密顿算符}:$\hat{H} = \frac{\hat{p}^2}{2m} + V = -\frac{\hbar^2}{2m}\nabla^2 + V$

\subsection{对易关系}

对易子定义:$[\hat{A},\hat{B}] = \hat{A}\hat{B} - \hat{B}\hat{A}$

\textbf{正则对易关系}:
\begin{equation}
[\hat{x},\hat{p}_x] = i\hbar
\end{equation}

\begin{derivation}
证明 $[\hat{x},\hat{p}_x] = i\hbar$:

作用于任意波函数 $\psi(x)$:
\begin{align*}
[\hat{x},\hat{p}_x]\psi &= \hat{x}\hat{p}_x\psi - \hat{p}_x\hat{x}\psi \\
&= x\left(-i\hbar\frac{\partial\psi}{\partial x}\right) - \left(-i\hbar\frac{\partial(x\psi)}{\partial x}\right) \\
&= -i\hbar x\frac{\partial\psi}{\partial x} + i\hbar\left(\psi + x\frac{\partial\psi}{\partial x}\right) \\
&= i\hbar\psi
\end{align*}

因此 $[\hat{x},\hat{p}_x] = i\hbar$(恒等算符)
\end{derivation}

\begin{keypoint}
不对易的物理量不能同时有确定值(不确定性原理的根源)。
\end{keypoint}

\subsection{本征值问题}

本征方程:$\hat{A}\psi = a\psi$

其中 $a$ 为本征值(可观测量的可能测量值),$\psi$ 为本征函数(对应的态)。

\textbf{厄米算符性质}:
\begin{itemize}\setlength{\itemsep}{0pt}\setlength{\parskip}{0pt}
\item 本征值为实数
\item 不同本征值对应的本征函数正交
\item 本征函数系完备
\end{itemize}

\section{一维无限深势阱}

\subsection{问题描述}

势能函数:
\begin{equation}
V(x) = \begin{cases}
0 & 0 \le x \le a \\
\infty & x < 0 \text{ 或 } x > a
\end{cases}
\end{equation}

边界条件:$\psi(0) = \psi(a) = 0$

\subsection{求解过程}

在势阱内 $(0 < x < a)$,薛定谔方程:
\begin{equation}
-\frac{\hbar^2}{2m}\frac{d^2\psi}{dx^2} = E\psi
\end{equation}

引入 $k^2 = \frac{2mE}{\hbar^2}$,通解:
\begin{equation}
\psi(x) = A\sin(kx) + B\cos(kx)
\end{equation}

应用边界条件 $\psi(0) = 0 \Rightarrow B = 0$

$\psi(a) = 0 \Rightarrow \sin(ka) = 0 \Rightarrow ka = n\pi$

\subsection{本征值与本征函数}

\textbf{能量本征值}(量子化):
\begin{equation}
E_n = \frac{n^2\pi^2\hbar^2}{2ma^2} = \frac{n^2h^2}{8ma^2}, \quad n = 1,2,3,\ldots
\end{equation}

\textbf{归一化波函数}:
\begin{equation}
\psi_n(x) = \sqrt{\frac{2}{a}}\sin\left(\frac{n\pi x}{a}\right)
\end{equation}

\begin{keypoint}
\textbf{零点能}:基态能量 $E_1 = \frac{\pi^2\hbar^2}{2ma^2} \neq 0$,这是量子效应的体现,不能用经典力学解释。
\end{keypoint}

\begin{exampleblock}{例题:能级间隔}
计算一维势阱中相邻能级的间隔。

能级差:
\begin{align*}
\Delta E_{n+1,n} &= E_{n+1} - E_n \\
&= \frac{(n+1)^2 - n^2}{2ma^2}\pi^2\hbar^2 \\
&= \frac{2n+1}{2ma^2}\pi^2\hbar^2
\end{align*}

可见能级间隔随 $n$ 增大而增大(与经典不同)。

对于宏观粒子(如质量 $m = 1$ g,$a = 1$ cm),
\begin{equation*}
E_1 \sim 10^{-63} \text{ J}
\end{equation*}
能级间隔极小,可视为连续(对应原理)。
\end{exampleblock}

\subsection{波函数性质}

\textbf{正交归一性}:
\begin{equation}
\int_0^a \psi_n^*\psi_m\,dx = \delta_{nm}
\end{equation}

\textbf{宇称}:
\begin{itemize}\setlength{\itemsep}{0pt}
\item $n$ 为奇数:偶宇称(关于 $x = a/2$ 对称)
\item $n$ 为偶数:奇宇称(关于 $x = a/2$ 反对称)
\end{itemize}

\textbf{节点数}:第 $n$ 个态有 $n-1$ 个节点(零点)

\section{概率密度与期望值}

\subsection{概率分布}

概率密度:
\begin{equation}
\rho_n(x) = |\psi_n(x)|^2 = \frac{2}{a}\sin^2\left(\frac{n\pi x}{a}\right)
\end{equation}

\begin{exampleblock}{基态概率分布特点}
$n=1$ 时:
\begin{equation*}
\rho_1(x) = \frac{2}{a}\sin^2\left(\frac{\pi x}{a}\right)
\end{equation*}

\begin{itemize}\setlength{\itemsep}{0pt}
\item 在 $x = a/2$ 处最大(最可能位置)
\item 在边界处为零
\item 与经典均匀分布完全不同
\end{itemize}

高能级时,概率分布趋于均匀(对应原理)。
\end{exampleblock}

\subsection{物理量期望值}

\textbf{位置平均值}:
\begin{equation}
\langle x \rangle_n = \int_0^a \psi_n^* x \psi_n\,dx = \frac{a}{2}
\end{equation}

(对所有 $n$ 均成立,由对称性)

\textbf{动量平均值}:
\begin{equation}
\langle p \rangle_n = \int_0^a \psi_n^* \left(-i\hbar\frac{d}{dx}\right) \psi_n\,dx = 0
\end{equation}

\begin{derivation}
计算 $\langle x^2 \rangle$ 和 $\langle p^2 \rangle$:

\textbf{位置平方}:
\begin{align*}
\langle x^2 \rangle &= \frac{2}{a}\int_0^a x^2\sin^2\left(\frac{n\pi x}{a}\right)dx \\
&= \frac{a^2}{3} - \frac{a^2}{2n^2\pi^2}
\end{align*}

\textbf{动量平方}:
\begin{align*}
\langle p^2 \rangle &= \int_0^a \psi_n^*\left(-\hbar^2\frac{d^2}{dx^2}\right)\psi_n\,dx \\
&= \frac{n^2\pi^2\hbar^2}{a^2}
\end{align*}

验证能量关系:
\begin{equation*}
\langle E \rangle = \frac{\langle p^2 \rangle}{2m} = \frac{n^2\pi^2\hbar^2}{2ma^2} = E_n
\end{equation*}
\end{derivation}

\subsection{不确定性关系验证}

位置标准偏差:
\begin{equation}
\Delta x = \sqrt{\langle x^2 \rangle - \langle x \rangle^2} = a\sqrt{\frac{1}{12} - \frac{1}{2n^2\pi^2}}
\end{equation}

动量标准偏差:
\begin{equation}
\Delta p = \sqrt{\langle p^2 \rangle - \langle p \rangle^2} = \frac{n\pi\hbar}{a}
\end{equation}

不确定性乘积:
\begin{equation}
\Delta x \cdot \Delta p = n\pi\hbar\sqrt{\frac{1}{12} - \frac{1}{2n^2\pi^2}} > \frac{\hbar}{2}
\end{equation}

基态 $(n=1)$ 时:$\Delta x \cdot \Delta p \approx 0.568\hbar > \frac{\hbar}{2}$ ✓

\section{三维势箱}

\subsection{势能与边界条件}

三维直角势箱:
\begin{equation}
V(x,y,z) = \begin{cases}
0 & 0 \le x \le a, 0 \le y \le b, 0 \le z \le c \\
\infty & \text{其他}
\end{cases}
\end{equation}

边界条件:在六个面上 $\psi = 0$

\subsection{分离变量求解}

设 $\psi(x,y,z) = X(x)Y(y)Z(z)$,代入薛定谔方程可得三个独立的一维方程:
\begin{align}
-\frac{\hbar^2}{2m}\frac{d^2X}{dx^2} &= E_x X \\
-\frac{\hbar^2}{2m}\frac{d^2Y}{dy^2} &= E_y Y \\
-\frac{\hbar^2}{2m}\frac{d^2Z}{dz^2} &= E_z Z
\end{align}

其中 $E = E_x + E_y + E_z$

\subsection{能级与简并}

\textbf{波函数}:
\begin{equation}
\psi_{n_xn_yn_z} = \sqrt{\frac{8}{abc}}\sin\left(\frac{n_x\pi x}{a}\right)\sin\left(\frac{n_y\pi y}{b}\right)\sin\left(\frac{n_z\pi z}{c}\right)
\end{equation}

\textbf{能量本征值}:
\begin{equation}
E_{n_xn_yn_z} = \frac{\pi^2\hbar^2}{2m}\left(\frac{n_x^2}{a^2} + \frac{n_y^2}{b^2} + \frac{n_z^2}{c^2}\right)
\end{equation}

其中 $n_x, n_y, n_z = 1,2,3,\ldots$

\begin{exampleblock}{例题:立方势箱的简并}
对于立方势箱 $(a=b=c)$:
\begin{equation*}
E_{n_xn_yn_z} = \frac{\pi^2\hbar^2}{2ma^2}(n_x^2 + n_y^2 + n_z^2)
\end{equation*}

\textbf{基态}:$(1,1,1)$,$E_{111} = \frac{3\pi^2\hbar^2}{2ma^2}$,非简并

\textbf{第一激发态}:$(2,1,1), (1,2,1), (1,1,2)$
\begin{equation*}
E = \frac{6\pi^2\hbar^2}{2ma^2}
\end{equation*}
三重简并

\textbf{第二激发态}:$(2,2,1)$ 及其排列
\begin{equation*}
E = \frac{9\pi^2\hbar^2}{2ma^2}
\end{equation*}
三重简并
\end{exampleblock}

\begin{keypoint}
简并度与对称性密切相关。对称性越高,简并度越大。
\end{keypoint}

\section{一维环形势箱}

\subsection{问题描述}

粒子限制在半径为 $r_0$ 的圆环上运动(二维平面内的一维运动)。

势能函数:
\begin{equation}
V(\phi) = \begin{cases}
0 & r = r_0 \\
\infty & r \neq r_0
\end{cases}
\end{equation}

其中 $\phi$ 为方位角,$0 \le \phi < 2\pi$。

\subsection{薛定谔方程求解}

在环上($r = r_0$),薛定谔方程简化为:
\begin{equation}
-\frac{\hbar^2}{2I}\frac{d^2\psi}{d\phi^2} = E\psi
\end{equation}

其中转动惯量 $I = mr_0^2$。

引入 $k^2 = \frac{2IE}{\hbar^2}$,通解为:
\begin{equation}
\psi(\phi) = Ae^{ik\phi} + Be^{-ik\phi}
\end{equation}

\subsection{周期性边界条件}

\textbf{单值性条件}:$\psi(\phi + 2\pi) = \psi(\phi)$

这要求:
\begin{equation}
e^{ik \cdot 2\pi} = 1 \implies k = m_l
\end{equation}

其中 $m_l = 0, \pm 1, \pm 2, \pm 3, \ldots$(整数)

\begin{keypoint}
环形势箱的量子数可以取零和负值,这与直线势箱($n \ge 1$)不同。
\end{keypoint}

\subsection{能量本征值与本征函数}

\textbf{能量本征值}:
\begin{equation}
E_{m_l} = \frac{m_l^2\hbar^2}{2I} = \frac{m_l^2\hbar^2}{2mr_0^2}
\end{equation}

特点:
\begin{itemize}\setlength{\itemsep}{0pt}\setlength{\parskip}{0pt}
\item 基态($m_l = 0$)能量为零(无零点能)
\item $E_{m_l} = E_{-m_l}$,除基态外所有能级二重简并
\item 能级间隔 $\Delta E = \frac{(2|m_l|+1)\hbar^2}{2I}$
\end{itemize}

\textbf{归一化波函数}:
\begin{equation}
\psi_{m_l}(\phi) = \frac{1}{\sqrt{2\pi}}e^{im_l\phi}
\end{equation}

归一化条件:
\begin{equation}
\int_0^{2\pi} \psi_{m_l}^*\psi_{m_l'}\,d\phi = \delta_{m_l m_l'}
\end{equation}

\subsection{角动量与物理意义}

环形势箱中粒子的角动量算符:
\begin{equation}
\hat{L}_z = -i\hbar\frac{\partial}{\partial\phi}
\end{equation}

作用于本征函数:
\begin{equation}
\hat{L}_z\psi_{m_l} = m_l\hbar\psi_{m_l}
\end{equation}

\textbf{物理意义}:
\begin{itemize}\setlength{\itemsep}{0pt}\setlength{\parskip}{0pt}
\item $m_l > 0$:逆时针旋转
\item $m_l < 0$:顺时针旋转
\item $m_l = 0$:静止(无角动量)
\end{itemize}

能量与角动量关系:
\begin{equation}
E = \frac{L_z^2}{2I} = \frac{(m_l\hbar)^2}{2I}
\end{equation}

这与经典转子能量 $E = \frac{L^2}{2I}$ 一致。

\begin{exampleblock}{例题:环形势箱的概率分布}
对于任意本征态 $\psi_{m_l} = \frac{1}{\sqrt{2\pi}}e^{im_l\phi}$:

概率密度:
\begin{equation*}
|\psi_{m_l}|^2 = \frac{1}{2\pi}
\end{equation*}

结论:\textbf{所有能量本征态的概率密度均匀分布},与量子数 $m_l$ 无关。

但动量(角动量)却是确定的:$L_z = m_l\hbar$。

这说明粒子位置完全不确定,但角动量完全确定,符合不确定性关系:
\begin{equation*}
\Delta\phi \cdot \Delta L_z \ge \frac{\hbar}{2}
\end{equation*}
\end{exampleblock}

\subsection{与氢原子的联系}

环形势箱模型与氢原子方位角部分完全相同:

氢原子波函数:$\psi_{nlm} = R_{nl}(r)Y_l^m(\theta,\phi)$

其中 $Y_l^m(\theta,\phi) \propto P_l^m(\cos\theta)e^{im\phi}$

方位角部分:
\begin{equation}
\Phi_m(\phi) = \frac{1}{\sqrt{2\pi}}e^{im\phi}
\end{equation}

与环形势箱波函数形式相同!

\begin{keypoint}
环形势箱是理解原子轨道角动量量子化的简化模型。
\end{keypoint}

\begin{exampleblock}{应用:苯分子的 $\pi$ 电子}
苯分子($\text{C}_6\text{H}_6$)有 6 个 $\pi$ 电子在环形共轭体系中运动。

近似为环形势箱,环周长 $L = 6 \times 1.40 = 8.4$ \AA

半径:$r_0 = \frac{L}{2\pi} = 1.34$ \AA

能级:$E_{m_l} = \frac{m_l^2\hbar^2}{2mr_0^2}$

电子填充:
\begin{itemize}\setlength{\itemsep}{0pt}
\item $m_l = 0$:2 个电子(基态)
\item $m_l = \pm 1$:4 个电子(第一激发态,简并)
\end{itemize}

HOMO $\to$ LUMO 跃迁:$m_l = 1 \to m_l = 2$

\begin{align*}
\Delta E &= \frac{(2^2 - 1^2)\hbar^2}{2m(4.20 \times 10^{-10})^2} = \frac{3\hbar^2}{2m(4.20 \times 10^{-10})^2} \\
&\approx 5.4 \text{ eV}
\end{align*}

对应波长:$\lambda \approx 230$ nm(紫外区)

实验值约 $260$ nm,模型定性正确。
\end{exampleblock}

\section{共轭体系的自由电子模型}

\subsection{模型假设}

对于共轭分子(如丁二烯 $\text{C}_4\text{H}_6$),$\pi$ 电子近似为在一维势箱中运动。

势箱长度:$a \approx (N_C - 1) \times 1.40$ \AA

其中 $N_C$ 为碳原子数,$1.40$ \AA 为 C-C 键长。

\subsection{能级填充}

每个能级可容纳 2 个电子(自旋相反)。

\textbf{基态}:$N_{\pi}$ 个 $\pi$ 电子从最低能级依次填充。

HOMO(最高占据轨道):$n = N_{\pi}/2$

LUMO(最低空轨道):$n = N_{\pi}/2 + 1$

\subsection{光学跃迁}

最低能量跃迁(HOMO $\to$ LUMO):
\begin{equation}
\Delta E = E_{n+1} - E_n = \frac{(2n+1)\pi^2\hbar^2}{2ma^2}
\end{equation}

对应吸收波长:
\begin{equation}
\lambda = \frac{hc}{\Delta E} = \frac{2ma^2hc}{(2n+1)\pi^2\hbar^2}
\end{equation}

\begin{exampleblock}{例题:丁二烯的紫外吸收}
丁二烯 $\text{C}_4\text{H}_6$:4 个 $\pi$ 电子

势箱长度:$a = 3 \times 1.40 = 4.20$ \AA

HOMO:$n=2$,LUMO:$n=3$

\begin{align*}
\Delta E &= \frac{(2 \times 2 + 1)\pi^2\hbar^2}{2m(4.20 \times 10^{-10})^2} \\
&\approx 5.0 \text{ eV}
\end{align*}

对应波长:
\begin{equation*}
\lambda = \frac{1240 \text{ eV·nm}}{5.0 \text{ eV}} \approx 248 \text{ nm}
\end{equation*}

实验值约 $217$ nm,模型较粗糙但定性正确。
\end{exampleblock}

\section{隧穿效应}

\subsection{势垒穿透}

考虑矩形势垒:
\begin{equation}
V(x) = \begin{cases}
0 & x < 0 \\
V_0 & 0 \le x \le a \\
0 & x > a
\end{cases}
\end{equation}

当粒子能量 $E < V_0$ 时,经典粒子被完全反射,但量子粒子有概率穿透势垒。

\subsection{透射系数}

对于 $E < V_0$,透射系数:
\begin{equation}
T \approx \frac{16E(V_0-E)}{V_0^2}e^{-2\kappa a}
\end{equation}

其中 $\kappa = \frac{\sqrt{2m(V_0-E)}}{\hbar}$

\begin{keypoint}
隧穿概率随势垒宽度 $a$ 和高度 $(V_0-E)$ 的增加而指数衰减。
\end{keypoint}

\subsection{应用实例}

\textbf{$\alpha$ 衰变}:原子核中 $\alpha$ 粒子穿透库仑势垒

\textbf{扫描隧道显微镜(STM)}:电子在针尖与样品间隧穿

\textbf{氨分子反演}:氮原子穿透势垒导致能级分裂

\begin{exampleblock}{例题:隧穿概率估算}
电子穿透势垒:$V_0 = 5$ eV,$a = 1$ nm,$E = 3$ eV

\begin{equation*}
\kappa = \frac{\sqrt{2 \times 9.109 \times 10^{-31} \times 2 \times 1.602 \times 10^{-19}}}{\hbar} \approx 7.26 \times 10^9 \text{ m}^{-1}
\end{equation*}

\begin{equation*}
T \approx \frac{16 \times 3 \times 2}{25}e^{-2 \times 7.26 \times 10^9 \times 10^{-9}} \approx 0.038e^{-14.5} \approx 2 \times 10^{-8}
\end{equation*}

虽然概率很小,但对微观粒子这是可观测的量子效应。
\end{exampleblock}

\section{常用公式速查}

\begin{scriptsize}
\begin{itemize}\setlength{\itemsep}{0pt}
\item 普朗克常数:$h = 6.626 \times 10^{-34}$ J·s,$\hbar = \frac{h}{2\pi} = 1.055 \times 10^{-34}$ J·s
\item 电子质量:$m_e = 9.109 \times 10^{-31}$ kg
\item 一维势阱能级:$E_n = \frac{n^2h^2}{8ma^2}$
\item 三维立方势箱:$E = \frac{h^2}{8ma^2}(n_x^2 + n_y^2 + n_z^2)$
\item 一维环形势箱:$E_{m_l} = \frac{m_l^2\hbar^2}{2I}$,$m_l = 0, \pm 1, \pm 2, \ldots$
\item 零点能:$E_0 = \frac{h^2}{8ma^2}$ (一维直线),$E_0 = 0$ (环形)
\item 对易子:$[\hat{x},\hat{p}_x] = i\hbar$
\item 不确定关系:$\Delta x \cdot \Delta p \ge \frac{\hbar}{2}$
\item de Broglie 波长:$\lambda = \frac{h}{p}$
\item 能量-波长关系:$E(\text{eV}) = \frac{1240}{\lambda(\text{nm})}$
\item 隧穿因子:$e^{-2\kappa a}$,$\kappa = \frac{\sqrt{2m(V_0-E)}}{\hbar}$
\end{itemize}
\end{scriptsize}

\end{multicols}

\end{document}

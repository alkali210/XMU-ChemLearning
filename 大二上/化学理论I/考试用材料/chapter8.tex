% 编译提示:使用 XeLaTeX 编译
% 需要 ctex 宏包支持中文

\documentclass[a4paper,9pt]{article}
\usepackage{amsmath}
\usepackage{amssymb}
\usepackage{geometry}
\usepackage{ctex}
\usepackage{xcolor}
\usepackage{multicol}
\usepackage{titlesec}
\usepackage{fancyhdr}
\usepackage{tcolorbox}
\usepackage{amsthm}

\geometry{left=1cm,right=1cm,top=1cm,bottom=1.8cm}

\setlength{\columnsep}{0.8cm}
\setlength{\columnseprule}{0.4pt}
\renewcommand{\columnseprulecolor}{\color{gray!50}}

% 定义颜色方案
\definecolor{sectioncolor}{RGB}{0,102,204}
\definecolor{subsectioncolor}{RGB}{0,153,153}
\definecolor{keycolor}{RGB}{204,102,0}
\definecolor{derivcolor}{RGB}{100,149,237}
\definecolor{boxbg}{RGB}{240,248,255}
\definecolor{keybg}{RGB}{255,248,240}

% 自定义章节标题格式
\titleformat{\section}
  {\normalfont\large\bfseries\color{sectioncolor}}
  {\thesection}{0.5em}{}
\titlespacing*{\section}{0pt}{1ex plus 0.5ex minus 0.2ex}{0.8ex plus 0.2ex}

\titleformat{\subsection}
  {\normalfont\normalsize\bfseries\color{subsectioncolor}}
  {\thesubsection}{0.5em}{}
\titlespacing*{\subsection}{0pt}{0.8ex plus 0.3ex minus 0.1ex}{0.5ex plus 0.1ex}

% 自定义定理环境
\newtheorem{theorem}{原理}

% 自定义关键点环境
\newenvironment{keypoint}
{\par\vspace{0.3em}\noindent\begin{tcolorbox}[
  colback=keybg,
  colframe=keycolor,
  boxrule=0.5pt,
  arc=2pt,
  left=2pt,
  right=2pt,
  top=2pt,
  bottom=2pt,
  boxsep=0pt]
\small\noindent\textcolor{keycolor}{\textbf{关键点:}}}
{\end{tcolorbox}\par\vspace{0.3em}}

% 自定义推导环境
\newenvironment{derivation}
{\par\vspace{0.3em}\noindent\begin{tcolorbox}[
  colback=boxbg,
  colframe=derivcolor,
  boxrule=0.5pt,
  arc=2pt,
  left=2pt,
  right=2pt,
  top=2pt,
  bottom=2pt,
  boxsep=0pt]
\small\noindent\textcolor{derivcolor}{\textbf{推导:}}\par\linespread{1.1}\selectfont}
{\end{tcolorbox}\par\vspace{0.3em}}

% 自定义例题环境
\newenvironment{exampleblock}[1]
{\par\vspace{0.3em}\noindent\begin{tcolorbox}[
  colback=boxbg,
  colframe=derivcolor,
  boxrule=0.5pt,
  arc=2pt,
  left=2pt,
  right=2pt,
  top=2pt,
  bottom=2pt,
  boxsep=0pt]
\small\noindent\textcolor{derivcolor}{\textbf{#1}}\par\linespread{1.1}\selectfont}
{\end{tcolorbox}\par\vspace{0.3em}}

% 页眉页脚设置
\pagestyle{fancy}
\fancyhf{}
\fancyfoot[C]{\scriptsize8-\thepage}
\renewcommand{\headrulewidth}{0pt}
\renewcommand{\footrulewidth}{0.4pt}
\renewcommand{\footrule}{\hbox to\headwidth{\color{gray!50}\leaders\hrule height \footrulewidth\hfill}}

% 紧凑标题
\title{\vspace{-2em}\Large\bfseries\color{sectioncolor} Chapter 8 Collections - 多电子原子\vspace{-1em}}
\author{}
\date{2025 年 11 月 1 日}

\begin{document}

\maketitle
\thispagestyle{fancy}

\begin{multicols}{2}

\section{多电子原子的哈密顿算符}

\subsection{完整哈密顿算符}

$N$ 电子原子的哈密顿算符(SI 单位):
\begin{equation}
\hat{H} = -\sum_{i=1}^{N}\frac{\hbar^2}{2m_e}\nabla_i^2 - \sum_{i=1}^{N}\frac{Ze^2}{4\pi\varepsilon_0 r_i} + \sum_{i<j}\frac{e^2}{4\pi\varepsilon_0 r_{ij}}
\end{equation}

三项分别为:电子动能、核-电子吸引、电子-电子排斥。

原子单位制下($\hbar = m_e = e = 4\pi\varepsilon_0 = 1$):
\begin{equation}
\hat{H} = -\frac{1}{2}\sum_{i=1}^{N}\nabla_i^2 - \sum_{i=1}^{N}\frac{Z}{r_i} + \sum_{i<j}\frac{1}{r_{ij}}
\end{equation}

\begin{keypoint}
由于电子-电子排斥项 $\sum_{i<j} \frac{1}{r_{ij}}$ 的存在,薛定谔方程无法精确求解,必须采用近似方法。
\end{keypoint}

\subsection{原子单位制}

原子单位的定义和换算关系:

\begin{align}
\text{长度:} a_0 &= \frac{4\pi\varepsilon_0\hbar^2}{m_e e^2} = 0.529 \text{ \AA} \\
\text{能量:} E_h &= \frac{m_e e^4}{16\pi^2\varepsilon_0^2\hbar^2} = 27.211 \text{ eV} \\
\text{电荷:} e &= 1.602 \times 10^{-19} \text{ C}
\end{align}

\begin{exampleblock}{例题8-1:单位换算}
将 $1E_h$ 换算为 J、eV、cm$^{-1}$。

\textbf{解:}$1E_h = 27.211$ eV $= 4.3597 \times 10^{-18}$ J

利用 $E = h\nu = hc\tilde{\nu}$:
\begin{equation*}
\tilde{\nu} = \frac{E}{hc} = \frac{4.3597 \times 10^{-18}}{(6.626 \times 10^{-34})(2.998 \times 10^{10})} = 2.1947 \times 10^5 \text{ cm}^{-1}
\end{equation*}
\end{exampleblock}

\section{中心力场近似}

\subsection{基本假设}

将电子-电子排斥用一个球对称的平均势能 $U_i(r_i)$ 代替:
\begin{equation}
\hat{H} \approx \sum_{i=1}^{N} \hat{h}(i), \quad \hat{h}(i) = -\frac{1}{2}\nabla_i^2 - \frac{Z_{\text{eff}}(r_i)}{r_i}
\end{equation}

其中有效核电荷 $Z_{\text{eff}}$ 考虑了其他电子的屏蔽效应。

\subsection{单电子波函数}

在中心力场近似下,总波函数可分离为单电子轨道的乘积:
\begin{equation}
\Psi(1,2,\ldots,N) = \psi_1(1)\psi_2(2)\cdots\psi_N(N)
\end{equation}

每个单电子波函数形式类似氢原子:
\begin{equation}
\psi_{nlm}(r,\theta,\phi) = R_{nl}(r)Y_l^m(\theta,\phi)
\end{equation}

但径向函数 $R_{nl}(r)$ 由有效势 $V_{\text{eff}}(r) = -\frac{Z_{\text{eff}}(r)}{r}$ 决定,需自洽求解。

\subsection{屏蔽效应与Slater规则}

\textbf{Slater 规则}用于估算有效核电荷:
\begin{equation}
Z_{\text{eff}} = Z - \sigma
\end{equation}

屏蔽常数 $\sigma$ 的计算规则:
\begin{enumerate}\setlength{\itemsep}{0pt}\setlength{\parskip}{0pt}
\item 将电子按 $(1s)(2s,2p)(3s,3p)(3d)(4s,4p)(4d)(4f)\ldots$ 分组
\item 对于 $ns$ 或 $np$ 电子:
\begin{itemize}\setlength{\itemsep}{0pt}
\item 同组其他电子贡献 0.35(1s 为 0.30)
\item $n-1$ 壳层电子贡献 0.85
\item 更内层电子贡献 1.00
\end{itemize}
\item 对于 $nd$ 或 $nf$ 电子:
\begin{itemize}\setlength{\itemsep}{0pt}
\item 同组其他电子贡献 0.35
\item 所有内层电子贡献 1.00
\end{itemize}
\end{enumerate}

\begin{exampleblock}{例题8-2:计算氮原子的有效核电荷}
氮原子电子组态:$1s^2 2s^2 2p^3$,$Z = 7$。

对 2p 电子:
\begin{align*}
\sigma &= 2 \times 0.85 \quad (\text{1s 电子}) \\
&\quad + 2 \times 0.35 \quad (\text{2s 电子}) \\
&\quad + 2 \times 0.35 \quad (\text{其他 2p 电子}) \\
&= 1.70 + 0.70 + 0.70 = 3.10
\end{align*}

$Z_{\text{eff}} = 7 - 3.10 = 3.90$
\end{exampleblock}

\section{自旋与Pauli原理}

\subsection{电子自旋}

电子具有内禀角动量——自旋:
\begin{align}
\hat{S}^2 |s,m_s\rangle &= s(s+1)\hbar^2 |s,m_s\rangle \\
\hat{S}_z |s,m_s\rangle &= m_s\hbar |s,m_s\rangle
\end{align}

对电子:$s = 1/2$,$m_s = \pm 1/2$(常记为 $\alpha, \beta$ 或 $\uparrow, \downarrow$)

完整的单电子态需包含自旋:$\psi_{nlm}(r,\theta,\phi) \times \chi_{m_s}$

\subsection{Pauli不相容原理}

\begin{theorem}[Pauli 原理]
两个全同费米子(如电子)不能占据完全相同的量子态。
\end{theorem}

推论:
\begin{itemize}\setlength{\itemsep}{0pt}\setlength{\parskip}{0pt}
\item 每个轨道 $(n,l,m_l)$ 最多容纳 2 个电子(自旋相反)
\item 多电子波函数必须对交换任意两个电子反对称
\end{itemize}

\subsection{量子力学第六假设}

\begin{keypoint}
\textbf{全同粒子假设}:全同粒子体系的波函数必须满足交换对称性要求。

对于费米子(如电子,自旋为半整数):波函数对交换任意两个粒子必须\textbf{反对称}
\begin{equation}
\Psi(\ldots,i,\ldots,j,\ldots) = -\Psi(\ldots,j,\ldots,i,\ldots)
\end{equation}

对于玻色子(自旋为整数):波函数必须对称。
\end{keypoint}

这个假设直接导出Pauli不相容原理:若两个费米子处于相同态,则波函数为零。

\subsection{Slater行列式}

满足反对称性的 $N$ 电子波函数:
\begin{equation}
\Psi = \frac{1}{\sqrt{N!}}\begin{vmatrix}
\psi_1(1) & \psi_2(1) & \cdots & \psi_N(1) \\
\psi_1(2) & \psi_2(2) & \cdots & \psi_N(2) \\
\vdots & \vdots & \ddots & \vdots \\
\psi_1(N) & \psi_2(N) & \cdots & \psi_N(N)
\end{vmatrix}
\end{equation}

性质:
\begin{itemize}\setlength{\itemsep}{0pt}\setlength{\parskip}{0pt}
\item 交换任意两个电子(行交换)改变符号
\item 两个电子占据同一轨道时行列式为零
\item 自动满足归一化(若单电子轨道归一化)
\end{itemize}

\section{原子光谱项}

\subsection{角动量耦合}

多电子原子的总角动量:
\begin{align}
\vec{L} &= \sum_i \vec{l}_i, \quad L = |l_1+l_2|, \ldots, |l_1 - l_2| \\
\vec{S} &= \sum_i \vec{s}_i, \quad S = s_1 + s_2, \ldots, |s_1 - s_2|
\end{align}

\textbf{LS 耦合}(Russell-Saunders 耦合):先耦合轨道角动量和自旋角动量,再耦合得总角动量:
\begin{equation}
\vec{J} = \vec{L} + \vec{S}, \quad J = L+S, L+S-1, \ldots, |L-S|
\end{equation}

\subsection{光谱项符号}

光谱项记为:$^{2S+1}L_J$

其中:
\begin{itemize}\setlength{\itemsep}{0pt}\setlength{\parskip}{0pt}
\item $2S+1$:自旋多重度
\item $L$:用 S, P, D, F, G, H, $\ldots$ 表示 $L = 0, 1, 2, 3, 4, 5, \ldots$
\item $J$:总角量子数下标
\end{itemize}

\subsection{写出光谱项的方法}

\textbf{步骤}:
\begin{enumerate}\setlength{\itemsep}{0pt}\setlength{\parskip}{0pt}
\item 确定可能的 $L$ 值(轨道角动量耦合)
\item 确定可能的 $S$ 值(自旋角动量耦合)
\item 对每组 $(L,S)$,确定 $J$ 值:$J = L+S, \ldots, |L-S|$
\item 考虑 Pauli 原理排除不可能的组合
\end{enumerate}

\begin{exampleblock}{例题1:$p^1$ 单电子组态}
$l = 1, s = 1/2$

$L = 1$ (P),$S = 1/2$

$J = 1 + 1/2 = 3/2$ 或 $|1 - 1/2| = 1/2$

光谱项:$^2P_{3/2}, ^2P_{1/2}$
\end{exampleblock}

\begin{exampleblock}{例题2:$p^2$ 等价电子}
两个 $p$ 电子:$l_1 = l_2 = 1$

\textbf{轨道角动量耦合}:$L = 2, 1, 0$ (D, P, S)

\textbf{自旋角动量耦合}:$S = 1$ (平行) 或 $0$ (反平行)

可能的微观状态组合(考虑Pauli原理):

\begin{itemize}\setlength{\itemsep}{0pt}
\item $^3P$:$L=1, S=1$,$J=2,1,0$ $\Rightarrow$ $^3P_2, ^3P_1, ^3P_0$
\item $^1D$:$L=2, S=0$,$J=2$ $\Rightarrow$ $^1D_2$
\item $^1S$:$L=0, S=0$,$J=0$ $\Rightarrow$ $^1S_0$
\end{itemize}

基态(Hund规则):$^3P_0$
\end{exampleblock}

\begin{exampleblock}{例题3:$d^2$ 等价电子}
两个 $d$ 电子:$l_1 = l_2 = 2$

\textbf{轨道角动量}:$L = 4, 3, 2, 1, 0$ (G, F, D, P, S)

\textbf{自旋}:$S = 1, 0$

考虑Pauli原理后,实际存在的项(部分):

\begin{itemize}\setlength{\itemsep}{0pt}
\item $^3F$:$L=3, S=1$,$J=4,3,2$ $\Rightarrow$ $^3F_4, ^3F_3, ^3F_2$
\item $^3P$:$L=1, S=1$,$J=2,1,0$ $\Rightarrow$ $^3P_2, ^3P_1, ^3P_0$
\item $^1G$:$L=4, S=0$,$J=4$ $\Rightarrow$ $^1G_4$
\item $^1D$:$L=2, S=0$,$J=2$ $\Rightarrow$ $^1D_2$
\item $^1S$:$L=0, S=0$,$J=0$ $\Rightarrow$ $^1S_0$
\end{itemize}

基态(Hund规则):$^3F_2$ ($S$最大,$L$最大,未满半壳层取$J=|L-S|$)
\end{exampleblock}

\begin{exampleblock}{例题4:$p^1d^1$ 非等价电子}
$p$ 电子:$l_1 = 1$,$d$ 电子:$l_2 = 2$

\textbf{轨道角动量}:
\begin{equation*}
L = |l_1 + l_2|, \ldots, |l_1 - l_2| = 3, 2, 1 \quad \text{(F, D, P)}
\end{equation*}

\textbf{自旋}:两个电子,$S = 1$ 或 $0$

\textbf{所有可能的光谱项}:
\begin{itemize}\setlength{\itemsep}{0pt}
\item $L=3, S=1$:$^3F_{4,3,2}$
\item $L=3, S=0$:$^1F_3$
\item $L=2, S=1$:$^3D_{3,2,1}$
\item $L=2, S=0$:$^1D_2$
\item $L=1, S=1$:$^3P_{2,1,0}$
\item $L=1, S=0$:$^1P_1$
\end{itemize}

非等价电子无Pauli限制,所有组合都存在。

基态(Hund规则):$^3F_2$ ($S=1$最大,$L=3$最大,未满半壳层$J=|3-1|=2$)
\end{exampleblock}

\begin{exampleblock}{例题5:$d^5$ 半满壳层}
五个 $d$ 电子占据五个不同的 $m_l$ 轨道,自旋全部平行。

电子组态:$(m_l = 2,1,0,-1,-2$,全部$\uparrow)$

\textbf{轨道角动量}:$\sum m_l = 0$,故 $M_L = 0$,且只能是 $L = 0$ (S)

\textbf{自旋}:五个电子全平行,$M_S = 5/2$,$S = 5/2$

$2S+1 = 6$ (六重态)

光谱项:$^6S$,$J = S = 5/2$

基态:$^6S_{5/2}$

这是球对称态,非常稳定(如Mn²⁺, Fe³⁺)。
\end{exampleblock}

\subsection{Hund规则}

确定基态光谱项:
\begin{enumerate}\setlength{\itemsep}{0pt}\setlength{\parskip}{0pt}
\item \textbf{最大自旋多重度}:$2S+1$ 最大的项能量最低
\item \textbf{最大总轨道角动量}:$S$ 相同时,$L$ 最大的项能量最低
\item \textbf{总角量子数}:
\begin{itemize}\setlength{\itemsep}{0pt}
\item 未满半壳层:$J = |L - S|$ 最低
\item 超过半壳层:$J = L + S$ 最低
\end{itemize}
\end{enumerate}

\begin{exampleblock}{应用 Hund 规则}
碳原子 $2p^2$ 的基态项:

按 Hund 规则:
\begin{enumerate}\setlength{\itemsep}{0pt}
\item $S = 1$ 的 $^3P$ 最低
\item $L = 1$ 已是最大
\item 未满半壳层($p^2 < p^3$),$J = |1-1| = 0$
\end{enumerate}

基态光谱项:$^3P_0$
\end{exampleblock}

\section{精细结构与自旋-轨道耦合}

\subsection{自旋-轨道相互作用}

电子自旋磁矩与轨道磁矩相互作用:
\begin{equation}
\hat{H}_{SO} = \zeta(r) \vec{L} \cdot \vec{S}
\end{equation}

其中 $\zeta(r) \propto Z^4$,重原子效应更显著。

导致能级分裂:
\begin{equation}
\Delta E_{SO} = \frac{1}{2}\zeta [J(J+1) - L(L+1) - S(S+1)]
\end{equation}

\begin{exampleblock}{钠 D 双线}
钠原子 $3p$ 态由于自旋-轨道耦合分裂:

$^2P \to ^2P_{3/2}$ 和 $^2P_{1/2}$

$3s \, ^2S_{1/2} \to 3p \, ^2P_{1/2}$:589.6 nm(D$_2$)

$3s \, ^2S_{1/2} \to 3p \, ^2P_{3/2}$:589.0 nm(D$_1$)

分裂间隔:$\Delta\tilde{\nu} \approx 17$ cm$^{-1}$
\end{exampleblock}

\section{Hartree-Fock 自洽场方法}

\subsection{基本思想}

每个电子在原子核和其他电子平均场中运动:
\begin{equation}
\hat{h}(i)\psi_i(i) = \varepsilon_i\psi_i(i)
\end{equation}

有效哈密顿算符:
\begin{equation}
\hat{h}(i) = -\frac{1}{2}\nabla_i^2 - \frac{Z}{r_i} + V_{\text{eff}}(i)
\end{equation}

其中有效势 $V_{\text{eff}}$ 由其他电子密度决定。

\subsection{自洽迭代过程}

\begin{enumerate}\setlength{\itemsep}{0pt}\setlength{\parskip}{0pt}
\item 假设初始轨道 $\{\psi_i^{(0)}\}$
\item 计算有效势 $V_{\text{eff}}^{(1)}$
\item 解单电子方程得新轨道 $\{\psi_i^{(1)}\}$
\item 检查收敛性:$|\psi_i^{(n)} - \psi_i^{(n-1)}| < \epsilon$
\item 若未收敛,返回步骤 2
\end{enumerate}

\subsection{Koopmans 定理}

\begin{theorem}[Koopmans 定理]
原子的第一电离能近似等于最高占据轨道能量的负值:
\begin{equation}
IE \approx -\varepsilon_{\text{HOMO}}
\end{equation}
\end{theorem}

\begin{exampleblock}{例题:估算电离能}
氖原子 HF 计算:$\varepsilon_{2p} = -0.85$ a.u.

$IE \approx 0.85 \times 27.211 = 23.1$ eV

实验值:21.6 eV(误差约 7\%)
\end{exampleblock}

\section{周期表与原子性质}

\subsection{原子半径趋势}

\textbf{同周期}:从左到右,$Z_{\text{eff}}$ 增大,半径减小

\textbf{同族}:从上到下,$n$ 增大,半径增大

例外:镧系收缩(4f 电子屏蔽效应弱)

\subsection{电离能}

第一电离能 $IE_1$:移除一个电子所需能量

\textbf{趋势}:
\begin{itemize}\setlength{\itemsep}{0pt}\setlength{\parskip}{0pt}
\item 同周期:总体增大(惰性气体最大)
\item 同族:向下减小
\item 半满或全满壳层电离能较高
\end{itemize}

\subsection{电子亲和能}

电子亲和能 $EA$:原子获得一个电子释放的能量

\textbf{趋势}:
\begin{itemize}\setlength{\itemsep}{0pt}\setlength{\parskip}{0pt}
\item 卤素最大
\item 惰性气体为负(不稳定)
\item 半满壳层(如 N)较小
\end{itemize}

\section{常用公式速查}

\begin{scriptsize}
\begin{itemize}\setlength{\itemsep}{0pt}
\item 原子单位能量:$1 E_h = 27.211$ eV $= 2.1947 \times 10^5$ cm$^{-1}$
\item 有效核电荷:$Z_{\text{eff}} = Z - \sigma$
\item 轨道能量:$E_{nl} \approx -\frac{Z_{\text{eff}}^2}{n^2} \times 13.6$ eV
\item 自旋多重度:$2S+1$
\item 角动量耦合:$L = l_1+l_2, \ldots, |l_1-l_2|$;$S = s_1+s_2, \ldots, |s_1-s_2|$
\item 总角动量:$J = L+S, \ldots, |L-S|$
\item 精细结构分裂:$\Delta E_{SO} = \frac{\zeta}{2}[J(J+1)-L(L+1)-S(S+1)]$
\item 光谱项:$^{2S+1}L_J$(L用SPDFGH表示)
\item Slater 行列式:$\Psi = \frac{1}{\sqrt{N!}}\det[\psi_i(j)]$
\item Hund规则:(1)最大$S$ (2)最大$L$ (3)半满前$J=|L-S|$,半满后$J=L+S$
\end{itemize}
\end{scriptsize}

\end{multicols}

\end{document}

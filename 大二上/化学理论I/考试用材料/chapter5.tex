% 编译提示:使用 XeLaTeX 编译
% 需要 ctex 宏包支持中文

\documentclass[a4paper,9pt]{article}
\usepackage{amsmath}
\usepackage{amssymb}
\usepackage{geometry}
\usepackage{ctex}
\usepackage{xcolor}
\usepackage{multicol}
\usepackage{titlesec}
\usepackage{fancyhdr}
\usepackage{tcolorbox}
\usepackage{amsthm}

\geometry{left=1cm,right=1cm,top=1cm,bottom=1.8cm}

\setlength{\columnsep}{0.8cm}
\setlength{\columnseprule}{0.4pt}
\renewcommand{\columnseprulecolor}{\color{gray!50}}

% 定义颜色方案
\definecolor{sectioncolor}{RGB}{0,102,204}      % 深蓝色 - 用于章节标题
\definecolor{subsectioncolor}{RGB}{0,153,153}   % 青色 - 用于小节标题
\definecolor{keycolor}{RGB}{204,102,0}          % 橙色 - 用于关键点
\definecolor{derivcolor}{RGB}{100,149,237}      % 浅蓝色 - 用于推导框
\definecolor{boxbg}{RGB}{240,248,255}           % 淡蓝色背景
\definecolor{keybg}{RGB}{255,248,240}           % 淡橙色背景

% 自定义章节标题格式
\titleformat{\section}
  {\normalfont\large\bfseries\color{sectioncolor}}
  {\thesection}{0.5em}{}
\titlespacing*{\section}{0pt}{1ex plus 0.5ex minus 0.2ex}{0.8ex plus 0.2ex}

\titleformat{\subsection}
  {\normalfont\normalsize\bfseries\color{subsectioncolor}}
  {\thesubsection}{0.5em}{}
\titlespacing*{\subsection}{0pt}{0.8ex plus 0.3ex minus 0.1ex}{0.5ex plus 0.1ex}

% 自定义定理环境
\newtheorem{theorem}{原理}
\newtheorem{proofpart}{证明思路}
\renewcommand{\theproofpart}{}

% 自定义关键点环境
\newenvironment{keypoint}
{\par\vspace{0.3em}\noindent\begin{tcolorbox} [
  colback=keybg,
  colframe=keycolor,
  boxrule=0.5pt,
  arc=2pt,
  left=2pt,
  right=2pt,
  top=2pt,
  bottom=2pt,
  boxsep=0pt]
\small\noindent\textcolor{keycolor}{\textbf{关键点:}}}
{\end{tcolorbox}\par\vspace{0.3em}}

% 自定义推导环境 - 修复字体和行距
\newenvironment{derivation}
{\par\vspace{0.3em}\noindent\begin{tcolorbox} [
  colback=boxbg,
  colframe=derivcolor,
  boxrule=0.5pt,
  arc=2pt,
  left=2pt,
  right=2pt,
  top=2pt,
  bottom=2pt,
  boxsep=0pt]
\small\noindent\textcolor{derivcolor}{\textbf{推导:}}\par\linespread{1.1}\selectfont}
{\end{tcolorbox}\par\vspace{0.3em}}

% 自定义例题环境 - 修复字体和行距
\newenvironment{exampleblock}[1]
{\par\vspace{0.3em}\noindent\begin{tcolorbox} [
  colback=boxbg,
  colframe=derivcolor,
  boxrule=0.5pt,
  arc=2pt,
  left=2pt,
  right=2pt,
  top=2pt,
  bottom=2pt,
  boxsep=0pt]
\small\noindent\textcolor{derivcolor}{\textbf{#1}}\par\linespread{1.1}\selectfont}
{\end{tcolorbox}\par\vspace{0.3em}}

% 页眉页脚设置
\pagestyle{fancy}
\fancyhf{}
\fancyfoot[C]{\scriptsize5-\thepage}
\renewcommand{\headrulewidth}{0pt}
\renewcommand{\footrulewidth}{0.4pt}
\renewcommand{\footrule}{\hbox to\headwidth{\color{gray!50}\leaders\hrule height \footrulewidth\hfill}}

% 紧凑标题
\title{\vspace{-2em}\Large\bfseries\color{sectioncolor} Chapter 5 Collections - 谐振子与刚性转子\vspace{-1em}}
\author{}
\date{2025 年 11 月 1 日}

\begin{document}

\maketitle
\thispagestyle{fancy}

\begin{multicols}{2}

\section{谐振子}

\subsection{谐振子遵守胡克定律}

如图所示,质量为 $m$ 的小球通过弹簧与壁相连,平衡位置为 $l_0$,弹性系数为 $k$。

胡克定律:$F = -k(l-l_0)$

势能函数(选平衡位置为零点):
\begin{equation}
V(x) = \frac{1}{2}kx^2
\end{equation}

经典能量:$E = T + V = \frac{1}{2}m\dot{x}^2 + \frac{1}{2}kx^2$

角频率:$\omega = \sqrt{k/m}$

\subsection{双原子分子的谐振子模型}

\textbf{质心坐标系}:两个质量为 $m_1, m_2$ 的原子,质心位置固定,相对运动用相对坐标 $x = x_1 - x_2$ 描述。

\textbf{约化质量}:$\mu = \frac{m_1m_2}{m_1+m_2}$

薛定谔方程:
\begin{equation}
-\frac{\hbar^2}{2\mu}\frac{d^2\psi}{dx^2} + \frac{1}{2}kx^2\psi = E\psi
\end{equation}

引入无量纲坐标 $\xi = \sqrt{\frac{k}{\mu}}\left(\frac{\mu}{\hbar}\right)^{1/2}x = \alpha x$,其中 $\alpha = \left(\frac{k\mu}{\hbar^2}\right)^{1/4}$

\subsection{谐振子近似源自核间势能在其最小值附近的展开}

双原子分子的势能曲线 $V(r)$ 在平衡位置 $r_e$ 附近泰勒展开:
\begin{align}
V(r) &= V(r_e) + \left(\frac{dV}{dr}\right)_{r_e}(r-r_e) + \frac{1}{2!}\left(\frac{d^2V}{dr^2}\right)_{r_e}(r-r_e)^2 + \ldots \\
&\approx V(r_e) + \frac{1}{2}k(r-r_e)^2
\end{align}

其中 $k = \left(\frac{d^2V}{dr^2}\right)_{r_e}$ 为力常数,平衡位置满足 $\left(\frac{dV}{dr}\right)_{r_e} = 0$

令 $x = r - r_e$,选零点 $V(r_e) = 0$:
\begin{equation}
V(x) = \frac{1}{2}kx^2
\end{equation}

\subsection{Morse势能函数}

实际分子的势能曲线可用Morse势更精确描述:
\begin{equation}
V(r) = D[1 - e^{-\beta(r-r_e)}]^2
\end{equation}

其中 $D$ 为离解能,$\beta = \sqrt{k/(2D)}$

\textbf{Morse势的能级}:
\begin{equation}
E_v = \hbar\omega\left(v+\frac{1}{2}\right) - \hbar\omega x_e\left(v+\frac{1}{2}\right)^2, \quad v=0,1,2,\ldots
\end{equation}

其中 $x_e = \frac{\hbar\omega}{4D}$ 为非谐性常数。

\begin{keypoint}
Morse势考虑了非谐性,能级间距随 $v$ 增大而减小,更符合实际分子。
\end{keypoint}

\subsection{谐振子的能级为 $E_v = \hbar\omega(v+\frac{1}{2})$}

一维谐振子的能量本征值:
\begin{equation}
E_v = \hbar\omega\left(v+\frac{1}{2}\right) = h\nu\left(v+\frac{1}{2}\right), \quad v = 0,1,2,\ldots
\end{equation}

其中 $\omega = \sqrt{k/\mu}$,$\nu = \omega/(2\pi)$

\textbf{零点能}:基态能量 $E_0 = \frac{1}{2}\hbar\omega \neq 0$

由不确定性原理,粒子不能静止在平衡位置。

\textbf{能级间距}:相邻能级间距为常数
\begin{equation}
\Delta E = E_{v+1} - E_v = \hbar\omega
\end{equation}

波函数:
\begin{equation}
\psi_v(x) = N_v H_v(\alpha x)e^{-\alpha^2x^2/2}
\end{equation}

其中 $H_v$ 为厄米多项式,$N_v$ 为归一化常数。

\begin{keypoint}
量子谐振子具有零点能,能级等间距,这是量子效应的体现。
\end{keypoint}

\subsection{谐振子可解释双原子分子的红外光谱}

\textbf{选择定则}:
\begin{equation}
\Delta v = \pm 1
\end{equation}

对于吸收光谱,$\Delta v = +1$,则:
\begin{equation}
\Delta E = E_{v+1} - E_v = \hbar\omega
\end{equation}

吸收频率:
\begin{equation}
\nu_{abs} = \frac{\Delta E}{h} = \frac{\omega}{2\pi} = \frac{1}{2\pi}\sqrt{\frac{k}{\mu}}
\end{equation}

波数:
\begin{equation}
\tilde{\nu}_{abs} = \frac{\nu}{c} = \frac{1}{2\pi c}\sqrt{\frac{k}{\mu}}
\end{equation}

\begin{derivation}
选择定则来源于跃迁偶极矩:
\begin{equation}
\mu_{v'v} = \int \psi_{v'}^* \mu(x) \psi_v dx
\end{equation}

偶极矩展开:$\mu(x) = \mu_0 + \left(\frac{d\mu}{dx}\right)_0 x + \ldots$

只有 $\Delta v = \pm 1$ 时,积分 $\int \psi_{v'}^* x \psi_v dx$ 不为零。

注:只有具有永久偶极矩或其导数不为零的分子才能产生红外吸收光谱。同核双原子分子(如H$_2$, N$_2$)没有红外吸收光谱。
\end{derivation}

\begin{exampleblock}{例题:$^{79}$Br$^{19}$F的红外光谱}
$^{79}$Br$^{19}$F在红外光谱中有一条强吸收峰在380 cm$^{-1}$,计算力常数。

\textbf{解}:
\begin{align*}
\mu &= \frac{(78.9)(19.0)}{(78.9+19.0)} \times 1.661 \times 10^{-27} \text{ kg} = 2.52 \times 10^{-26} \text{ kg} \\
k &= [2\pi(2.998 \times 10^{10})(380)]^2 \times (2.52 \times 10^{-26}) \\
&= 129 \text{ kg·s}^{-2} = 129 \text{ N·m}^{-1}
\end{align*}
\end{exampleblock}

\subsection{维里定理}

对于谐振子,动能和势能的期望值相等:
\begin{equation}
\langle T \rangle = \langle V \rangle = \frac{1}{2}E_v
\end{equation}

证明:利用 $\langle \hat{p}^2/(2\mu) \rangle$ 和 $\langle kx^2/2 \rangle$ 的对称性。

\subsection{常见双原子分子的振动光谱数据}

\begin{scriptsize}
\begin{tabular}{cccc}
\hline
分子 & $\tilde{\nu}/\text{cm}^{-1}$ & $k/(\text{N·m}^{-1})$ & $r_e/\text{pm}$ \\
\hline
H$_2$ & 4401 & 570 & 74.1 \\
D$_2$ & 2990 & 527 & 74.1 \\
H$^{81}$Br & 2630 & 408 & 141.4 \\
$^{35}$Cl$^{37}$Cl & 554 & 319 & 198.8 \\
$^{14}$N$^{14}$N & 2330 & 2243 & 109.4 \\
\hline
\end{tabular}
\end{scriptsize}

\section{多维谐振子}

\subsection{二维各向同性谐振子}

哈密顿算符可分离:
\begin{equation}
\hat{H} = \hat{H}_x + \hat{H}_y
\end{equation}

能量本征值:
\begin{equation}
E_{v_x,v_y} = \hbar\omega\left(v_x + v_y + 1\right) = \hbar\omega(v+1)
\end{equation}

其中 $v = v_x + v_y = 0,1,2,\ldots$

\textbf{简并度}:能级 $E_v$ 的简并度为 $g_v = v+1$

\subsection{三维各向同性谐振子}

能量本征值:
\begin{equation}
E_{v_x,v_y,v_z} = \hbar\omega\left(v_x+v_y+v_z+\frac{3}{2}\right) = \hbar\omega\left(v+\frac{3}{2}\right)
\end{equation}

\textbf{简并度}:$g_v = \frac{(v+1)(v+2)}{2}$

\section{刚性转子}

\subsection{模型与哈密顿算符}

两个质量为 $m_1, m_2$ 的质点,通过固定长度 $r_0$ 的刚性杆连接。

转动惯量:$I = \mu r_0^2$,$\mu = \frac{m_1m_2}{m_1+m_2}$

哈密顿算符:
\begin{equation}
\hat{H} = \frac{\hat{L}^2}{2I}
\end{equation}

\subsection{刚性转子的能级是 $E_J = \hbar^2J(J+1)/(2I)$}

能量本征值:
\begin{equation}
E_J = \frac{\hbar^2}{2I}J(J+1), \quad J = 0,1,2,\ldots
\end{equation}

用转动常数表示:
\begin{equation}
B = \frac{\hbar}{4\pi cI} \quad (\text{cm}^{-1})
\end{equation}

\begin{equation}
E_J = hcBJ(J+1)
\end{equation}

\textbf{简并度}:$g_J = 2J+1$(对应 $M = -J, \ldots, J$)

\begin{keypoint}
刚性转子没有零点能,$E_0 = 0$。能级间距随 $J$ 增大而增大。
\end{keypoint}

\subsection{刚性转子是旋转双原子分子的一个模型}

本征函数为球谐函数 $Y_J^M(\theta,\phi)$

能级间距:
\begin{align}
\Delta E &= E_{J+1} - E_J = \frac{\hbar^2}{2I}[(J+1)(J+2) - J(J+1)] \\
&= \frac{\hbar^2}{I}(J+1) = hcB \cdot 2(J+1)
\end{align}

\subsection{微波光谱选择定则}

纯转动光谱(微波光谱)的选择定则:
\begin{equation}
\Delta J = \pm 1
\end{equation}

吸收谱线($\Delta J = +1$):
\begin{equation}
\tilde{\nu}_J = 2B(J+1), \quad J=0,1,2,\ldots
\end{equation}

\textbf{谱线特征}:等间距谱线,间距为 $2B$

\begin{exampleblock}{例题:H$^{35}$Cl微波光谱}
根据教材式(5.61),H$^{35}$Cl微波光谱中的谱线间距为 $2B(Hz)$ 或 $2\tilde{B}(cm^{-1})$。

若 $\tilde{\nu} = 2886$ cm$^{-1}$,则 $\tilde{B} = \tilde{\nu}/(4\pi) \approx 10.4$ cm$^{-1}$

转动惯量:
\begin{equation*}
I = \frac{h}{8\pi^2c\tilde{B}} = \frac{6.626 \times 10^{-34}}{8\pi^2(2.998 \times 10^{10})(10.4)} \approx 2.7 \times 10^{-47} \text{ kg·m}^2
\end{equation*}

利用 $I = \mu r_0^2$ 可求键长 $r_0$。
\end{exampleblock}

\subsection{离心畸变}

高速转动时,离心力使键长增大,修正能量:
\begin{equation}
E_J = hc[BJ(J+1) - D_JJ^2(J+1)^2]
\end{equation}

$D_J$ 为离心畸变常数(很小的正数)。

\section{常用公式速查}

\begin{scriptsize}
\begin{itemize}\setlength{\itemsep}{0pt}
\item \textbf{谐振子}
\begin{itemize}\setlength{\itemsep}{0pt}
\item 能量:$E_v = (v+\frac{1}{2})\hbar\omega$,$\omega = \sqrt{k/\mu}$
\item 零点能:$E_0 = \frac{1}{2}\hbar\omega$
\item 选择定则:$\Delta v = \pm 1$
\item 维里定理:$\langle T \rangle = \langle V \rangle = E_v/2$
\item 吸收频率:$\tilde{\nu} = \frac{1}{2\pi c}\sqrt{k/\mu}$
\item Morse势:$E_v = \hbar\omega(v+\frac{1}{2}) - \hbar\omega x_e(v+\frac{1}{2})^2$
\item 力常数:$k = \left(\frac{d^2V}{dr^2}\right)_{r_e}$
\end{itemize}
\item \textbf{刚性转子}
\begin{itemize}\setlength{\itemsep}{0pt}
\item 能量:$E_J = \frac{\hbar^2}{2I}J(J+1) = hcBJ(J+1)$
\item 转动常数:$B = \frac{\hbar}{4\pi cI}$ (cm$^{-1}$)
\item 简并度:$g_J = 2J+1$
\item 选择定则:$\Delta J = \pm 1$
\item 谱线波数:$\tilde{\nu} = 2B(J+1)$,间距 $2B$
\item 转动惯量:$I = \mu r_0^2$,$\mu = m_1m_2/(m_1+m_2)$
\end{itemize}
\item \textbf{多维谐振子简并度}
\begin{itemize}\setlength{\itemsep}{0pt}
\item 2D:$g_v = v+1$
\item 3D:$g_v = (v+1)(v+2)/2$
\end{itemize}
\end{itemize}
\end{scriptsize}

\end{multicols}

\end{document}
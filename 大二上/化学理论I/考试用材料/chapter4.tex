% 编译提示:使用 XeLaTeX 编译
% 需要 ctex 宏包支持中文

\documentclass[a4paper,9pt]{article}
\usepackage{amsmath}
\usepackage{amssymb}
\usepackage{geometry}
\usepackage{ctex}
\usepackage{xcolor}
\usepackage{multicol}
\usepackage{titlesec}
\usepackage{fancyhdr}
\usepackage{tcolorbox}
\usepackage{amsthm}

\geometry{left=1cm,right=1cm,top=1cm,bottom=1.8cm}

\setlength{\columnsep}{0.8cm}
\setlength{\columnseprule}{0.4pt}
\renewcommand{\columnseprulecolor}{\color{gray!50}}

% 定义颜色方案
\definecolor{sectioncolor}{RGB}{0,102,204}
\definecolor{subsectioncolor}{RGB}{0,153,153}
\definecolor{keycolor}{RGB}{204,102,0}
\definecolor{derivcolor}{RGB}{100,149,237}
\definecolor{boxbg}{RGB}{240,248,255}
\definecolor{keybg}{RGB}{255,248,240}

% 自定义章节标题格式
\titleformat{\section}
  {\normalfont\large\bfseries\color{sectioncolor}}
  {\thesection}{0.5em}{}
\titlespacing*{\section}{0pt}{1ex plus 0.5ex minus 0.2ex}{0.8ex plus 0.2ex}

\titleformat{\subsection}
  {\normalfont\normalsize\bfseries\color{subsectioncolor}}
  {\thesubsection}{0.5em}{}
\titlespacing*{\subsection}{0pt}{0.8ex plus 0.3ex minus 0.1ex}{0.5ex plus 0.1ex}

% 自定义定理环境
\newtheorem{theorem}{假设}

% 自定义关键点环境
\newenvironment{keypoint}
{\par\vspace{0.3em}\noindent\begin{tcolorbox}[
  colback=keybg,
  colframe=keycolor,
  boxrule=0.5pt,
  arc=2pt,
  left=2pt,
  right=2pt,
  top=2pt,
  bottom=2pt,
  boxsep=0pt]
\small\noindent\textcolor{keycolor}{\textbf{关键点:}}}
{\end{tcolorbox}\par\vspace{0.3em}}

% 自定义推导环境
\newenvironment{derivation}
{\par\vspace{0.3em}\noindent\begin{tcolorbox}[
  colback=boxbg,
  colframe=derivcolor,
  boxrule=0.5pt,
  arc=2pt,
  left=2pt,
  right=2pt,
  top=2pt,
  bottom=2pt,
  boxsep=0pt]
\small\noindent\textcolor{derivcolor}{\textbf{推导:}}\par\linespread{1.1}\selectfont}
{\end{tcolorbox}\par\vspace{0.3em}}

% 自定义例题环境
\newenvironment{exampleblock}[1]
{\par\vspace{0.3em}\noindent\begin{tcolorbox}[
  colback=boxbg,
  colframe=derivcolor,
  boxrule=0.5pt,
  arc=2pt,
  left=2pt,
  right=2pt,
  top=2pt,
  bottom=2pt,
  boxsep=0pt]
\small\noindent\textcolor{derivcolor}{\textbf{#1}}\par\linespread{1.1}\selectfont}
{\end{tcolorbox}\par\vspace{0.3em}}

% 页眉页脚设置
\pagestyle{fancy}
\fancyhf{}
\fancyfoot[C]{\scriptsize4-\thepage}
\renewcommand{\headrulewidth}{0pt}
\renewcommand{\footrulewidth}{0.4pt}
\renewcommand{\footrule}{\hbox to\headwidth{\color{gray!50}\leaders\hrule height \footrulewidth\hfill}}

% 紧凑标题
\title{\vspace{-2em}\Large\bfseries\color{sectioncolor} Chapter 4 Collections - 量子力学基本假设\vspace{-1em}}
\author{}
\date{2025 年 11 月 1 日}

\begin{document}

\maketitle
\thispagestyle{fancy}

\begin{multicols}{2}

\section{量子力学基本假设}

\subsection{假设1:波函数}

\begin{theorem}[波函数]
一个量子力学体系的状态由一个称为波函数的函数 $\Psi(\vec{r}_1, \vec{r}_2, \ldots, t)$ 完全描述。
\end{theorem}

\textbf{波函数的性质}(良好行为):
\begin{itemize}\setlength{\itemsep}{0pt}\setlength{\parskip}{0pt}
\item \textbf{单值性}:在任意点,$\Psi$ 只有一个值。
\item \textbf{连续性}:$\Psi$ 及其一阶导数连续。
\item \textbf{平方可积}:$\int |\Psi|^2 d\tau$ 必须有限。
\end{itemize}

\textbf{Born概率诠释}:
\begin{equation}
P(V) = \int_V |\Psi|^2 d\tau
\end{equation}
表示在体积元 $V$ 内发现粒子的概率。

\textbf{归一化条件}:
\begin{equation}
\int_{\text{全空间}} |\Psi|^2 d\tau = 1
\end{equation}

\subsection{假设2:算符}

\begin{theorem}[算符]
每一个经典力学中的可观测量都对应量子力学中的一个线性、厄米算符。
\end{theorem}

\textbf{线性算符}:$\hat{A}(c_1f_1 + c_2f_2) = c_1\hat{A}f_1 + c_2\hat{A}f_2$

\textbf{厄米算符}:$\int f^* (\hat{A}g) d\tau = \int (\hat{A}f)^* g d\tau$

\begin{keypoint}
厄米算符的本征值必为实数,这保证了物理量的测量结果是实数。
\end{keypoint}

\begin{exampleblock}{常用算符}
\begin{center}
\begin{tabular}{ccc}
\hline
可观测量 & 经典量 & 量子算符 \\
\hline
位置 & $x$ & $\hat{x} = x$ \\
动量 & $p_x$ & $\hat{p}_x = -i\hbar\frac{\partial}{\partial x}$ \\
动能 & $T = \frac{p^2}{2m}$ & $\hat{T} = -\frac{\hbar^2}{2m}\nabla^2$ \\
势能 & $V(\vec{r})$ & $\hat{V}(\vec{r}) = V(\vec{r})$ \\
总能量 & $H = T+V$ & $\hat{H} = -\frac{\hbar^2}{2m}\nabla^2 + V$ \\
角动量 & $L_z = xp_y - yp_x$ & $\hat{L}_z = -i\hbar\frac{\partial}{\partial\phi}$ \\
\hline
\end{tabular}
\end{center}
\end{exampleblock}

\subsection{假设3:本征值}

\begin{theorem}[本征值]
对一个物理量进行测量,唯一可能得到的结果是其对应算符的本征值。
\end{theorem}

本征方程:
\begin{equation}
\hat{A}\psi_n = a_n\psi_n
\end{equation}

\begin{itemize}\setlength{\itemsep}{0pt}\setlength{\parskip}{0pt}
\item $a_n$:本征值(测量结果)
\item $\psi_n$:本征函数(测量后体系所处的态)
\end{itemize}

\textbf{厄米算符本征函数性质}:
\begin{itemize}\setlength{\itemsep}{0pt}\setlength{\parskip}{0pt}
\item 不同本征值对应的本征函数相互正交。
\item 本征函数系构成一个完备集。
\end{itemize}

\subsection{假设4:期望值}

\begin{theorem}[期望值]
对于处于归一化态 $\Psi$ 的体系,物理量 $A$ 的多次测量平均值(期望值)为:
\begin{equation}
\langle A \rangle = \int \Psi^* \hat{A} \Psi \, d\tau
\end{equation}
\end{theorem}

如果 $\Psi$ 是 $\hat{A}$ 的本征函数,$\hat{A}\Psi = a\Psi$,则:
\begin{equation}
\langle A \rangle = a \int \Psi^* \Psi \, d\tau = a
\end{equation}
此时测量结果是确定的。

\begin{exampleblock}{例题:一维势箱中的动量期望值}
对于一维势箱中的态 $\psi_n = \sqrt{\frac{2}{a}}\sin(\frac{n\pi x}{a})$:

\textbf{动量期望值}:
\begin{align*}
\langle p_x \rangle &= \int_0^a \psi_n^* \left(-i\hbar\frac{d}{dx}\right) \psi_n dx \\
&= -\frac{2i\hbar}{a}\frac{n\pi}{a} \int_0^a \sin\left(\frac{n\pi x}{a}\right)\cos\left(\frac{n\pi x}{a}\right) dx \\
&= 0
\end{align*}

\textbf{动量平方期望值}:
\begin{align*}
\langle p_x^2 \rangle &= \int_0^a \psi_n^* \left(-\hbar^2\frac{d^2}{dx^2}\right) \psi_n dx \\
&= \frac{n^2\pi^2\hbar^2}{a^2} \int_0^a \psi_n^* \psi_n dx = \frac{n^2\pi^2\hbar^2}{a^2}
\end{align*}
\end{exampleblock}

\subsection{假设5:时间演化}

\begin{theorem}[时间演化]
体系的波函数 $\Psi$ 随时间的演化由含时薛定谔方程决定:
\begin{equation}
i\hbar\frac{\partial\Psi}{\partial t} = \hat{H}\Psi
\end{equation}
\end{theorem}

对于定态问题($\hat{H}$ 不含时),解的形式为:
\begin{equation}
\Psi(\vec{r},t) = \psi(\vec{r})e^{-iEt/\hbar}
\end{equation}
其中 $\psi(\vec{r})$ 满足定态薛定谔方程 $\hat{H}\psi = E\psi$。

\subsection{假设6:全同粒子}

\begin{theorem}[全同粒子]
由全同粒子组成的体系,其总波函数在交换任意两个粒子坐标时必须满足对称性要求。
\end{theorem}

\begin{itemize}\setlength{\itemsep}{0pt}\setlength{\parskip}{0pt}
\item \textbf{费米子}(半整数自旋,如电子):波函数是\textbf{反对称}的。
\begin{equation}
\Psi(\ldots,i,\ldots,j,\ldots) = -\Psi(\ldots,j,\ldots,i,\ldots)
\end{equation}
\item \textbf{玻色子}(整数自旋,如光子):波函数是\textbf{对称}的。
\end{itemize}

\begin{keypoint}
Pauli不相容原理是反对称性假设的直接推论:如果两个电子处于完全相同的状态,则交换它们后波函数不变,但反对称性要求符号相反,唯一可能是波函数为零,即这种状态不存在。
\end{keypoint}

\section{算符的对易关系}

\subsection{对易子与不确定性}

\textbf{对易子}定义:
\begin{equation}
[\hat{A},\hat{B}] = \hat{A}\hat{B} - \hat{B}\hat{A}
\end{equation}

\begin{itemize}\setlength{\itemsep}{0pt}\setlength{\parskip}{0pt}
\item 若 $[\hat{A},\hat{B}] = 0$,则 $\hat{A}$ 和 $\hat{B}$ 对应的物理量可以同时有确定值,称它们是\textbf{相容的}。
\item 若 $[\hat{A},\hat{B}] \neq 0$,则它们是\textbf{不相容的},其测量值遵循不确定性原理。
\end{itemize}

\textbf{广义不确定性原理}:
\begin{equation}
\Delta A \cdot \Delta B \ge \frac{1}{2} \left| \langle[\hat{A},\hat{B}]\rangle \right|
\end{equation}

\subsection{基本对易关系}

\textbf{正则对易关系}:
\begin{equation}
[\hat{x}_i, \hat{p}_j] = i\hbar\delta_{ij}
\end{equation}

\textbf{角动量对易关系}:
\begin{align}
[\hat{L}_x, \hat{L}_y] &= i\hbar\hat{L}_z \quad (\text{及循环置换}) \\
[\hat{L}^2, \hat{L}_i] &= 0
\end{align}

\begin{derivation}
证明 $[\hat{x}, \hat{p}_x^2] = 2i\hbar\hat{p}_x$:

利用对易子恒等式 $[\hat{A},\hat{B}\hat{C}] = [\hat{A},\hat{B}]\hat{C} + \hat{B}[\hat{A},\hat{C}]$:
\begin{align*}
[\hat{x}, \hat{p}_x^2] &= [\hat{x}, \hat{p}_x]\hat{p}_x + \hat{p}_x[\hat{x}, \hat{p}_x] \\
&= (i\hbar)\hat{p}_x + \hat{p}_x(i\hbar) \\
&= 2i\hbar\hat{p}_x
\end{align*}
\end{derivation}

\begin{exampleblock}{例题:常用对易子计算}
\textbf{示例1}:计算 $[\hat{x}, \hat{H}]$,其中 $\hat{H} = \frac{\hat{p}_x^2}{2m} + V(\hat{x})$

\begin{align*}
[\hat{x}, \hat{H}] &= [\hat{x}, \frac{\hat{p}_x^2}{2m}] + [\hat{x}, V(\hat{x})] \\
&= \frac{1}{2m}[\hat{x}, \hat{p}_x^2] + 0 \\
&= \frac{1}{2m} \cdot 2i\hbar\hat{p}_x = \frac{i\hbar\hat{p}_x}{m}
\end{align*}

\textbf{示例2}:计算 $[\hat{p}_x, \hat{x}^2]$

\begin{align*}
[\hat{p}_x, \hat{x}^2] &= [\hat{p}_x, \hat{x}]\hat{x} + \hat{x}[\hat{p}_x, \hat{x}] \\
&= (-i\hbar)\hat{x} + \hat{x}(-i\hbar) \\
&= -2i\hbar\hat{x}
\end{align*}

\textbf{规律}:$[\hat{x}, \hat{p}_x^n] = ni\hbar\hat{p}_x^{n-1}$,$[\hat{p}_x, \hat{x}^n] = -ni\hbar\hat{x}^{n-1}$
\end{exampleblock}

\subsection{对易子的重要恒等式}

\begin{keypoint}
\textbf{雅可比恒等式}:
\begin{equation}
[\hat{A},[\hat{B},\hat{C}]] + [\hat{B},[\hat{C},\hat{A}]] + [\hat{C},[\hat{A},\hat{B}]] = 0
\end{equation}

\textbf{乘积对易式}:
\begin{align}
[\hat{A},\hat{B}\hat{C}] &= [\hat{A},\hat{B}]\hat{C} + \hat{B}[\hat{A},\hat{C}] \\
[\hat{A}\hat{B},\hat{C}] &= \hat{A}[\hat{B},\hat{C}] + [\hat{A},\hat{C}]\hat{B}
\end{align}
\end{keypoint}

\section{期望值的时间演化}

\subsection{海森堡运动方程}

任意算符 $\hat{A}$ 的期望值随时间的变化率:
\begin{equation}
\frac{d\langle A \rangle}{dt} = \frac{1}{i\hbar}\langle[\hat{A},\hat{H}]\rangle + \left\langle\frac{\partial\hat{A}}{\partial t}\right\rangle
\end{equation}

\begin{keypoint}
如果算符 $\hat{A}$ 不显含时间且与哈密顿算符 $\hat{H}$ 对易,则 $\langle A \rangle$ 是守恒量。
\end{keypoint}

\subsection{埃伦费斯特定理}

\begin{exampleblock}{定理1:位置期望值的演化}
\begin{equation}
\frac{d\langle x \rangle}{dt} = \frac{\langle p_x \rangle}{m}
\end{equation}

\textbf{推导}:
\begin{align*}
\frac{d\langle x \rangle}{dt} &= \frac{1}{i\hbar}\langle[\hat{x},\hat{H}]\rangle = \frac{1}{i\hbar}\left\langle\left[\hat{x}, \frac{\hat{p}_x^2}{2m}\right]\right\rangle \\
&= \frac{1}{2mi\hbar}\langle[\hat{x},\hat{p}_x^2]\rangle = \frac{1}{2mi\hbar}\langle 2i\hbar\hat{p}_x \rangle \\
&= \frac{\langle p_x \rangle}{m}
\end{align*}
\end{exampleblock}

\begin{exampleblock}{定理2:动量期望值的演化}
\begin{equation}
\frac{d\langle p_x \rangle}{dt} = \left\langle -\frac{\partial V}{\partial x} \right\rangle
\end{equation}

\textbf{推导}:
\begin{align*}
\frac{d\langle p_x \rangle}{dt} &= \frac{1}{i\hbar}\langle[\hat{p}_x,\hat{H}]\rangle = \frac{1}{i\hbar}\langle[\hat{p}_x, V(x)]\rangle
\end{align*}
作用于波函数 $\psi$:
\begin{align*}
[\hat{p}_x, V]\psi &= -i\hbar\frac{\partial}{\partial x}(V\psi) - V(-i\hbar\frac{\partial\psi}{\partial x}) \\
&= -i\hbar\left(\frac{\partial V}{\partial x}\psi + V\frac{\partial\psi}{\partial x}\right) + i\hbar V\frac{\partial\psi}{\partial x} \\
&= -i\hbar\frac{\partial V}{\partial x}\psi
\end{align*}
所以 $[\hat{p}_x, V] = -i\hbar\frac{\partial V}{\partial x}$,代入即得证。
\end{exampleblock}

\begin{keypoint}
埃伦费斯特定理表明,量子力学期望值的演化规律与经典力学方程形式上一致,是对应原理的体现。
\end{keypoint}

\section{态的叠加原理}

\subsection{展开定理}

任何一个任意的、行为良好的函数(态)$\Psi$ 都可以展开为某个厄米算符 $\hat{A}$ 的本征函数系 $\{\psi_n\}$ 的线性组合:
\begin{equation}
\Psi = \sum_n c_n \psi_n
\end{equation}

展开系数的计算:
\begin{equation}
c_n = \int \psi_n^* \Psi \, d\tau
\end{equation}

\subsection{测量的概率}

如果体系处于叠加态 $\Psi = \sum_n c_n \psi_n$,测量物理量 $A$:
\begin{itemize}\setlength{\itemsep}{0pt}\setlength{\parskip}{0pt}
\item 测量结果必然是某个本征值 $a_k$。
\item 测得 $a_k$ 的概率为 $|c_k|^2$。
\item 测量后,体系的态\textbf{坍缩}到对应的本征态 $\psi_k$。
\end{itemize}

\begin{keypoint}
归一化要求:$\sum_n |c_n|^2 = 1$
\end{keypoint}

\begin{exampleblock}{例题:氢原子2p态的叠加}
考虑氢原子处于 $2p$ 态的线性叠加:
\begin{equation*}
\Psi = \frac{1}{\sqrt{2}}(\psi_{211} + \psi_{210})
\end{equation*}

测量 $L_z$ 时:
\begin{itemize}\setlength{\itemsep}{0pt}
\item 得到 $+\hbar$ 的概率:$|\frac{1}{\sqrt{2}}|^2 = \frac{1}{2}$
\item 得到 $0$ 的概率:$|\frac{1}{\sqrt{2}}|^2 = \frac{1}{2}$
\end{itemize}

$L_z$ 期望值:
\begin{equation*}
\langle L_z \rangle = \frac{1}{2}(+\hbar) + \frac{1}{2}(0) = \frac{\hbar}{2}
\end{equation*}

能量期望值($E_{2p}$ 简并):

\begin{equation*}
\langle E \rangle = E_{2p}
\end{equation*}
是确定的,因为两态能量相同。
\end{exampleblock}

\section{测量与波函数坍缩}

\subsection{测量过程}

\textbf{测量前}:体系处于叠加态
\begin{equation}
\Psi = \sum_n c_n \psi_n
\end{equation}

\textbf{测量}:得到某个本征值 $a_k$,概率为 $|c_k|^2$

\textbf{测量后}:波函数坍缩到
\begin{equation}
\Psi' = \psi_k
\end{equation}

\begin{keypoint}
\textbf{量子测量的特点}:
\begin{enumerate}\setlength{\itemsep}{0pt}
\item 测量改变体系状态(非定域性)
\item 重复测量同一物理量立即得到相同结果
\item 测量不对易的物理量会破坏之前的测量结果
\end{enumerate}
\end{keypoint}

\subsection{不确定性关系的推导}

\begin{derivation}
对于任意两个厄米算符 $\hat{A}$ 和 $\hat{B}$:

定义偏差算符:$\Delta\hat{A} = \hat{A} - \langle A \rangle$,$\Delta\hat{B} = \hat{B} - \langle B \rangle$

利用柯西-施瓦茨不等式:
\begin{equation*}
\langle(\Delta\hat{A})^2\rangle \langle(\Delta\hat{B})^2\rangle \ge \left|\langle\Delta\hat{A}\Delta\hat{B}\rangle\right|^2
\end{equation*}

将 $\langle\Delta\hat{A}\Delta\hat{B}\rangle$ 分解为对称和反对称部分:
\begin{equation*}
\langle\Delta\hat{A}\Delta\hat{B}\rangle = \frac{1}{2}\langle\{\Delta\hat{A},\Delta\hat{B}\}\rangle + \frac{1}{2}\langle[\Delta\hat{A},\Delta\hat{B}]\rangle
\end{equation*}

其中反对称部分:
\begin{equation*}
\langle[\Delta\hat{A},\Delta\hat{B}]\rangle = \langle[\hat{A},\hat{B}]\rangle
\end{equation*}

因此:
\begin{equation*}
\Delta A \cdot \Delta B \ge \frac{1}{2}|\langle[\hat{A},\hat{B}]\rangle|
\end{equation*}

对于位置和动量:$[\hat{x},\hat{p}_x] = i\hbar$,得:
\begin{equation*}
\Delta x \cdot \Delta p_x \ge \frac{\hbar}{2}
\end{equation*}
\end{derivation}

\section{常用公式速查}

\begin{scriptsize}
\begin{itemize}\setlength{\itemsep}{0pt}
\item 基本对易子:$[\hat{x},\hat{p}_x] = i\hbar$,$[\hat{x},\hat{p}_y] = 0$
\item 不确定关系:$\Delta x \cdot \Delta p_x \ge \frac{\hbar}{2}$
\item 角动量对易:$[\hat{L}_x,\hat{L}_y] = i\hbar\hat{L}_z$(循环)
\item 海森堡方程:$\frac{d\langle A \rangle}{dt} = \frac{1}{i\hbar}\langle[\hat{A},\hat{H}]\rangle + \langle\frac{\partial\hat{A}}{\partial t}\rangle$
\item 埃伦费斯特定理:$\frac{d\langle x \rangle}{dt} = \frac{\langle p \rangle}{m}$,$\frac{d\langle p \rangle}{dt} = -\langle\nabla V\rangle$
\item 态叠加:$\Psi = \sum_n c_n\psi_n$,$\sum_n|c_n|^2 = 1$
\item 期望值:$\langle A \rangle = \sum_n |c_n|^2 a_n$
\item 测量概率:$P(a_n) = |c_n|^2 = |\langle\psi_n|\Psi\rangle|^2$
\item 对易子恒等式:$[\hat{A},\hat{B}\hat{C}] = [\hat{A},\hat{B}]\hat{C} + \hat{B}[\hat{A},\hat{C}]$
\item 雅可比恒等式:$[\hat{A},[\hat{B},\hat{C}]] + [\hat{B},[\hat{C},\hat{A}]] + [\hat{C},[\hat{A},\hat{B}]] = 0$
\end{itemize}
\end{scriptsize}

\end{multicols}

\end{document}

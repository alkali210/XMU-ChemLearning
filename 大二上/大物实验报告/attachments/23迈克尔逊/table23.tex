\documentclass{article}
\usepackage{ctex}
\usepackage{geometry}
\usepackage{tabularx}
\usepackage{amsmath}

\geometry{a4paper, margin=1in}
\newcolumntype{C}{>{\centering\arraybackslash}X}

\begin{document}

\begin{table}[h!]
    \renewcommand{\arraystretch}{2}
    \centering
    \begin{tabularx}{\textwidth}{|c|C|C|C|C|C|C|}
    \hline
    测量次数($\Delta$m=50) & 1 & 2 & 3 & 4 & 5 & 6 \\
    \hline
    d$_1$位置(mm) & 59.70038 & 59.66900 & 59.65277 & 59.61223 & 59.62818 & 59.64398 \\
    \hline
    d$_2$位置(mm) & 59.68445 & 59.65277 & 59.63691 & 59.62818 & 59.64398 & 59.65973 \\
    \hline
    现象描述 & 陷入 & 陷入 & 陷入 & 冒出 & 冒出 & 冒出 \\
    \hline
    $\Delta$d (mm) & 0.01593 & 0.01623 & 0.01586 & 0.01595 & 0.01580 & 0.01575 \\
    \hline
    $\lambda_n$ (nm) & 637.2 & 649.2 & 634.4 & 638.0 & 632.0 & 630.0 \\
    \hline
    平均值$\lambda$ (nm) & \multicolumn{6}{c|}{636.8} \\
    \hline
    \end{tabularx}
\end{table}

\section*{计算过程与不确定度分析}

\subsection*{1. 各次测量波长 $\lambda_n$ 的计算}
根据公式 $\lambda = \frac{2\Delta d}{\Delta m}$,其中 $\Delta m = 50$。
\begin{align*}
    \lambda_1 &= \frac{2 \times 0.01593\,\text{mm}}{50} = 0.0006372\,\text{mm} = 637.2\,\text{nm} \\
    \lambda_2 &= \frac{2 \times 0.01623\,\text{mm}}{50} = 0.0006492\,\text{mm} = 649.2\,\text{nm} \\
    \lambda_3 &= \frac{2 \times 0.01586\,\text{mm}}{50} = 0.0006344\,\text{mm} = 634.4\,\text{nm} \\
    \lambda_4 &= \frac{2 \times 0.01595\,\text{mm}}{50} = 0.0006380\,\text{mm} = 638.0\,\text{nm} \\
    \lambda_5 &= \frac{2 \times 0.01580\,\text{mm}}{50} = 0.0006320\,\text{mm} = 632.0\,\text{nm} \\
    \lambda_6 &= \frac{2 \times 0.01575\,\text{mm}}{50} = 0.0006300\,\text{mm} = 630.0\,\text{nm}
\end{align*}
\textbf{有效位数说明:} $\Delta d$ 的测量值为4位有效数字,$\Delta m=50$ 为精确计数,可视为无限位有效数字。因此计算得到的波长 $\lambda_n$ 也应保留4位有效数字。

\subsection*{2. 平均波长 $\bar{\lambda}$ 的计算}
\begin{align*}
    \bar{\lambda} &= \frac{1}{6} \sum_{n=1}^{6} \lambda_n \\
    &= \frac{637.2 + 649.2 + 634.4 + 638.0 + 632.0 + 630.0}{6} \,\text{nm} \\
    &= \frac{3820.8}{6} \,\text{nm} \\
    &= 636.8\,\text{nm}
\end{align*}
\textbf{有效位数说明:} 进行加法运算时,结果的末位与所有加数中末位最粗的对齐。此处所有 $\lambda_n$ 值均精确到小数点后一位,因此和 $3820.8$ 也精确到小数点后一位。除以整数6时,根据有效数字修约规则以及物理实验数据处理习惯,平均值的最后一位应与测量值对齐,故保留到小数点后一位,为 $636.8$。

\subsection*{3. A类不确定度 $u_A(\lambda)$ 的计算}
A类不确定度由贝塞尔公式计算:
\begin{align*}
    u_A(\lambda) &= \sqrt{\frac{\sum_{n=1}^{6} (\lambda_n - \bar{\lambda})^2}{6(6-1)}} \\
    &= \sqrt{\frac{(0.4)^2 + (12.4)^2 + (-2.4)^2 + (1.2)^2 + (-4.8)^2 + (-6.8)^2}{30}} \,\text{nm} \\
    &= \sqrt{\frac{0.16 + 153.76 + 5.76 + 1.44 + 23.04 + 46.24}{30}} \,\text{nm} \\
    &= \sqrt{\frac{230.4}{30}} \,\text{nm} \\
    &= \sqrt{7.68} \,\text{nm} \approx 2.77\,\text{nm}
\end{align*}
\textbf{有效位数说明:} 不确定度的计算结果通常取一到两位有效数字。此处修约后取两位有效数字,故 $u_A(\lambda) \approx 2.8\,\text{nm}$。

\subsection*{4. 相对误差 $E_r$ 的计算}
取标准波长 $\lambda_0 = 632.8\,\text{nm}$,计算相对误差:
\begin{align*}
    E_r &= \left| \frac{\bar{\lambda} - \lambda_0}{\lambda_0} \right| \times 100\% \\
    &= \left| \frac{636.8 - 632.8}{632.8} \right| \times 100\% \\
    &= \left| \frac{4.0}{632.8} \right| \times 100\% \\
    &\approx 0.006321 \times 100\% \\
    &\approx 0.63\%
\end{align*}
\textbf{有效位数说明:} 减法运算 $636.8 - 632.8 = 4.0$,根据加减法规则,结果保留到小数点后一位,所以 $4.0$ 是两位有效数字。除法运算中,分母 $632.8$ 有四位有效数字,分子 $4.0$ 有两位有效数字,故结果应保留两位有效数字,为 $0.63\%$。

\end{document}
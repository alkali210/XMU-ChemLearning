% filepath: untitled:Untitled-1
\documentclass{ctexart}
\usepackage{amsmath}
\usepackage{geometry}
\geometry{a4paper, left=2cm, right=2cm, top=2cm, bottom=2cm}

\begin{document}

\noindent
\textbf{1. 测量二极管伏安特性}

\vspace{1em}
\noindent
\textbf{(1) 测量二极管正向伏安特性}

\noindent
附参考数据表
\begin{center}
\begin{tabular}{|l|p{1cm}|p{1cm}|p{1cm}|p{1cm}|p{1cm}|p{1cm}|p{1cm}|p{1cm}|p{1cm}|}
\hline
$I$ (mA) & 0.00 & 0.30 & 0.50 & 1.00 & 2.00 & 5.00 & 8.00 & 11.00 & 15.00 \\
\hline
$U$ (V)  & 0.0030 & 0.1910 & 0.2312 &0.2892&0.3918&0.6340&0.8322&1.0084 &1.2452 \\
\hline
\end{tabular}
\end{center}

\vspace{1em}
\noindent
\textbf{(2) 测量二极管反向伏安特性}

\noindent
附参考数据表
\begin{center}
\begin{tabular}{|l|p{1cm}|p{1cm}|p{1cm}|p{1cm}|p{1cm}|p{1cm}|p{1cm}|p{1cm}|p{1cm}|p{1cm}|}
\hline
$U$ (V)      & 0.00 & 1.00 & 2.00 & 3.00 & 5.00 & 7.00 & 9.00 & 11.00 & 13.00 & 15.00 \\
\hline
$I$ ($\mu$A) &1.21  &9.86  &11.52 &14.20 &16.45 &19.10 &21.68 &24.96  &27.50  &31.30  \\
\hline
\end{tabular}
\end{center}

\vspace{1em}
\noindent
\textbf{2. 描绘二极管的伏安特性曲线, 计算有关数值}

\noindent
以 $U$ 为横轴, $I$ 为纵轴, 作二极管的 $I-U$ 特性曲线, 二极管的正反向特性曲线画在同一坐标系上, 正向特性曲线在第一象限, 反向特性曲线在第三象限, 正反向坐标轴选取不同单位。

\vspace{0.5em}
\noindent
由曲线可求出当正向 $I=1.50\,\text{mA}$ 及反向 $U=-6.50\,\text{V}$ 时, 它们各自的动态电阻 $r=\dfrac{\mathrm{d}U}{\mathrm{d}I}$ 和静态电阻 $R=\dfrac{U}{I}$。

\vspace{1em}
\noindent
\textbf{3. 测量小灯泡的伏安特性}

\noindent
电压表量程:0$\sim$7.5V ;\hspace{2cm} 准确度等级: 0.5;\hspace{2cm} 内阻: $3.75\,\text{k}\Omega$

\noindent
电流表量程:0$\sim$750mA ;\hspace{1.65cm} 准确度等级: 0.5;\hspace{2cm} 内阻: $0.06\,\Omega$

\noindent
附参考数据表
\begin{center}
\begin{tabular}{|l|p{1cm}|p{1cm}|p{1cm}|p{1cm}|p{1cm}|p{1cm}|p{1cm}|p{1cm}|}
\hline
$U$ (V)  & 0.00 & 0.25 & 0.50 & 1.00 & 1.50 & 2.00 & 2.50 & 3.00 \\
\hline
$I$ (mA) &0.00 &120.0 &170.0 &200.0 &230.0 &250.0 &270.0 &285.0 \\
\hline
\end{tabular}
\end{center}

\vspace{1em}
\noindent
\textbf{4. 描绘小灯泡的伏安特性曲线, 计算有关数值}

\noindent
以 $U$ 为横轴, $I$ 为纵轴, 作小灯泡的 $I-U$ 特性曲线。

\noindent
由曲线求出在电压为 $2.70\,\text{V}$ 情况下, 小灯泡的直流电阻及其接入误差的相对值。

\end{document}
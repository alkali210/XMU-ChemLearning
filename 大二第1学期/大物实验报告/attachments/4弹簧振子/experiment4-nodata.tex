\documentclass[a4paper]{article}
\usepackage{ctex}
\usepackage{amsmath}
\usepackage{multirow}
\usepackage{tabularx}
\usepackage{diagbox}
\usepackage{geometry}
\usepackage{graphicx}
\geometry{a4paper, left=1cm, right=2cm, top=0.5cm, bottom=0.5cm}

\begin{document}

\section*{1. 气垫导轨的水平调节}

\begin{tabular*}{\textwidth}{@{\extracolsep{\fill}}|c|c|c|c|}
\hline
& $\Delta t_1$ (ms) & $\Delta t_2$ (ms) & $\displaystyle\frac{|\Delta t_1 - \Delta t_2|}{(\Delta t_1 + \Delta t_2)/2}$ (\%) \\
\hline
 左$\rightarrow$右 &  &  &  \\
\hline
 右$\rightarrow$左 &  &  &  \\
\hline
\end{tabular*}

\section*{2. 弹簧振子简谐振动周期与振幅的关系}

周期数\underline{\hspace{1em}\hspace{1em}}, (振子) $m_1$= \underline{\hspace{1em}\hspace{1em}}g, (两个弹簧) $m_s$= \underline{\hspace{1em}\hspace{1em}}g。

\vspace{1em}

\begin{tabular*}{\textwidth}{@{\extracolsep{\fill}}|c|c|c|c|c|}
\hline
\diagbox{时间 $t$ (s)}{次数}{振幅 A (cm)} & 18.00 & 20.00 & 22.00 & 24.00 \\
\hline
1 &  &  &  &  \\
\hline
2 &  &  &  &  \\
\hline
3 &  &  &  &  \\
\hline
4 &  &  &  &  \\
\hline
5 &  &  &  &  \\
\hline
6 &  &  &  &  \\
\hline
$\bar{t}$ (s) &  &  &  &  \\
\hline
周期 $T$ (s) &  &  &  &  \\
\hline
倔强系数 $k$ (N/m) &  &  &  &  \\
\hline
\end{tabular*}

\vspace{1em}
弹簧振子简谐振动周期与振幅的关系为:

\section*{3. 弹簧振子简谐振动周期与振子质量的关系}
(设定的振幅 A=\underline{\hspace{1em}\hspace{1em}}cm, 周期数\underline{\hspace{1em}\hspace{1em}})

\vspace{1em}

\renewcommand{\arraystretch}{1.2}
\newcolumntype{C}{>{\centering\arraybackslash}X}
\begin{tabularx}{\textwidth}{|>{\centering\arraybackslash}m{3cm}|C|C|C|C|C|}
\hline
& $m\quad$ (g) & \multicolumn{2}{c|}{时间 $t$ (s)} & 周期 $T$ (s) & $T^2$ (s$^2$) \\
\hline
\multirow{2}{=}{滑块} & \multirow{2}{=}{} &  &  & \multirow{2}{=}{} & \multirow{2}{=}{} \\
\cline{3-4}
 & &  &  & & \\
\hline
\multirow{2}{=}{滑块 +2个骑码} & \multirow{2}{=}{} &  &  & \multirow{2}{=}{} & \multirow{2}{=}{} \\
\cline{3-4}
 & &  &  & & \\
\hline
\multirow{2}{=}{滑块 +4个骑码} & \multirow{2}{=}{} &  &  & \multirow{2}{=}{} & \multirow{2}{=}{} \\
\cline{3-4}
 & &  &  & & \\
\hline
\multirow{2}{=}{滑块 +6个骑码} & \multirow{2}{=}{} &  &  & \multirow{2}{=}{} & \multirow{2}{=}{} \\
\cline{3-4}
 & &  &  & & \\
\hline
\end{tabularx}

\end{document}
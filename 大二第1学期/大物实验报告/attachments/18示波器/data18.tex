\documentclass{article}
\usepackage{ctex}
\usepackage{amsmath}
\usepackage{array}
\usepackage{geometry}
\usepackage{multirow}
\usepackage{makecell}
\usepackage{graphicx}
\usepackage{float} % 添加 float 包以使用 [H] 选项

\geometry{a4paper, margin=2cm}

\newcolumntype{M}[1]{>{\centering\arraybackslash}m{#1}}

\begin{document}

\section*{数据处理}

\vspace{0.3cm}

\noindent 1. 测量方波参数(表 3-18-1)(原始数据见原始数据表格)

\vspace{0.3cm}

\begin{center}
\renewcommand{\arraystretch}{1.5}
\begin{tabular}{|M{1.5cm}|M{1.5cm}|M{2.5cm}|M{2.5cm}|M{2.5cm}|M{2.5cm}|M{1.5cm}|}
\hline
信号频率 & 占空比 & $V_H$ (V) & $V_L$ (V) & $T_{U+}$ (ms) & $T_{U-}$ (ms) & $T$ (ms) \\
\hline
\multirow{3}{*}{2 kHz} & $30.0\%$ & 2.0 & -1.0 & 0.15 & 0.34 & 0.50 \\
\cline{2-7}
 & $50.0\%$ & 1.5 & -1.5 & 0.25 & 0.25 & 0.50 \\
\cline{2-7}
 & $70.0\%$ & 0.9 & -2.1 & 0.34 & 0.16 & 0.50 \\
\hline
\end{tabular}
\end{center}

\vspace{0.6cm}

\noindent 2. 测量正弦波的周期与频率(表 3-18-2)(原始数据见原始数据表格)

\vspace{0.3cm}

\begin{center}
\renewcommand{\arraystretch}{1.5}
\begin{tabular}{|M{4cm}|M{2.5cm}|M{2.5cm}|M{2.5cm}|M{2.5cm}|}
\hline
频率示值 (Hz) & 1 k & 20 k & 50 k & 100 k \\
\hline
测量周期 $T$ & 0.98 ms & 50.0 $\mathrm{\mu}$s & 20.0 $\mathrm{\mu}$s & 10.0 $\mathrm{\mu}$s \\
\hline
计算频率 $f$ & 1.02 k & 20.0 k & 50.0 k & 100.0 k \\
\hline
\end{tabular}
\end{center}

\vspace{0.6cm}

\noindent 3. 用图形表示三种情况下的李萨如图形,并用割线法计算未知频率。

\vspace{0.3cm}

\begin{center}
\renewcommand{\arraystretch}{1.5}
\begin{tabular}{|M{3cm}|M{2.5cm}|M{2.5cm}|M{2.5cm}|M{2.5cm}|M{2.5cm}|}
\hline
 李萨如图形 & $f_Y$(Hz) & $n_X$ & $n_Y$ & $f_X$测量值(Hz) & \multicolumn{1}{c|}{$\displaystyle f_X=f_Y\frac{n_Y}{n_X}$} \\
\hline
\rule{0pt}{0.25cm}
\includegraphics[width=3cm, height=4cm, keepaspectratio]{Figure_1.png} \rule{0pt}{0.25cm} & 1.0000 k & 2 & 2 & 999.98 & 1.0000 k \\
\hline
\rule{0pt}{0.25cm}
\includegraphics[width=3cm, height=4cm, keepaspectratio]{Figure_2.png} \rule{0pt}{0.25cm} & 1.0000 k & 2 & 4 & 1.9999 k & 2.0000 k \\
\hline
\rule{0pt}{0.25cm}
\includegraphics[width=3cm, height=4cm, keepaspectratio]{Figure_3.png} \rule{0pt}{0.25cm} & 2.0000 k & 4 & 6 & 2.9999 k & 3.0000 k \\
\hline
\end{tabular}
\end{center}

\end{document}
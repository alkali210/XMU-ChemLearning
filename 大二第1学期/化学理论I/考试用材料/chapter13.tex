% 编译提示:使用 XeLaTeX 编译
% 需要 ctex 宏包支持中文

\documentclass[a4paper,9pt]{article}
\usepackage{amsmath}
\usepackage{amssymb}
\usepackage{geometry}
\usepackage{ctex}
\usepackage{xcolor}
\usepackage{multicol}
\usepackage{titlesec}
\usepackage{fancyhdr}
\usepackage{tcolorbox}
\usepackage{graphicx}
\usepackage[version=4]{mhchem}

\geometry{left=1cm,right=1cm,top=1cm,bottom=1.8cm}

\setlength{\columnsep}{0.8cm}
\setlength{\columnseprule}{0.4pt}
\renewcommand{\columnseprulecolor}{\color{gray!50}}

% 定义颜色方案
\definecolor{sectioncolor}{RGB}{0,102,204}      % 深蓝色 - 用于章节标题
\definecolor{subsectioncolor}{RGB}{0,153,153}   % 青色 - 用于小节标题
\definecolor{keycolor}{RGB}{204,102,0}          % 橙色 - 用于关键点
\definecolor{derivcolor}{RGB}{100,149,237}      % 浅蓝色 - 用于推导框
\definecolor{boxbg}{RGB}{240,248,255}           % 淡蓝色背景
\definecolor{keybg}{RGB}{255,248,240}           % 淡橙色背景

% 自定义章节标题格式
\titleformat{\section}
  {\normalfont\large\bfseries\color{sectioncolor}}
  {\thesection}{0.5em}{}
\titlespacing*{\section}{0pt}{1ex plus 0.5ex minus 0.2ex}{0.8ex plus 0.2ex}

\titleformat{\subsection}
  {\normalfont\normalsize\bfseries\color{subsectioncolor}}
  {\thesubsection}{0.5em}{}
\titlespacing*{\subsection}{0pt}{0.8ex plus 0.3ex minus 0.1ex}{0.5ex plus 0.1ex}

\setcounter{tocdepth}{4}
\setcounter{secnumdepth}{4}

% 自定义关键点环境
\newenvironment{keypoint}
{\par\vspace{0.3em}\noindent\begin{tcolorbox}[
  colback=keybg,
  colframe=keycolor,
  boxrule=0.5pt,
  arc=2pt,
  left=2pt,
  right=2pt,
  top=2pt,
  bottom=2pt,
  boxsep=0pt,
  before skip=0.15em,
  after skip=0.3em]
\small\noindent\textcolor{keycolor}{\textbf{关键点:}}}
{\end{tcolorbox}\par\vspace{0.3em}}

% 自定义推导环境
\newenvironment{derivation}
{\par\vspace{0.15em}\noindent\begin{tcolorbox}[
  colback=boxbg,
  colframe=derivcolor,
  boxrule=0.5pt,
  arc=2pt,
  left=2pt,
  right=2pt,
  top=2pt,
  bottom=2pt,
  boxsep=0pt,
  before skip=0.15em,
  after skip=0.15em]
\scriptsize\noindent\textcolor{derivcolor}{\textbf{推导:}}\par}
{\end{tcolorbox}\par\vspace{0.15em}}

% 自定义示例环境
\newenvironment{examplebox}
{\par\vspace{0.15em}\noindent\begin{tcolorbox}[
  colback=boxbg,
  colframe=derivcolor,
  boxrule=0.5pt,
  arc=2pt,
  left=2pt,
  right=2pt,
  top=2pt,
  bottom=2pt,
  boxsep=0pt,
  before skip=0.15em,
  after skip=0.15em]
\scriptsize\noindent\textcolor{derivcolor}{\textbf{示例:}}\par
\setlength{\parskip}{0.1ex}\setlength{\itemsep}{0pt}\setlength{\parsep}{0pt}}
{\end{tcolorbox}\par\vspace{0.15em}}

% 页眉页脚设置
\pagestyle{fancy}
\fancyhf{}
\fancyfoot[C]{\scriptsize13-\thepage}
\renewcommand{\headrulewidth}{0pt}
\renewcommand{\footrulewidth}{0.4pt}
\renewcommand{\footrule}{\hbox to\headwidth{\color{gray!50}\leaders\hrule height \footrulewidth\hfill}}

% 紧凑标题
\title{\vspace{-2em}\Large\bfseries\color{sectioncolor} Chapter 13 Collections - 分子光谱\vspace{-1em}}
\author{}
\date{\today}

\begin{document}

\maketitle
\thispagestyle{fancy}


\begin{multicols}{2}
    
\scriptsize{\tableofcontents}


\section{振动跃迁与双原子分子 (Vibrational Spectroscopy)}

\subsection{简谐振子模型 (Harmonic Oscillator)}

\textbf{经典力学}:
\begin{itemize}
    \item 回复力:$F = -k(r - r_e) = -kx$
    \item 势能:$V(x) = \frac{1}{2}kx^2$
    \item 角频率:$\omega = \sqrt{k/\mu}$,频率 $\nu = \frac{1}{2\pi}\sqrt{k/\mu}$
\end{itemize}

\textbf{量子力学}:
\begin{equation}
E_v = \left(v + \frac{1}{2}\right)\hbar\omega = \left(v + \frac{1}{2}\right)h\nu, \quad v = 0,1,2,\cdots
\end{equation}
项值(cm$^{-1}$):$G(v) = \tilde{\nu}_e(v + 1/2)$

\textbf{零点能}:$E_0 = \frac{1}{2}\hbar\omega = \frac{1}{2}h\nu$(量子效应,不可消除)

\subsection{Morse振子模型 (Anharmonic Oscillator)}

\textbf{Morse势函数}:
\begin{equation}
V_M(x) = D_e\left[1 - e^{-\beta x}\right]^2, \quad x = r - r_e
\end{equation}
其中 $D_e$ 为势阱深度(解离能),$\beta$ 为势阱陡度。

\textbf{近谐近似}($x \ll 1$):$e^{-\beta x} \approx 1 - \beta x$
\begin{equation}
V_M(x) \approx D_e\beta^2 x^2 = \frac{1}{2}kx^2 \quad \Rightarrow \quad k = 2D_e\beta^2
\end{equation}

\textbf{能级公式}:
\begin{equation}
G(v) = \tilde{\omega}_e\left(v + \frac{1}{2}\right) - \tilde{\omega}_ex_e\left(v + \frac{1}{2}\right)^2
\end{equation}

\begin{keypoint}
\textbf{非谐性常数}:
\begin{equation}
x_e = \frac{\hbar\beta^2}{2\mu\omega} = \frac{\tilde{\omega}_e}{4D_e}
\end{equation}
\textbf{解离能计算}:
\begin{itemize}
    \item 势阱深度:$D_e = \frac{\tilde{\omega}_e^2}{4\tilde{\omega}_ex_e} = \frac{\tilde{\omega}_e}{4x_e}$
    \item 零点能修正:$D_0 = D_e - G(0) = D_e - \frac{\tilde{\omega}_e}{2} + \frac{\tilde{\omega}_ex_e}{4}$
    \item 最大振动量子数:$v_{max} = \frac{1}{2x_e} - \frac{1}{2} = \frac{1-x_e}{2x_e}$
\end{itemize}
\end{keypoint}

\textbf{振动跃迁}:
\begin{itemize}
    \item \textbf{基频吸收} ($v=0\to1$):$\Delta\tilde{\nu}_{1\leftarrow0} = \tilde{\omega}_e - 2\tilde{\omega}_ex_e$
    \item \textbf{第一泛频} ($v=0\to2$):$\Delta\tilde{\nu}_{2\leftarrow0} = 2\tilde{\omega}_e - 6\tilde{\omega}_ex_e$
    \item \textbf{热带} ($v=1\to2$,高温可见):$\Delta\tilde{\nu}_{2\leftarrow1} = \tilde{\omega}_e - 4\tilde{\omega}_ex_e$
    \item \textbf{相邻能级差}:$\Delta G_{v+1/2} = \tilde{\omega}_e - 2\tilde{\omega}_ex_e(v+1)$
\end{itemize}

\subsection{刚性转子与非刚性转子}

\textbf{刚性转子能级}(双原子分子):
\begin{equation}
F(J) = BJ(J+1), \quad J = 0,1,2,\cdots
\end{equation}
其中 $F(J)$ 为项值(cm$^{-1}$),$B = \frac{h}{8\pi^2 cI} = \frac{\hbar}{4\pi cI}$ 为转动常数,$I = \mu R_e^2$ 为转动惯量。

\textbf{非刚性转子修正}:
\begin{equation}
F(J) = BJ(J+1) - DJ^2(J+1)^2
\end{equation}
$D$ 为离心畸变常数,$D = \frac{4B^3}{\tilde{\nu}_e^2}$。

\textbf{谐振子能级}:
\begin{equation}
G(v) = \tilde{\nu}_e\left(v+\frac{1}{2}\right), \quad v = 0,1,2,\cdots
\end{equation}

\textbf{非谐振子能级}:
\begin{equation}
G(v) = \tilde{\nu}_e\left(v+\frac{1}{2}\right) - \tilde{\nu}_ex_e\left(v+\frac{1}{2}\right)^2
\end{equation}
$x_e$ 为非谐性常数($\ll 1$)。

\begin{keypoint}
\textbf{解离能计算}:
\begin{itemize}
    \item 势能曲线零点:$D_e = \frac{\tilde{\omega}_e^2}{4\tilde{\omega}_ex_e}$
    \item 零点能修正:$D_0 = D_e - \frac{\tilde{\omega}_e}{2} + \frac{\tilde{\omega}_ex_e}{4}$
\end{itemize}
\textbf{振动频率}:
\begin{equation}
\tilde{\omega}_e = \frac{1}{2\pi c}\sqrt{\frac{k}{\mu}}
\end{equation}
$k$ 为力常数,$\mu$ 为约化质量。
\end{keypoint}

\subsection{Morse势与光谱常数解题技巧}

\textbf{关键公式组合}:
\begin{itemize}
    \item $\Delta G_{v+1/2} = G(v+1) - G(v) = \omega_e - 2\omega_e x_e(v+1)$
    \item 解离能:$D_e = \frac{\omega_e^2}{4\omega_e x_e} = \frac{\omega_e}{4x_e}$
    \item 零点能修正:$D_0 = D_e - \frac{\omega_e}{2}$
    \item 非谐性导致谱线间距递减
\end{itemize}

\textbf{解题步骤}:
\begin{enumerate}
    \item 从两个跃迁能量求解 $\omega_e$ 和 $\omega_e x_e$
    \item 判断HOMO/LUMO:激发对应成键$\to$反键
    \item 键长变化:激发态通常键长增大
    \item FC因子最大处:$R'$ 和 $R''$ 波函数重叠最大
\end{enumerate}

\subsection{同位素效应}

\textbf{频率比}:
\begin{equation}
\frac{\omega_2}{\omega_1} = \sqrt{\frac{\mu_1}{\mu_2}}
\end{equation}

\textbf{约化质量}(双原子):
\begin{equation}
\mu = \frac{m_1 m_2}{m_1 + m_2}
\end{equation}

\textbf{常见计算}:
\begin{itemize}
    \item $^{12}C^{16}O$ vs $^{13}C^{16}O$:频率比 $\approx 1.02$
    \item $H_2$ vs $D_2$:频率比 $= \sqrt{2} \approx 1.41$
    \item 重同位素 $\to$ 频率降低,谱线红移
\end{itemize}




\section{转动光谱 (Rotational Spectroscopy)}

\subsection{刚性转子类型}

\textbf{惯性矩主轴}:$I_a \leq I_b \leq I_c$

\textbf{分类}:
\begin{itemize}
    \item \textbf{球形陀螺}:$I_a = I_b = I_c$($CH_4$,$SF_6$)
    \item \textbf{对称陀螺}:
    \begin{itemize}
        \item 长形(prolate):$I_a < I_b = I_c$($CH_3Cl$)
        \item 扁形(oblate):$I_a = I_b < I_c$($C_6H_6$)
    \end{itemize}
    \item \textbf{非对称陀螺}:$I_a < I_b < I_c$($H_2O$)
\end{itemize}

\textbf{转动常数}:
\begin{equation}
A = \frac{\hbar}{4\pi cI_a}, \quad B = \frac{\hbar}{4\pi cI_b}, \quad C = \frac{\hbar}{4\pi cI_c}
\end{equation}

\subsection{线性分子与对称陀螺}

\textbf{线性分子能级}:
\begin{equation}
F(J) = BJ(J+1), \quad J = 0,1,2,\cdots
\end{equation}

\textbf{纯转动跃迁}:$\Delta J = \pm 1$
\begin{equation}
\tilde{\nu}(J \to J+1) = 2B(J+1), \quad J = 0,1,2,\cdots
\end{equation}

\textbf{对称陀螺能级}:
\begin{equation}
F(J,K) = BJ(J+1) + (A-B)K^2
\end{equation}
$K$ 为角动量在对称轴上的投影量子数,$|K| \leq J$。

\textbf{选律}:
\begin{itemize}
    \item \textbf{纯转动(微波)}:$\Delta J = \pm 1$
    \item \textbf{$K$ 选律取决于偶极矩方向}:$\mu\parallel$(沿对称轴)$\Rightarrow\Delta K = 0$;$\mu\perp$(垂直对称轴)$\Rightarrow\Delta K = \pm 1$
\end{itemize}


\subsection{振-转耦合光谱}

\textbf{振-转能级}:
\begin{equation}
E_{v,J} = G(v) + F(J) = \tilde{\nu}_e\left(v+\frac{1}{2}\right) + BJ(J+1)
\end{equation}

非谐振子-非刚性转子:
\begin{equation}
\hspace{-5pt}E_{v,J} = \tilde{\nu}_e\left(v+\frac{1}{2}\right) - \tilde{\nu}_ex_e\left(v+\frac{1}{2}\right)^2 + B_vJ(J+1) - D_vJ^2(J+1)^2
\end{equation}
其中 $B_v = B_e - \alpha_e(v+1/2)$,考虑振动对转动常数的影响。

\textbf{选律}:
\begin{itemize}
    \item \textbf{振动}:$\Delta v = \pm 1$(基频),$\pm 2, \pm 3, \cdots$(泛频,弱)
    \item \textbf{转动}:$\Delta J = \pm 1$
\end{itemize}

\textbf{谱线结构}:
\begin{itemize}
    \item \textbf{P 支} ($\Delta J = -1$):$\tilde{\nu}_P(\Delta J = -1) = \tilde{\nu}_{v',v''} - \tilde{B}_{v'}(J+1) - (\tilde{B}_{v'} - \tilde{B}_{v''})J^2$
    \item \textbf{R 支} ($\Delta J = +1$):$\tilde{\nu}_R(\Delta J = +1) = \tilde{\nu}_{v',v''} + 2\tilde{B}_{v'} + (3\tilde{B}_{v'} - \tilde{B}_{v''})J + (\tilde{B}_{v'} - \tilde{B}_{v''})J^2$
\end{itemize}

简化形式($B_v \approx$ 常数):
\begin{align}
\tilde{\nu}_P(J) &= \tilde{\nu}_0 - 2BJ, \quad J = 1,2,3,\cdots \\
\tilde{\nu}_R(J) &= \tilde{\nu}_0 + 2B(J+1), \quad J = 0,1,2,\cdots
\end{align}

\textbf{谱线间距}:相邻谱线间距 $\approx 2B$。

\begin{examplebox}
\textbf{$^{12}C^{16}O$ 振-转光谱}:\\
$\tilde{\nu}_e = 2170$ cm$^{-1}$,$B_e = 1.931$ cm$^{-1}$\\
P支:$J = 1 \to 0$:$\tilde{\nu} = 2170 - 2(1.931)(1) = 2166$ cm$^{-1}$\\
R支:$J = 0 \to 1$:$\tilde{\nu} = 2170 + 2(1.931)(1) = 2174$ cm$^{-1}$\\
谱线间距:$\Delta\tilde{\nu} \approx 2B_e = 3.86$ cm$^{-1}$
\end{examplebox}

\subsection{离心畸变与非谐性}

\textbf{离心畸变}:高 $J$ 值时,分子被拉伸,转动常数减小。
\begin{equation}
F(J) = BJ(J+1) - DJ^2(J+1)^2
\end{equation}

\textbf{非谐性常数 $x_e$}:
\begin{itemize}
    \item 典型值:$x_e \approx 0.01 - 0.03$
    \item 导致泛频间距递减:$\Delta G_{v+1/2} = G(v+1) - G(v) = \tilde{\nu}_e - 2\tilde{\nu}_ex_e(v+1)$
    \item 解离能估算:$D_e = \frac{\tilde{\nu}_e}{4x_e}$
\end{itemize}

\begin{examplebox}
\textbf{$H^{35}Cl$ 光谱常数}:\\
$\tilde{\nu}_e = 2990.95$ cm$^{-1}$,$\tilde{\nu}_ex_e = 52.82$ cm$^{-1}$\\
$\Delta G_{1/2} = 2990.95 - 2(52.82) = 2885.3$ cm$^{-1}$\\
$\Delta G_{3/2} = 2990.95 - 4(52.82) = 2779.7$ cm$^{-1}$\\
间距递减:$\Delta G_{1/2} - \Delta G_{3/2} = 2\tilde{\nu}_ex_e = 105.6$ cm$^{-1}$
\end{examplebox}


\subsection{能级布居与Boltzmann分布}

\begin{keypoint}
\textbf{热平衡下的能级布居}(普适公式):
\begin{equation}
\frac{N_{\text{upper}}}{N_{\text{lower}}} = \frac{g_{\text{upper}}}{g_{\text{lower}}} \exp\left(-\frac{\Delta E}{k_B T}\right)
\end{equation}
其中 $g$ 为能级简并度,$\Delta E = E_{\text{upper}} - E_{\text{lower}}$ 为能级差。

\textbf{特殊情况}:
\begin{itemize}
    \item \textbf{振动能级}:$g_v = 1$(非简并),$\Delta E = h\nu$
    \begin{equation}
    \frac{N_v}{N_0} = \exp\left(-\frac{vh\nu}{k_B T}\right)
    \end{equation}
    \item \textbf{转动能级}:$g_J = 2J+1$(简并度),$\Delta E = hcBJ(J+1)$
    \begin{equation}
    \frac{N_J}{N_0} = (2J+1)\exp\left(-\frac{hcBJ(J+1)}{k_BT}\right)
    \end{equation}
    \item \textbf{电子能级}:$g_e$ 取决于电子态多重度 $2S+1$
\end{itemize}

\textbf{温度效应}:
\begin{itemize}
    \item 低温:仅最低能级有显著布居
    \item 高温:高能级布居增加,热带(hot bands)出现
    \item 最大布居:简并度与Boltzmann因子的竞争结果
\end{itemize}
\end{keypoint}

\subsection{转动能级布居与最大强度}

\textbf{应用Boltzmann分布}
\begin{equation}
\frac{N_J}{N_0} = (2J+1)\exp\left(-\frac{hc\tilde{B}J(J+1)}{k_BT}\right)
\end{equation}

\textbf{最大布居量子数}(对 $J$ 求导令 $dN_J/dJ = 0$):
\begin{equation}
    J_{max} = \sqrt{\frac{k_BT}{2hc\tilde{B}}} - \frac{1}{2} 
\end{equation}
\hspace{3em}\scriptsize{*如果以$cm^{-1}$为单位,取$c=\tilde{c}=3\times10^{10}$ cm/s}

\textbf{键长计算}(从转动常数):
\begin{equation}
R = \sqrt{\frac{\hbar}{4\pi c\tilde{B}\mu}} = \sqrt{\frac{h}{8\pi^2 c\tilde{B}\mu}}
\end{equation}

\begin{examplebox}
\textbf{$^{12}C^{16}O$ 转动光谱}:\\
$\tilde{B} = 1.931$ cm$^{-1}$,$\tilde{D} = 6.1 \times 10^{-6}$ cm$^{-1}$\\
$J = 0 \to 1$:$\tilde{\nu} = 2\tilde{B} = 3.862$ cm$^{-1}$\\
$J = 1 \to 2$:$\tilde{\nu} = 4\tilde{B} = 7.724$ cm$^{-1}$\\
间距恒定:$\Delta\tilde{\nu} = 2\tilde{B}$(忽略离心畸变)
\end{examplebox}

\subsection{核自旋统计权重}

\textbf{同核双原子分子}:核自旋影响能级占据。
\begin{itemize}
    \item \textbf{Bose 子}(整数自旋,如 $^{16}O_2$):对称波函数
    \begin{itemize}
        \item 偶数 $J$:核自旋对称 $\to$ 权重大
        \item 奇数 $J$:核自旋反对称 $\to$ 权重小
    \end{itemize}
    \item \textbf{Fermi 子}(半整数自旋,如 $^{1}H_2$):反对称波函数
    \begin{itemize}
        \item 偶数 $J$:para-H$_2$(核自旋反对称)$\to$ 权重 1
        \item 奇数 $J$:ortho-H$_2$(核自旋对称)$\to$ 权重 3
    \end{itemize}
\end{itemize}

\textbf{强度比}:
\begin{equation}
\frac{I(\text{ortho})}{I(\text{para})} = \frac{(2I+1)(I+1)}{I(2I+1)} = \frac{I+1}{I}
\end{equation}
$I$ 为核自旋量子数。对于 $^{1}H_2$($I = 1/2$),比值 = 3:1。


\section{原子与分子光谱项符号 (Term Symbols)}

\subsection{原子光谱项}

\textbf{原子光谱项符号}:
\begin{equation}
^{2S+1}L_J
\end{equation}
\begin{itemize}
    \item $2S+1$:自旋多重度,S为总自旋角动量,各电子 $s_i$ 的矢量和,$\vec{S} = \sum_i \vec{s}_i$\\
    取值范围($n$电子):$S = 0, 1, 2, \ldots$ 或 $1/2, 3/2, \ldots$
    \item $L$:总轨道角动量,各电子 $l_i$ 的矢量和,$\vec{L} = \sum_i \vec{l}_i$\\
    取值范围(两电子):$|l_1 - l_2| \leq L \leq l_1 + l_2$
    \item $J$:总角动量,$L$ 与 $S$ 的矢量和,$\vec{J} = \vec{L} + \vec{S}$\\
    取值范围:$|L - S| \leq J \leq L + S$,共 $2\min(L,S)+1$ 个 $J$ 值
\end{itemize}

\textbf{例}:两个 $p$ 电子($l_1 = l_2 = 1$)$\to$ $L = 0, 1, 2$(对应 S, P, D 项)

\begin{keypoint}
\textbf{$L$ 的字母对应}:
\begin{center}
\begin{tabular}{ccccccccc}
$L$ & 0 & 1 & 2 & 3 & 4 & 5 & 6 & ... \\
符号 & S & P & D & F & G & H & I & ... \\
\end{tabular}
\end{center}
\textbf{示例}:$^3P_2$($S=1$, $L=1$, $J=2$),$^1S_0$($S=0$, $L=0$, $J=0$)
\end{keypoint}

\subsection{双原子分子光谱项}

\textbf{分子光谱项符号}:
\begin{equation}
^{2S+1}\Lambda_{\Omega}^{(+/-)} \quad \text{或} \quad ^{2S+1}\Lambda_{g/u}^{(+/-)}
\end{equation}
\begin{itemize}
    \item $2S+1$:自旋多重度
    \item $\Lambda$:轨道角动量在分子轴上的投影,用希腊字母表示
    \item $\Omega$:总角动量投影
    \item $+/-$:仅$\Sigma$态,对镜面的对称性
    \item $g/u$:仅同核分子,对反演中心的对称性
\end{itemize}

\begin{examplebox}
$^3\Sigma_g^-$(O$_2$基态),$^2\Pi_{1/2}$(NO基态分量),$^1\Sigma^+$(CO基态)
\end{examplebox}

\textbf{常见分子基态}:
\begin{itemize}
    \item H$_2$:$^1\Sigma_g^+$;N$_2$:$^1\Sigma_g^+$;O$_2$:$^3\Sigma_g^-$(顺磁!)
    \item NO:$^2\Pi$;CO:$^1\Sigma^+$;OH:$^2\Pi$
\end{itemize}

\section{电子光谱 (Electronic Spectroscopy)}

\subsection{Frank-Condon 原理}

\textbf{核心思想}:电子跃迁远快于核运动($\sim 10^{-15}$ s vs $10^{-13}$ s)。

\begin{keypoint}
\textbf{Frank-Condon 因子}:
\begin{equation}
|S_{v'v''}|^2 = \left|\int \chi_{v'}^* \chi_{v''} d\tau\right|^2
\end{equation}
决定振动谱带强度。
\textbf{物理意义}:
\begin{itemize}
    \item 跃迁"垂直"发生($R$ 不变)
    \item 最强跃迁:$v'$ 和 $v''$ 波函数最大重叠处
    \item 平衡核间距变化大 $\to$ 跃迁到高振动态
\end{itemize}
\end{keypoint}

\textbf{振动进程 (Progression)}:
\begin{itemize}
    \item $R_e' \approx R_e''$ $\to$ $0 \to 0$ 最强
    \item $R_e' \gg R_e''$ $\to$ $0 \to v'$ ($v'$ 较大) 最强
\end{itemize}

\subsection{电子跃迁与HOMO/LUMO分析技巧}

\textbf{判断准则}:
\begin{itemize}
    \item HOMO $\to$ LUMO:第一激发态(S$_0 \to$ S$_1$)
    \item $\sigma$ 成键 $\to$ $\sigma^*$ 反键:键长显著增大
    \item $\pi$ 成键 $\to$ $\pi^*$ 反键:键长略增
    \item $n$ 非键 $\to$ $\pi^*$:键长变化小
    \item C=O键伸缩频率变化:成键电子移除$\to$频率降低
\end{itemize}

\textbf{FC因子应用技巧}:
\begin{itemize}
    \item $R_e' \approx R_e''$ $\to$ $0 \to 0$ 跃迁最强(键长变化小)
    \item $R_e' \gg R_e''$ $\to$ $0 \to v'$ (高$v'$) 最强(键长显著增大)
    \item FC包络宽窄反映几何结构变化程度
    \item 不是跃迁概率,是振动态重叠积分平方
\end{itemize}

\subsection{电子-振动-转动耦合}

\textbf{总能量}:
\begin{equation}
E = E_{elec} + G(v) + F(J)
\end{equation}

\textbf{选律}:
\begin{itemize}
    \item \textbf{电子}:$\Delta\Lambda = 0, \pm 1$($\Lambda$ 为电子轨道角动量投影)
    \item \textbf{振动}:$\Delta v = 0, \pm 1, \pm 2, \cdots$(无限制,但 FC 因子决定强度)
    \item \textbf{转动}:$\Delta J = 0, \pm 1$(禁止 $J' = 0 \leftrightarrow J'' = 0$)
\end{itemize}

\textbf{谱带结构}:P、Q、R 支
\begin{itemize}
    \item P 支:$\Delta J = -1$,$\tilde{\nu}_P = \tilde{\nu}_0 - (\tilde{B}' + \tilde{B}'')J + (\tilde{B}' - \tilde{B}'')J^2$
    \item Q 支:$\Delta J = 0$(仅当 $\Delta\Lambda \neq 0$),$\tilde{\nu}_Q = \tilde{\nu}_0 + (\tilde{B}' - \tilde{B}'')J(J+1)$
    \item R 支:$\Delta J = +1$,$\tilde{\nu}_R = \tilde{\nu}_0 + 2\tilde{B}' + (3\tilde{B}' - \tilde{B}'')J + (\tilde{B}' - \tilde{B}'')J^2$
\end{itemize}

\textbf{带原点 (Band Origin)}:
\begin{equation}
\tilde{\nu}_0 = \tilde{\omega}_e - 2\tilde{\omega}_ex_e \quad (\text{对 } v=0\to1)
\end{equation}

\textbf{Q支展宽}:若 $\tilde{B}' \neq \tilde{B}''$,Q支不重合而展宽。
\textbf{双原子 IR}:通常只有 P、R 支($\Delta\Lambda = 0$,无 Q 支)。

\subsection{预解离与辐射寿命}

\textbf{预解离 (Predissociation)}:
\begin{itemize}
    \item 电子激发态 $\to$ 与排斥态交叉 $\to$ 解离
    \item 光谱特征:谱线突然变宽或消失
\end{itemize}

\textbf{辐射寿命 $\tau$}:
\begin{equation}
\tau = \frac{1}{A_{21}} \approx 10^{-8} \text{ s}
\end{equation}
$A_{21}$ 为 Einstein 自发辐射系数。

\textbf{谱线宽度}:
\begin{equation}
\Delta\tilde{\nu} \approx \frac{1}{2\pi c\tau}
\end{equation}
寿命短 $\to$ 谱线宽。

\section{拉曼光谱 (Raman Spectroscopy)}

\subsection{拉曼散射原理}

\textbf{散射类型}:
\begin{itemize}
    \item \textbf{Rayleigh 散射}:$\nu_{scattered} = \nu_0$(弹性)
    \item \textbf{Stokes 拉曼}:$\nu_{scattered} = \nu_0 - \nu_{vib}$(分子吸收能量)
    \item \textbf{Anti-Stokes 拉曼}:$\nu_{scattered} = \nu_0 + \nu_{vib}$(分子失去能量)
\end{itemize}

\textbf{强度比}:
\begin{equation}
\frac{I_{AS}}{I_S} \propto e^{-h\nu_{vib}/kT}
\end{equation}
Anti-Stokes 通常很弱(需要分子已在激发态)。低温时 Stokes 线更强,高温时两者趋近。

\subsection{转动拉曼光谱}

\textbf{选律}:$\Delta J = 0, \pm 2$(线性分子)

\textbf{能级差}:$\varepsilon_J = \tilde{B}J(J+1)$,$\Delta\varepsilon = \tilde{B}(4J+6)$

\textbf{Stokes线}($\Delta J = +2$):
\begin{equation}
\tilde{\nu}_S = \tilde{\nu}_{laser} - \tilde{B}(4J+6), \quad J = 0,1,2,\cdots
\end{equation}

\textbf{Anti-Stokes线}($\Delta J = -2$):
\begin{equation}
\tilde{\nu}_{AS} = \tilde{\nu}_{laser} + \tilde{B}(4J-2), \quad J = 2,3,4,\cdots
\end{equation}

\textbf{谱线间距}:$\Delta\tilde{\nu} = 4\tilde{B}$(等间距,可精确测定 $\tilde{B}$)

\begin{keypoint}
\textbf{拉曼活性判据}:
\begin{equation}
\left(\frac{\partial\alpha_{ij}}{\partial Q}\right)_0 \neq 0
\end{equation}
极化率张量分量随简正坐标变化。
\textbf{物理本质}:分子极化率 $\alpha$ 随振动发生变化。
\begin{itemize}
    \item 诱导偶极矩:$\mu_{ind} = \alpha E$
    \item $\alpha$ 与键强度/键长相关:键伸缩使 $\alpha$ 变化 $\to$ Raman活性
    \item 引入平均极化率:$\bar{\alpha} = (\alpha_{xx} + \alpha_{yy} + \alpha_{zz})/3$
\end{itemize}
\textbf{优势}:
\begin{itemize}
    \item 可见光激发(不需 UV-IR 光源)
    \item 水溶液测量(水的拉曼信号弱)
    \item 低频振动($< 400$ cm$^{-1}$)
    \item 与 IR 互补(互斥规则)
\end{itemize}
\end{keypoint}

\subsection{共振拉曼与表面增强拉曼}

\textbf{共振拉曼 (RR)}:
\begin{itemize}
    \item 激发光频率接近电子吸收带
    \item 强度增强 $10^3 - 10^6$ 倍
    \item 选择性激发特定发色团振动
\end{itemize}

\textbf{表面增强拉曼 (SERS)}:
\begin{itemize}
    \item 分子吸附在金属纳米结构表面
    \item 局域表面等离激元共振
    \item 增强因子 $10^{6} - 10^{14}$
    \item 单分子检测可能
\end{itemize}

\section{光电子能谱 (Photoelectron Spectroscopy, PES)}

\subsection{基本原理}

\textbf{光电效应}:
\begin{equation}
h\nu = IE + KE_{electron}
\end{equation}

\subsection{PES成键性判断技巧}
	\textbf{核心逻辑}:
\begin{itemize}
    \item 谱带精细结构间距 = 离子态振动频率 $\omega_e^+$
    \item $\omega_e^+ < \omega_e$:移除成键电子(键变弱,频率降低)
    \item $\omega_e^+ > \omega_e$:移除反键电子(键变强,频率升高)
    \item $\omega_e^+ \approx \omega_e$:移除非键电子(键长几乎不变)
    \item FC包络宽窄:反映中性分子与离子几何差异
\end{itemize}

\textbf{PES与电子光谱结合分析}:
\begin{itemize}
    \item \textbf{轨道能级对应}:PES给出IE,电子光谱给出HOMO-LUMO间隙
    \item \textbf{精细结构}:第一激发态和离子态的振动间距可判断轨道性质
    \item \textbf{FC因子一致性}:两种光谱的强度分布都受FC原理支配
    \item \textbf{典型例题(ClO 等双原子)}:用 PES 第一电离带的振动间距得到 $\omega_e^+$,再与电子光谱振动进程间距(激发态 $\omega_e'$)对比,判断涉及轨道的成键/反键/非键性质
\end{itemize}

$IE$ = 电离能(分子轨道能量),$KE$ = 光电子动能。

\textbf{Koopmans 定理}:
\begin{equation}
IE_i \approx -\epsilon_i
\end{equation}
电离能近似等于分子轨道能量的负值(忽略轨道弛豫)。

\begin{keypoint}
\textbf{振动精细结构}:
\begin{itemize}
    \item PES 谱带显示离子振动能级(不是中性分子)
    \item 间距 = 离子基态振动频率 $\tilde{\nu}_e^+$
    \item \textbf{成键判据}:
    \begin{itemize}
        \item $\tilde{\nu}_e^+ < \tilde{\nu}_e$ $\to$ 移除成键电子(键变弱)
        \item $\tilde{\nu}_e^+ > \tilde{\nu}_e$ $\to$ 移除反键电子(键变强)
        \item $\tilde{\nu}_e^+ \approx \tilde{\nu}_e$ $\to$ 移除非键电子
    \end{itemize}
    \item \textbf{FC 包络}:反映中性分子与离子几何差异
\end{itemize}
\end{keypoint}

\begin{examplebox}
\textbf{$N_2$ 的 PES}:\\
最高占据 MO:$3\sigma_g$ (成键)\\
$\tilde{\nu}_e(N_2) = 2359$ cm$^{-1}$\\
$\tilde{\nu}_e^+(N_2^+) = 2207$ cm$^{-1}$ $< \tilde{\nu}_e$\\
结论:$3\sigma_g$ 为强成键轨道,移除后键长增大、振动频率降低。
\end{examplebox}

\subsection{光谱常数计算实例}

\begin{examplebox}
\textbf{$CN$ 分子电子光谱与 PES 分析}:\\
\textbf{振动能级公式}(非谐振子):
\begin{equation}
G(v) = \omega_e(v+1/2) - \omega_e x_e(v+1/2)^2
\end{equation}
\textbf{光谱常数求解}:\\
已知基频 ($v=0 \to 1$) 和第一泛频 ($v=0 \to 2$) 或热带 ($v=1 \to 2$):
\begin{itemize}
    \item $\Delta G_{1/2} = G(1)-G(0) = \omega_e - 2\omega_e x_e$
    \item $\Delta G_{3/2} = G(2)-G(1) = \omega_e - 4\omega_e x_e$
    \item 联立求解 $\omega_e$ (谐振频率) 和 $\omega_e x_e$ (非谐性常数)
\end{itemize}
\textbf{解离能}:$D_e = \frac{\omega_e^2}{4\omega_e x_e}$,$D_0 = D_e - \frac{\omega_e}{2}$ (零点能修正)\\
\textbf{PES 振动精细结构}:\\
电离过程 $CN \xrightarrow{h\nu} CN^+ + e^-$
\begin{itemize}
    \item 谱带精细结构间距 = 离子态振动频率 $\omega_e^+$(不是中性分子)
    \item \textbf{成键性判据}:
    \begin{itemize}
        \item $\omega_e^+ < \omega_e$ $\to$ 移除成键电子
        \item $\omega_e^+ > \omega_e$ $\to$ 移除反键电子
        \item $\omega_e^+ \approx \omega_e$ $\to$ 移除非键电子
    \end{itemize}
\end{itemize}
\end{examplebox}

\subsection{He(I) 与 He(II) 光源}

\textbf{常用光源}:
\begin{itemize}
    \item \textbf{He(I)}:21.22 eV (584 Å),研究价电子
    \item \textbf{He(II)}:40.81 eV (304 Å),探测内层电子
    \item \textbf{X-射线}:XPS (ESCA),测定核心电子结合能
\end{itemize}

\textbf{化学位移}:
\begin{itemize}
    \item 核心电子 IE 受化学环境影响
    \item 氧化态增加 $\to$ IE 增大
    \item 元素鉴定与化学态分析
\end{itemize}


\end{multicols}

\clearpage

\vspace{1em}
\noindent\textbf{\Large 附录:关键常数与数据}

\begin{multicols}{2}
\scriptsize

\textbf{电磁波谱区域}(按能量递减):
\begin{itemize}
    \item \textbf{射频 (RF)}:$10^3 - 10^{11}$ Hz,核磁共振 (NMR)
    \item \textbf{微波}:$10^{11} - 10^{13}$ Hz,分子转动
    \item \textbf{红外 (IR)}:$10^{13} - 10^{14}$ Hz,分子振动
    \item \textbf{可见-紫外 (UV-Vis)}:$10^{14} - 10^{16}$ Hz,电子跃迁
\end{itemize}

\textbf{常用换算关系}:
\begin{center}
\scalebox{0.8}{
    \scriptsize
    \begin{tabular}{lcccc}
    \hline
    & J$\cdot$mol$^{-1}$ & cal$\cdot$mol$^{-1}$ & eV & cm$^{-1}$ \\
    \hline
    1 J$\cdot$mol$^{-1}$ & 1 & 0.2390 & $1.036\times10^{-5}$ & $8.359\times10^{-2}$ \\
    1 cal$\cdot$mol$^{-1}$ & 4.184 & 1 & $4.336\times10^{-5}$ & 0.3497 \\
    1 eV & $9.649\times10^4$ & $2.306\times10^4$ & 1 & $8.065\times10^3$ \\
    1 cm$^{-1}$ & $1.196\times10^{-2}$ & 2.859 & $1.240\times10^{-4}$ & 1 \\
    \hline
    \end{tabular}
}
\end{center}

\begin{itemize}
    \item $kT$ (300 K) = 208.5 cm$^{-1}$ = 2.48 kJ/mol
    \item 1 Hartree (Eh) = 27.211 eV = 4.360 $\times$ 10$^{-18}$ J
    \item 1 Eh = 219474.6 cm$^{-1}$ = 627.51 kcal/mol
    \item 1 Bohr (a$_0$) = 0.5292 Å = 52.92 pm
    \item 1 a.u. 偶极矩 = 2.542 Debye
    \item 1 cm$^{-1}$ = 4.556 $\times$ 10$^{-6}$ Eh
    \item 1 amu = 1.6605 $\times$ 10$^{-27}$ kg
\end{itemize}

\textbf{典型光谱范围}:
\begin{itemize}
    \item NMR:$10^{-3} - 1$ cm$^{-1}$
    \item 微波/转动:$0.1 - 100$ cm$^{-1}$
    \item 远红外:$10 - 400$ cm$^{-1}$
    \item 中红外:$400 - 4000$ cm$^{-1}$
    \item 近红外:$4000 - 12500$ cm$^{-1}$
    \item 可见光:$12500 - 25000$ cm$^{-1}$
    \item 紫外:$25000 - 50000$ cm$^{-1}$
\end{itemize}

\textbf{特征振动频率 (cm$^{-1}$)}:
\begin{itemize}
    \item C–H 伸缩:2850–3300
    \item O–H 伸缩:3200–3650
    \item N–H 伸缩:3300–3500
    \item C=O 伸缩:1650–1850
    \item C=C 伸缩:1620–1680
    \item \ce{C#C} 伸缩:2100–2260
    \item \ce{C#N} 伸缩:2210–2280
\end{itemize}

\textbf{双原子分子数据示例}:
\begin{center}
\begin{tabular}{lccc}
\hline
分子 & $\tilde{\omega}_e$/cm$^{-1}$ & $\tilde{B}_e$/cm$^{-1}$ & $R_e$/pm \\
\hline
H$_2$ & 4401 & 60.85 & 74.1 \\
HCl & 2991 & 10.59 & 127.5 \\
CO & 2170 & 1.93 & 112.8 \\
N$_2$ & 2359 & 2.01 & 109.8 \\
O$_2$ & 1580 & 1.45 & 120.7 \\
\hline
\end{tabular}
\end{center}

\textbf{选律总结}:
\begin{center}
\begin{tabular}{lcc}
\hline
跃迁类型 & $\Delta v$ & $\Delta J$ \\
\hline
纯转动 & 0 & $\pm 1$ \\
振-转(红外) & $\pm 1$ & $\pm 1$ \\
电子光谱 & 任意 & $0, \pm 1$ \\
拉曼散射 & $\pm 1$ & $0, \pm 2$ \\
\hline
\end{tabular}
\end{center}

\end{multicols}

\footnotetext[1]{这份材料的章节顺序主要参考《物理化学-一种分子途径》,使用了AI辅助创建}
\footnotetext[2]{[Repository]https://github.com/alkali210/XMU-ChemLearning}

\end{document}

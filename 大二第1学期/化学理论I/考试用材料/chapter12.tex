% 编译提示:使用 XeLaTeX 编译
% 需要 ctex 宏包支持中文

\documentclass[a4paper,9pt]{article}
\usepackage{amsmath}
\usepackage{amssymb}
\usepackage{geometry}
\usepackage{ctex}
\usepackage{xcolor}
\usepackage{multicol}
\usepackage{titlesec}
\usepackage{fancyhdr}
\usepackage{tcolorbox}
\tcbuselibrary{breakable}
\usepackage{graphicx}
\usepackage[version=4]{mhchem}

\geometry{left=1cm,right=1cm,top=1cm,bottom=1.8cm}

\setlength{\columnsep}{0.8cm}
\setlength{\columnseprule}{0.4pt}
\renewcommand{\columnseprulecolor}{\color{gray!50}}

% 定义颜色方案
\definecolor{sectioncolor}{RGB}{0,102,204}      % 深蓝色 - 用于章节标题
\definecolor{subsectioncolor}{RGB}{0,153,153}   % 青色 - 用于小节标题
\definecolor{keycolor}{RGB}{204,102,0}          % 橙色 - 用于关键点
\definecolor{derivcolor}{RGB}{100,149,237}      % 浅蓝色 - 用于推导框
\definecolor{boxbg}{RGB}{240,248,255}           % 淡蓝色背景
\definecolor{keybg}{RGB}{255,248,240}           % 淡橙色背景

% 自定义章节标题格式
\titleformat{\section}
  {\normalfont\large\bfseries\color{sectioncolor}}
  {\thesection}{0.5em}{}
\titlespacing*{\section}{0pt}{1ex plus 0.5ex minus 0.2ex}{0.8ex plus 0.2ex}

\titleformat{\subsection}
  {\normalfont\normalsize\bfseries\color{subsectioncolor}}
  {\thesubsection}{0.5em}{}
\titlespacing*{\subsection}{0pt}{0.8ex plus 0.3ex minus 0.1ex}{0.5ex plus 0.1ex}

\setcounter{secnumdepth}{4}
\setcounter{tocdepth}{4}
% 自定义关键点环境
\newenvironment{keypoint}
{\par\vspace{0.3em}\noindent\begin{tcolorbox}[
  colback=keybg,
  colframe=keycolor,
  boxrule=0.5pt,
  arc=2pt,
  left=2pt,
  right=2pt,
  top=2pt,
  bottom=2pt,
  boxsep=0pt,
  breakable]
\small\noindent\textcolor{keycolor}{\textbf{关键点:}}}
{\end{tcolorbox}\par\vspace{0.3em}}

% 自定义推导环境
\newenvironment{derivation}
{\par\vspace{0.3em}\noindent\begin{tcolorbox}[
  colback=boxbg,
  colframe=derivcolor,
  boxrule=0.5pt,
  arc=2pt,
  left=2pt,
  right=2pt,
  top=2pt,
  bottom=2pt,
  boxsep=0pt,
  breakable]
\scriptsize\noindent\textcolor{derivcolor}{\textbf{推导:}}\par}
{\end{tcolorbox}\par\vspace{0.3em}}

% 自定义示例环境
\newenvironment{examplebox}
{\par\vspace{0.3em}\noindent\begin{tcolorbox}[
  colback=boxbg,
  colframe=derivcolor,
  boxrule=0.5pt,
  arc=2pt,
  left=2pt,
  right=2pt,
  top=2pt,
  bottom=2pt,
  boxsep=0pt,
  breakable]
\scriptsize\noindent\textcolor{derivcolor}{\textbf{示例:}}\par
\setlength{\parskip}{0.1ex}\setlength{\itemsep}{0pt}\setlength{\parsep}{0pt}}
{\end{tcolorbox}\par\vspace{0.3em}}

% 页眉页脚设置
\pagestyle{fancy}
\fancyhf{}
\fancyfoot[C]{\scriptsize12-\thepage}
\renewcommand{\headrulewidth}{0pt}
\renewcommand{\footrulewidth}{0.4pt}
\renewcommand{\footrule}{\hbox to\headwidth{\color{gray!50}\leaders\hrule height \footrulewidth\hfill}}

% 紧凑标题
\title{\vspace{-2em}\Large\bfseries\color{sectioncolor} Chapter 12 Collections - 对称性与群论\vspace{-1em}}
\author{}
\date{\today}

\begin{document}

\maketitle
\thispagestyle{fancy}

\begin{multicols}{2}

\scriptsize{\tableofcontents}

\section{分子对称性与点群判定 (Symmetry \& Point Groups)}

\subsection{对称操作的矩阵表示 (Matrix Representations)}
对称操作可以用矩阵形式描述,其作用于坐标向量 $(x, y, z)$ 或基函数组。
\begin{examplebox}
以 $H_2O$ ($C_{2v}$) 为例,坐标变换矩阵:
\begin{itemize}
    \item $\hat{E} \begin{pmatrix} x \\ y \\ z \end{pmatrix} = \begin{pmatrix} 1 & 0 & 0 \\ 0 & 1 & 0 \\ 0 & 0 & 1 \end{pmatrix} \begin{pmatrix} x \\ y \\ z \end{pmatrix}$ (迹 $\chi=3$)
    \item $\hat{C}_2(z) \begin{pmatrix} x \\ y \\ z \end{pmatrix} = \begin{pmatrix} -1 & 0 & 0 \\ 0 & -1 & 0 \\ 0 & 0 & 1 \end{pmatrix} \begin{pmatrix} x \\ y \\ z \end{pmatrix}$ (迹 $\chi=-1$)
    \item $\hat{\sigma}_v(xz) \begin{pmatrix} x \\ y \\ z \end{pmatrix} = \begin{pmatrix} 1 & 0 & 0 \\ 0 & -1 & 0 \\ 0 & 0 & 1 \end{pmatrix} \begin{pmatrix} x \\ y \\ z \end{pmatrix}$ (迹 $\chi=1$)
\end{itemize}
\textbf{群乘法表 (Group Multiplication Table)}:
群内操作的乘积(连续施加)必须仍为群内操作。例如在 $C_{2v}$ 中,$\hat{C}_2 \times \hat{\sigma}_v(xz) = \hat{\sigma}_v'(yz)$。
\end{examplebox}

\section{特征标计算与表示约化 (Representations)}

\subsection{特征标表符号详解 (Mulliken Symbols)}
不可约表示的符号(如 $A_1, B_{2g}, E'$ 等)蕴含了对称性信息:
\begin{itemize}
    \item \textbf{维度 (Dimension)}:
    \begin{itemize}
        \item $A, B$: 一维表示 ($\chi(E) = 1$)。
        \item $E$: 二维表示 ($\chi(E) = 2$)。
        \item $T$ (或 $F$): 三维表示 ($\chi(E) = 3$)。
    \end{itemize}
    \item \textbf{主轴 $C_n$ 对称性 ($A$ vs $B$)}:
    \begin{itemize}
        \item $A$: 对主轴 $C_n$ 操作是对称的 ($\chi(C_n) = +1$)。
        \item $B$: 对主轴 $C_n$ 操作是反对称的 ($\chi(C_n) = -1$)。
    \end{itemize}
    \item \textbf{中心反演 $i$ ($g$ vs $u$)}:
    \begin{itemize}
        \item $g$ (gerade, 偶): 对 $i$ 对称 ($\chi(i) = +1$)。
        \item $u$ (ungerade, 奇): 对 $i$ 反对称 ($\chi(i) = -1$)。
    \end{itemize}
    \item \textbf{镜面与次轴 ($1$ vs $2$ / $'$ vs $''$)}:
    \begin{itemize}
        \item 下标 $1/2$: 对垂直于主轴的 $C_2$ 轴对称($1$)/反对称($2$);若无 $C_2$ 轴,则指 $\sigma_v$。
        \item 上标 $'/''$: 对 $\sigma_h$ 对称($'$)/反对称($''$)。
    \end{itemize}
\end{itemize}

\subsection{特征标表的正交性定理 (Great Orthogonality Theorem)}
特征标表不仅仅是数字的集合,它们满足严格的数学性质:
\begin{itemize}
    \item \textbf{维数规则}:不可约表示维数的平方和等于群的阶数 $h$。
    \begin{equation}
    \sum_i [d_i]^2 = \sum_i [\chi_i(E)]^2 = h
    \end{equation}
    \item \textbf{行正交性}:不同不可约表示的特征标向量相互正交。
    \begin{equation}
    \sum_R \chi_i(R) \chi_j(R)^* = h \delta_{ij}
    \end{equation}
    \item \textbf{列正交性}:不同类操作的特征标列向量相互正交。
\end{itemize}

\begin{keypoint}
\textbf{计数法 (Counting Method) 求 $\Gamma$:}
对于每个对称操作 $R$,计算特征值 $\chi(R)$:
\[ \chi(R) = (\text{位置未变原子数}) \times (\text{每个原子的贡献}) \]
\textbf{快速记忆}:只需记住 $E, C_2, \sigma, i$ 四种,其他可推导。
\textbf{原子轨道贡献表:}
\begin{itemize}
    \item $s$ 轨道:总是 $+1$。
    \item $p$ 轨道:
    \begin{itemize}
        \item $E$ (恒等): $+3$
        \item $C_2$: 平行轴 $+1$,垂直轴 $-1$ $\to$ 总和 $-1$
        \item $\sigma$: 平行面 $+1$,垂直面 $-1$ $\to$ 总和 $+1$
        \item $i$: $-3$
    \end{itemize}
    \item 简单向量 ($x,y,z$):同 $p$ 轨道。
\end{itemize}
\end{keypoint}

\subsection{约化公式 (Reduction Formula)}
将可约表示 $\Gamma$ 分解为不可约表示 $\Gamma_i$ 的直和:
\begin{equation}
n_i = \frac{1}{h} \sum_{R} N_R \cdot \chi^{\Gamma}(R) \cdot \chi^{\Gamma_i}(R)^*
\end{equation}
其中:$h$ 为群的阶数,$N_R$ 为该类操作的数目,$\chi^{\Gamma}$ 为可约表示特征标,$\chi^{\Gamma_i}$ 为特征标表中不可约表示的特征标。

\begin{examplebox}
\textbf{$C_{2v}$ 约化示例} ($h=4$):$\Gamma_{red} = \{3, 1, 1, 3\}$。\\
计算各系数:\\
$n_{A_1} = \frac{1}{4}[1\cdot3\cdot1 + 1\cdot1\cdot1 + 1\cdot1\cdot1 + 1\cdot3\cdot1] = 2$\\
类似可得:$n_{A_2}=0, n_{B_1}=0, n_{B_2}=1$\\
结果:$\Gamma_{red} = 2A_1 + B_2$\\
\textbf{检验}:$\chi(E) = 2\cdot1 + 1\cdot1 = 3$ \checkmark,其他操作类似验证。
\end{examplebox}

\subsection{直积 (Direct Product)}
\textbf{快速判断}:查特征标表底部的直积表(若有),或逐个操作相乘验证。\\
\textbf{常用规则}:
\begin{itemize}
    \item $A \otimes A = A$;$A \otimes B = B$;$B \otimes B = A$
    \item $E \otimes E = A_1 + A_2 + B_1 + B_2$ (具体看点群)
    \item 任何表示与其自身直积必包含全对称表示
    \item $g \otimes g = g$;$g \otimes u = u$;$u \otimes u = g$
\end{itemize}
计算 $\Gamma_{A} \otimes \Gamma_{B}$ 的特征标:
\begin{equation}
\chi_{A \otimes B}(R) = \chi_A(R) \times \chi_B(R)
\end{equation}
\begin{keypoint}
\textbf{三重直积的物理意义}:判断矩阵元 $\langle \psi_i | \hat{O} | \psi_j \rangle$ 或积分 $\int \psi_i \hat{O} \psi_j \, d\tau$ 是否为零。
\begin{itemize}
    \item \textbf{核心原理}:只有当被积函数包含全对称成分时,积分才可能非零
    \item \textbf{判断条件}:$\Gamma_i \otimes \Gamma_{op} \otimes \Gamma_j$ 必须包含全对称表示($A_1, A_{1g}, \Sigma_g^+$ 等)
    \item \textbf{等价表述}:$\Gamma_i \otimes \Gamma_j$ 必须包含 $\Gamma_{op}$
\end{itemize}
\textbf{应用场景}:
\begin{itemize}
    \item \textbf{光谱跃迁}:$\hat{O} = \hat{\mu}$(偶极矩),$\Gamma_{op} = \Gamma_x, \Gamma_y, \Gamma_z$
    \item \textbf{振动跃迁}:基态 $\Gamma_i = A_1$,判断 $A_1 \otimes \Gamma_\mu \otimes \Gamma_{vib}$ 是否含 $A_1$
    \item \textbf{Raman 散射}:$\hat{O} = \hat{\alpha}$(极化率),$\Gamma_{op}$ 为二次函数表示
\end{itemize}
\textbf{速判规则}:$\Gamma \otimes \Gamma$ 总包含全对称表示;同表示直积含 $A_1$。
\end{keypoint}

\begin{examplebox}
\textbf{$C_{2v}$ 中计算 $A_2 \otimes B_1$}:\\
\begin{itemize}
    \item $E$: $1 \times 1 = 1$
    \item $C_2$: $1 \times (-1) = -1$
    \item $\sigma_v(xz)$: $-1 \times 1 = -1$
    \item $\sigma_v'(yz)$: $-1 \times (-1) = 1$
\end{itemize}
结果为 $\{1, -1, -1, 1\}$,对应 $B_2$。\\
\textbf{应用}:判断跃迁 $\psi(A_2) \xrightarrow{\hat{\mu}_z(A_1)} \psi'(B_1)$ 是否允许:\\
$A_2 \otimes A_1 \otimes B_1 = A_2 \otimes B_1 = B_2 \neq A_1$ $\to$ \textbf{禁戒}。
\end{examplebox}

\section{分子轨道理论 (MO Theory)}

\subsection{构造步骤}
\textbf{核心策略}:利用对称性匹配原理,只有相同不可约表示的轨道才能组合成键。
\begin{enumerate}
    \item \textbf{确定点群}。
    \item \textbf{中心原子轨道分类}:根据特征标表最右侧列 ($x, y, z, x^2, xy \dots$) 确定中心原子价轨道 ($s, p, d$) 的对称性。
    \item \textbf{配体群轨道 (LGOs/SALCs) 构造}:
    \begin{itemize}
        \item 确定配体轨道基组(如 4个 H 的 $1s$)。
        \item 求出 $\Gamma_{LGO}$ 并约化。
        \item \textbf{投影算符法 (Projection Operator)}:
        \begin{equation}
        \hat{P}_j = \frac{d_j}{h} \sum_R \chi_j(R)^* \hat{R}
        \end{equation}
        应用公式:$\psi_j \propto \hat{P}_j (\phi_{\text{basis}})$。
    \end{itemize}
    \begin{examplebox}
    \textbf{$H_2O$ ($C_{2v}$)} 求 $A_1$ 组合:\\
    对 $H_A$ ($1s$) 应用投影算符:\\
    $\hat{P}^{A_1} s_A \propto 1\cdot E(s_A) + 1\cdot C_2(s_A) + 1\cdot \sigma(s_A) + 1\cdot \sigma'(s_A)$\\
    $= s_A + s_B + s_B + s_A = 2(s_A + s_B)$。\\
    归一化:$\psi(A_1) = \frac{1}{\sqrt{2}}(s_A + s_B)$。\\
    \textbf{技巧}:若结果为零,说明该基函数不包含此不可约表示成分。
    \end{examplebox}
    \begin{itemize}
        \item 简便法:观察节点平面和相位变化,匹配特征标正负号。
    \end{itemize}
    \item \textbf{组装 MO 图}:
    \begin{itemize}
        \item 同对称性轨道组合。
        \item 能量原则:成键 < 非键 < 反键。
        \item 轨道重叠:能量相近且对称性匹配。
    \end{itemize}
\end{enumerate}

\subsection{久期行列式的分块对角化 (Block Diagonalization)}
\begin{keypoint}
群论在MO计算中的核心威力在于将高阶久期行列式分解为低阶子块。
\begin{itemize}
    \item \textbf{原理}:只有属于\textbf{相同不可约表示}的轨道之间才有非零的哈密顿矩阵元 ($H_{ij}$) 和重叠积分 ($S_{ij}$)。
    \item \textbf{效果}:一个 $N \times N$ 的大行列式分解为多个小行列式。
    \item \textbf{实例}:$CH_4$ ($T_d$) 的 8 个价轨道 ($C: s, p_x, p_y, p_z$ + $4H: s$) 原本构成 $8 \times 8$ 行列式。
    利用对称性分解为:
    \begin{itemize}
        \item $A_1$ 块 ($2 \times 2$): $C(2s)$ 与 $H(A_1)$ 组合。
        \item $T_2$ 块 ($2 \times 2$, 三重简并): $C(2p)$ 与 $H(T_2)$ 组合。
    \end{itemize}
    计算量大幅降低。
\end{itemize}
\end{keypoint}

\subsection{杂化轨道构建 (Hybrid Orbitals)}
\textbf{解题流程}:构造 $\Gamma_{hyb}$ $\to$ 约化 $\to$ 查表匹配轨道 $\to$ 确定杂化类型。
利用群论确定中心原子的杂化方式:
\begin{enumerate}
    \item \textbf{确定基向量}:以指向配体的向量为基,求出可约表示 $\Gamma_{hyb}$。
    \item \textbf{特征标计算}:$\chi(R)$ 等于操作 $R$ 下未移动的向量数目。
    \item \textbf{约化}:将 $\Gamma_{hyb}$ 分解为不可约表示。
    \item \textbf{匹配轨道}:在特征标表中找到对应不可约表示的原子轨道 ($s, p, d$)。
\end{enumerate}
\begin{examplebox}
    \textbf{$XY_3$ 平面三角形 ($D_{3h}$)}: $\Gamma_{hyb} = A_1' + E'$。\\
    查表:$A_1' \to s$, $E' \to (p_x, p_y)$。\\
    结论:$sp^2$ 杂化。\\
    \textbf{$XY_4$ 正四面体 ($T_d$)}: $\Gamma_{hyb} = A_1 + T_2$。\\
    查表:$A_1 \to s$, $T_2 \to (p_x, p_y, p_z)$。\\
    结论:$sp^3$ 杂化。\\
    \textbf{$XY_4$ 平面正方形 ($D_{4h}$)}: $\Gamma_{hyb} = A_{1g} + B_{1g} + E_u$。\\
    查表:$A_{1g} \to s$, $B_{1g} \to d_{x^2-y^2}$, $E_u \to (p_x, p_y)$。\\
    结论:$dsp^2$ 杂化。
\end{examplebox}

\begin{examplebox}
\textbf{甲烷 ($CH_4$) 的 $sp^3$ 杂化完整计算 ($T_d$)}
\begin{itemize}
    \item \textbf{基向量}:4个指向顶点的 C-H 键向量。
    \item \textbf{特征标计算 ($\Gamma_\sigma$)}:
    \begin{itemize}
        \item $E$ (不动): 4个向量均不变 $\to \chi=4$
        \item $8C_3$ (体对角线): 1个向量在轴上 $\to \chi=1$
        \item $3C_2$ (棱中点连线): 全部交换 $\to \chi=0$
        \item $6S_4$: 全部交换 $\to \chi=0$
        \item $6\sigma_d$: 2个向量在镜面上 $\to \chi=2$
    \end{itemize}
    结果:$\Gamma_\sigma = \{4, 1, 0, 0, 2\}$
    \item \textbf{分解}:查 $T_d$ 表可知 $\Gamma_\sigma = A_1 + T_2$。
    \item \textbf{轨道匹配}:$A_1 \to s$;$T_2 \to (p_x, p_y, p_z)$ 或 $(d_{xy}, d_{yz}, d_{xz})$
    \item \textbf{结论}:需 1个 $s$ 和 3个 $p$ 轨道组合,形成 $sp^3$ 杂化。
\end{itemize}
\end{examplebox}

\textbf{常见杂化类型与对称性}:
\begin{itemize}
    \item $sp$ ($D_{\infty h}$ 线性):$\Sigma_g^+ + \Sigma_u^+$,180°
    \item $sp^2$ ($D_{3h}$ 平面):$A_1' + E'$,120°
    \item $sp^3$ ($T_d$ 四面体):$A_1 + T_2$,109.5°
    \item $dsp^2$ ($D_{4h}$ 平面):$A_{1g} + B_{1g} + E_u$,90°
    \item $sp^3d$ ($D_{3h}$ 三角双锥):$2A_1' + A_2'' + E'$
    \item $d^2sp^3$ ($O_h$ 八面体):$A_{1g} + E_g + T_{1u}$,90°
\end{itemize}

\subsection{重要分子实例 (MO Diagrams)}
\textbf{MO 图绘制通用原则}:
\begin{enumerate}
    \item \textbf{能量匹配}:轨道能量相近才能有效组合 ($\Delta E < 10$ eV)。
    \item \textbf{对称性匹配}:只有相同不可约表示的轨道才有非零 $H_{ij}$。
    \item \textbf{重叠原则}:轨道空间重叠越大,相互作用越强。
    \item \textbf{能级排序规律}:成键 < 非键 < 反键;$\sigma$ 通常低于 $\pi$。
\end{enumerate}
\begin{itemize}
    \item \textbf{$H_2O$ (水, $C_{2v}$)}:
    \begin{itemize}
        \item \textbf{中心原子 O}:$2s(A_1), 2p_z(A_1), 2p_y(B_2), 2p_x(B_1)$
        \item \textbf{配体群轨道} $2H$: $\Gamma = A_1 + B_2$。
        \item \textbf{匹配成键}: 
            \begin{itemize}
                \item $A_1$: $O(2s) + O(2p_z)$ 与 $H(A_1)$ 形成 2个 $A_1$ MO (1$a_1$ 成键, 2$a_1$ 成键)
                \item $B_2$: $O(2p_y)$ 与 $H(B_2)$ 形成 1$b_2$ (成键)
                \item $B_1$: $O(2p_x)$ 为非键 (1$b_1$, 孤对电子)
            \end{itemize}
        \item \textbf{电子配置} (8个价e$^-$): $(1a_1)^2 (2a_1)^2 (1b_2)^2 (1b_1)^2$。
        \item HOMO = 1$b_1$ (非键,n轨道);LUMO = 3$a_1$ (反键)。
    \end{itemize}
    % 在 multicols 中不能使用浮动体 figure,直接插入图片
    \begin{center}
        \includegraphics[width=\linewidth]{attachments/image.png}
    \end{center}
    \item \textbf{$CH_4$ (甲烷, $T_d$)}:
    \begin{itemize}
        \item $C$: $2s(A_1), 2p(T_2)$。
        \item $4H$: $\Gamma = A_1 + T_2$。
        \item 匹配: $s-s$ 形成 $a_1$ 键,$p-s$ 形成 $t_2$ 键。无非键轨道。
    \end{itemize}
    \item \textbf{$AH_3$ (氨/三氟化硼)}:
    \begin{itemize}
        \item $BF_3$ ($D_{3h}$): $B(2p_z)$ 为 $A_2''$,与 $3F$ 的 $A_2''$ 组合形成 $\pi$ 键。
        \item $NH_3$ ($C_{3v}$): $N(2p_z)$ 为 $A_1$,与 $3H(A_1)$ 强相互作用。
    \end{itemize}
    \item \textbf{$SF_6$ (六氟化硫, $O_h$)}:
    \begin{itemize}
        \item $S$: $3s(A_{1g}), 3p(T_{1u}), 3d(E_g + T_{2g})$。
        \item $6F(\sigma)$: $A_{1g} + E_g + T_{1u}$。
        \item 结果: $s, p, d$ 均参与成键。$T_{2g}$ 为非键 (若仅考虑 $\sigma$ 作用)。
    \end{itemize}
    \item \textbf{乙硼烷 ($B_2H_6$, $D_{2h}$)}:
    \begin{itemize}
        \item 桥键 ($3c-2e$): 两个 $B(sp^3)$ 与桥 $H(s)$ 组合,形成 $A_g$ (成键) 和 $B_{3u}$ (成键)。
    \end{itemize}
\end{itemize}

\subsection{键级与重叠布居计算}

\textbf{重叠布居 (Overlap Population)}:
\begin{equation}
P_{AB} = 2\sum_i n_i c_{Ai} c_{Bi}
\end{equation}
$n_i$为轨道占据数,$c_{Ai}$, $c_{Bi}$为原子$A$, $B$在MO $i$中的系数。

\textbf{键级估算}:
\begin{itemize}
    \item 键级 $\approx \frac{1}{2}$(成键电子数 $-$ 反键电子数)
    \item 更精确:$\text{Bond Order} = \sum_{\text{bonding}} n_i - \sum_{\text{antibonding}} n_i$
    \item 键长 $\propto 1/$(键级)
\end{itemize}

\textbf{异核双原子修正Hückel方法}(CO、NO等):
\begin{itemize}
    \item O更负:$\alpha_O = \alpha_C + h\beta$($h \approx 1 \sim 2$)
    \item 久期方程:$\begin{vmatrix} \alpha_C - E & \beta \\ \beta & \alpha_O - E \end{vmatrix} = 0$
    \item 变换:$x = \frac{\alpha_C - E}{\beta}$,则 $x^2 - hx - 1 = 0$
    \item 解:$x = \frac{h \pm \sqrt{h^2 + 4}}{2}$
    \item 成键轨道偏向O(系数大),反键偏向C
\end{itemize}

\subsection{同核双原子分子处理技巧: $Sc_2$ ($D_{\infty h} \to D_{4h}$)}
\textbf{为何需要}:$D_{\infty h}$ 特征标表无限维,实际计算用有限子群近似。
\begin{examplebox}
\textbf{对称性下降法 (Descent of Symmetry)}:\\
处理 $D_{\infty h}$ 线性分子时,可利用其子群 $D_{4h}$ 的特征标表简化轨道分类。
\begin{itemize}
    \item \textbf{轨道对应}:$z$ 轴为主轴。
    $s, d_{z^2} \to a_{1g} + a_{2u}$ (在 $D_{4h}$ 中)。
    $d_{xz}, d_{yz} \to e_g + e_u$。
    \item \textbf{电子组态与基态}:
    $Sc$ ($4s^2 3d^1$)。MO 能级顺序通常为 $\sigma_g(s) < \sigma_u(s) < \pi_u(d) \approx \sigma_g(d)$。
    若基态为 $^5\Sigma_u^-$,说明有 4 个未成对电子分布在不同轨道,且总自旋 $S=2$。
    利用 $D_{4h}$ 组态 $(a_{1g})^1 (a_{2u})^1 (e_u)^2$ 可推导出总对称性为 $\Sigma_u^-$。
\end{itemize}
\end{examplebox}


\section{多原子分子振动的群论分析}

\subsection{简正模式与自由度}

\textbf{自由度分配}:
\begin{itemize}
    \item \textbf{总自由度}:$3N$($N$ 为原子数)
    \item \textbf{平动自由度}:3
    \item \textbf{转动自由度}:线性分子 2,非线性分子 3
    \item \textbf{振动自由度}:$3N-5$(线性),$3N-6$(非线性)
\end{itemize}

\begin{keypoint}
\textbf{简正模的定义与性质}:
\begin{itemize}
    \item \textbf{简正坐标 $Q_i$}:所有原子以\textbf{相同频率}和\textbf{相同相位}运动的集体振动模式。
    \item \textbf{数量}:非线性分子有 $3N-6$ 个,线性分子有 $3N-5$ 个简正模。
    \item \textbf{正交性}:不同简正模相互独立,无耦合,可独立激发。
    \item \textbf{对称性}:每个简正模属于某个不可约表示,由点群对称性决定。
\end{itemize}
\textbf{简正坐标变换}:
\begin{equation}
Q_k = \sum_{i=1}^{3N} L_{ki} x_i
\end{equation}
其中 $L_{ki}$ 为变换矩阵元素,$x_i$ 为笛卡尔坐标。
\end{keypoint}

\subsection{简正模式群论分析步骤}

\textbf{简正模式分析步骤}:
\begin{enumerate}
    \item \textbf{计算总自由度表示 $\Gamma_{3N}$}:
    \begin{equation}
    \chi_{3N}(R) = (\text{位置未变原子数 } N_{unshifted}) \times \chi_{xyz}(R)
    \end{equation}
    其中 $\chi_{xyz}(R) = 1 + 2\cos\theta$($\theta$ 为旋转角),通常为:$E:3$, $C_2:-1$, $C_3:0$($1+2\cos120^\circ=1-1=0$), $C_4:1$, $C_6:2$, $\sigma:1$, $i:-3$, $S_4:-1$, $S_6:0$。
    
    \item \textbf{减去平动与转动}:
    \[ \Gamma_{vib} = \Gamma_{3N} - \Gamma_{trans} (x,y,z) - \Gamma_{rot} (R_x, R_y, R_z) \]
    从特征标表中直接读取 $x,y,z$ 和 $R_x, R_y, R_z$ 对应的不可约表示。
\end{enumerate}

\textbf{常见错误}:忘记减去平动和转动;特征标计算时数错未移动的原子。

\begin{examplebox}
\textbf{$H_2O$ ($C_{2v}$) 简正模}:\\
$\Gamma_{vib} = 2A_1 + B_2$\\
\begin{itemize}
    \item $\nu_1(A_1)$:对称伸缩,3657 cm$^{-1}$
    \item $\nu_2(A_1)$:弯曲,1595 cm$^{-1}$
    \item $\nu_3(B_2)$:反对称伸缩,3756 cm$^{-1}$
\end{itemize}
全部 IR 和 Raman 活性(无反演中心)。
\end{examplebox}

\begin{examplebox}
\textbf{$CO_2$ ($D_{\infty h}$) 简正模}:\\
$\Gamma_{vib} = \Sigma_g^+ + \Sigma_u^+ + \Pi_u$(2个简并)\\
\begin{itemize}
    \item $\nu_1(\Sigma_g^+)$:对称伸缩,1333 cm$^{-1}$,仅 Raman 活性
    \item $\nu_2(\Pi_u)$:弯曲(简并),667 cm$^{-1}$,仅 IR 活性
    \item $\nu_3(\Sigma_u^+)$:反对称伸缩,2349 cm$^{-1}$,仅 IR 活性
    \item 有反演中心$i$,验证互斥规则
\end{itemize}
\end{examplebox}

\subsection{内坐标法 (Internal Coordinates Method)}

\textbf{为何使用内坐标}:笛卡尔坐标($3N$维)包含平动和转动,内坐标直接描述分子内部几何变化(键长、键角、二面角),更适合振动分析。

\begin{keypoint}
\textbf{内坐标类型}:
\begin{itemize}
    \item \textbf{键伸缩 (Bond Stretch)} $\Delta r_{ij}$:两原子间距变化。
    \item \textbf{键角弯曲 (Angle Bend)} $\Delta\theta_{ijk}$:三原子夹角变化。
    \item \textbf{二面角扭转 (Torsion)} $\Delta\phi_{ijkl}$:四原子定义的二面角变化。
    \item \textbf{面外弯曲 (Out-of-plane)} $\Delta\gamma$:原子偏离平面的角度。
\end{itemize}
\textbf{约化步骤(内坐标法)}:
\begin{enumerate}
    \item \textbf{选择内坐标基组}:如 $AX_3$ 分子选3个 A-X 键伸缩 + 3个 X-A-X 键角。
    \item \textbf{构造对称坐标 (Symmetry Coordinates)}:
    \begin{equation}
    S_i = \sum_j c_{ij} R_j
    \end{equation}
    其中 $R_j$ 为内坐标,$c_{ij}$ 由对称性确定。
    \item \textbf{计算可约表示 $\Gamma_{int}$}:
    \begin{equation}
    \chi_{int}(R) = \text{(未移动内坐标数)} \times (\text{贡献})
    \end{equation}
    \begin{itemize}
        \item \textbf{键伸缩}:未变键 $\to$ $+1$;反向 $\to$ $-1$。
        \item \textbf{键角}:未变角 $\to$ $+1$;通常 $C_2, \sigma$ 复杂些。
    \end{itemize}
    \item \textbf{约化 $\Gamma_{int}$}:用约化公式分解为不可约表示。
    \item \textbf{构造对称坐标}:利用投影算符或对称性观察法。
\end{enumerate}
\end{keypoint}

\begin{examplebox}
\textbf{$H_2O$ 内坐标详解 ($C_{2v}$)}:\\
\textbf{内坐标选择}:2个O-H键伸缩 ($r_1, r_2$) + 1个H-O-H键角 ($\theta$)\\
\textbf{伸缩振动} ($\Gamma_{stretch}$):
\begin{itemize}
    \item $E$: 2键不动 $\to$ $\chi=2$
    \item $C_2$: 2键互换 $\to$ $\chi=0$
    \item $\sigma_v(xz)$: 2键不动 $\to$ $\chi=2$
    \item $\sigma_v'(yz)$: 2键互换 $\to$ $\chi=0$
\end{itemize}
$\Gamma_{stretch} = \{2,0,2,0\}$ $\to$ 约化:$A_1 + B_2$\\
\textbf{对称坐标}:
\begin{itemize}
    \item $S_1(A_1)$: $(r_1 + r_2)/\sqrt{2}$ (对称伸缩,$\nu_1 \approx 3657$ cm$^{-1}$)
    \item $S_2(B_2)$: $(r_1 - r_2)/\sqrt{2}$ (反对称伸缩,$\nu_3 \approx 3756$ cm$^{-1}$)
\end{itemize}
弯曲振动:$\theta$ 属于 $A_1$ ($\nu_2 \approx 1595$ cm$^{-1}$)
\end{examplebox}

\begin{examplebox}
\textbf{$NH_3$ 内坐标 ($C_{3v}$)}:\\
\textbf{N-H伸缩} ($\Gamma_{stretch}$):
\begin{itemize}
    \item $E$: 3键不动 $\to$ $\chi=3$
    \item $2C_3$: 3键循环 $\to$ $\chi=0$
    \item $3\sigma_v$: 1键在镜面上 $\to$ $\chi=1$
\end{itemize}
$\Gamma_{stretch} = \{3,0,1\}$ $\to$ 约化:$A_1 + E$\\
\textbf{对称坐标}:
\begin{itemize}
    \item $S_1(A_1)$: $(r_1 + r_2 + r_3)/\sqrt{3}$ (全对称伸缩,$\nu_1$)\\
    $S_2, S_3(E)$: 简并伸缩 ($\nu_3$)
\end{itemize}
键角弯曲:同样得到 $A_1 + E$\\
$\Gamma_{vib} = 2A_1 + 2E$ (共4个基频)
\end{examplebox}

\vspace{0.3em}
\noindent\textbf{例}:\\[0.2em]
\includegraphics[width=\linewidth]{attachments/image_6.png}\\[0.2em]
\includegraphics[width=\linewidth]{attachments/image_5.png}

\subsection{红外与拉曼选律}

\begin{keypoint}
\textbf{选律判断}:
\begin{itemize}
    \item \textbf{IR 活性}:$\left(\frac{\partial\mu}{\partial Q}\right)_0 \neq 0$,偶极矩变化。查表:与 $x, y, z$ 同对称性。
    \item \textbf{Raman 活性}:$\left(\frac{\partial\alpha}{\partial Q}\right)_0 \neq 0$,极化率变化。查表:与 $x^2, xy, yz$ 等同对称性。
    \item \textbf{互斥规则}:有反演中心 $i$ $\to$ $g$ 仅 Raman,$u$ 仅 IR。
    \item \textbf{全对称模式}:总是 Raman 活性($\alpha_{xx} + \alpha_{yy} + \alpha_{zz}$ 总变化)。
\end{itemize}
\end{keypoint}

\textbf{去极化比 $\rho$}:
\begin{itemize}
    \item 极化(全对称):$0 < \rho < 0.75$
    \item 去极化(非全对称):$\rho = 0.75$
\end{itemize}

\subsection{IR/Raman光谱归属技巧}

\textbf{归属步骤}:
\begin{enumerate}
    \item \textbf{确定点群}:识别对称元素(重要:是否有反演中心)
    \item \textbf{群论分析}:计算 $\Gamma_{vib}$ 并约化为不可约表示
    \item \textbf{判断IR活性}:检查是否与 $x, y, z$ 同一表示
    \item \textbf{判断Raman活性}:检查是否与 $x^2, y^2, z^2, xy, xz, yz$ 同一表示
    \item \textbf{根据频率区间归属}:高频(伸缩)、低频(弯曲、摇摆)
\end{enumerate}

\textbf{关键技巧}:
\begin{itemize}
    \item \textbf{NO$_2^+$/NO$_3^-$类型}:线性分子无Q支,平面分子有多个简并振动
    \item \textbf{对称伸缩 vs 反对称伸缩}:前者通常Raman强,后者IR强
    \item \textbf{面外振动}:通常低频,对称性独特(如$A_2''$, $\Pi_u$)
\end{itemize}

\textbf{常见分子类型}:
\begin{itemize}
    \item \textbf{NO$_2^+$}($D_{\infty h}$):$\Sigma_g^+$(Raman),$\Sigma_u^+$(IR),$\Pi_u$(IR,简并)
    \item \textbf{NO$_3^-$}($D_{3h}$):$A_1'$(Raman),$A_2''$(IR,面外),$E'$(IR+Raman)
    \item \textbf{N$_2$O$_5$}($D_{2h}$):互斥规则适用,$g$/$u$分离清晰
    \item \textbf{N$_2$O$_3$}($C_{2v}$):无反演中心,IR和Raman重叠
\end{itemize}

\subsection{典型分子光谱实例}

\begin{itemize}
    \item \textbf{$N_2F_2$ - 异构体鉴别}:
    \begin{itemize}
        \item \textbf{反式 ($C_{2h}$)}: 具有反演中心。遵循互斥规则。
        \item \textbf{顺式 ($C_{2v}$)}: 无反演中心。大部分模式同时 IR 和 Raman 活性。
        \item \textbf{应用}: 通过比较 IR 和 Raman 谱峰重合情况可区分异构体。
    \end{itemize}

    \item \textbf{$XeF_4$ ($D_{4h}$) - 平面正方形}:
    \begin{itemize}
        \item $\Gamma_{vib} = A_{1g} + B_{1g} + B_{2g} + A_{2u} + B_{2u} + 2E_u$
        \item Raman 活性 ($g$): $A_{1g}, B_{1g}, B_{2g}$
        \item IR 活性 ($u$): $A_{2u}, E_u$
        \item \textbf{静止模式}: $B_{2u}$ (既非 IR 也非 Raman)
    \end{itemize}

    \item \textbf{$SF_6$ ($O_h$) - 正八面体}:
    \begin{itemize}
        \item $\Gamma_{vib} = A_{1g} + E_g + 2T_{1u} + T_{2g} + T_{2u}$
        \item Raman ($g$): $A_{1g}, E_g, T_{2g}$
        \item IR ($u$): $T_{1u}$
        \item \textbf{静止模式}: $T_{2u}$
    \end{itemize}
\end{itemize}

\subsection{点群判定与对称轨道构建技巧}

\textbf{解题流程}(适用于Li$_6$簇、原子簇等):
\begin{enumerate}
    \item \textbf{确定点群}:识别对称元素(旋转轴、镜面、反演中心)
    \item \textbf{构建可约表示}:$\chi(R) = $ 未移动原子数
    \item \textbf{约化}:$a_i = \frac{1}{h}\sum_R n_R \chi_i(R) \chi(R)$
    \item \textbf{构造对称轨道}:用投影算符或观察法
\end{enumerate}

\textbf{关键技巧}:
\begin{itemize}
    \item $C_{4v}$ vs $D_{3h}$:前者有主轴+垂直镜面,后者有主轴+水平镜面
    \item 反演中心判断:分子中心是否有对称中心
    \item 非等价原子:不能被对称操作互换的原子需分别处理
    \item SO归一化:$\sum c_i^2 = 1$
\end{itemize}

\textbf{常见错误}:
\begin{itemize}
    \item \textbf{点群误判}:忘记检查 $S_n$ 旋转反映轴
    \item \textbf{特征标计算}:只数未移动原子,忘记乘 $\chi_{xyz}$
    \item \textbf{约化公式}:忘记除以总操作数 $h$
\end{itemize}

\subsection{复杂分子点群判定技巧}

\textbf{分步判断法}:
\begin{enumerate}
    \item \textbf{检查线性性}:是否所有原子共线?$\to$ $C_{\infty v}$ 或 $D_{\infty h}$
    \item \textbf{寻找主轴}:最高阶旋转轴 $C_n$($n$ 最大)
    \item \textbf{检查水平镜面}:垂直于主轴 $\to$ $D_{nh}$
    \item \textbf{检查垂直镜面}:$n$ 个包含主轴的镜面 $\to$ $C_{nv}$ 或 $D_{nd}$
    \item \textbf{检查反演中心}:分子中心 $\to$ 添加下标 $g/u$
\end{enumerate}

\textbf{典型难点分子}:
\begin{itemize}
    \item \textbf{N$_2$O$_5$}:在理想化平面构型(考试常用近似)下为 $D_{2h}$(两个$C_2$轴互相垂直 + $\sigma_h$)
    \item \textbf{NO$_3^-$}:平面三角形,$D_{3h}$($C_3$轴 + $\sigma_h$ + 3$\sigma_v$)
    \item \textbf{NO$_2^+$}:线性,$D_{\infty h}$(无限个$C_2$轴垂直于主轴 + $\sigma_h$)
    \item \textbf{弯曲 vs 线性}:AB$_2$型分子,线性 $\to$ $D_{\infty h}$,弯曲 $\to$ $C_{2v}$
\end{itemize}

\textbf{快速检验}:
\begin{itemize}
    \item $D_{nh}$:必须有水平镜面且有$n$个$C_2$轴垂直于主轴
    \item $D_{nd}$:有$n$个垂直镜面但在两个$C_2$轴之间
    \item $C_{nv}$:有$n$个垂直镜面但无$C_2$轴垂直于主轴
    \item $S_n$:旋转反映轴 = $C_n$ + $\sigma_h$ 的组合
\end{itemize}


\section{晶体场与配位化学 (Transition Metals)}

\subsection{八面体配合物 ($O_h$)}
\begin{center}
    \includegraphics[width=\linewidth]{attachments/image_3.png}
\end{center}
\textbf{判断技巧}:先写电子组态,再判断 $\Delta_o$ 大小(光谱化学序列),最后决定高/低自旋。
\begin{itemize}
    \item \textbf{d轨道分裂}:$t_{2g}$ (低能, $d_{xy}, d_{yz}, d_{xz}$) 和 $e_g$ (高能, $d_{z^2}, d_{x^2-y^2}$)。
    \item \textbf{分裂能 $\Delta_o$}:
    \begin{itemize}
        \item $\pi$-供体配体 (如 $Cl^-$): 使得 $t_{2g}$ 变为反键/非键,$\Delta_o$ \textbf{减小} (弱场)。
        \item $\pi$-受体配体 (如 $CO, CN^-$): 使得 $t_{2g}$ 与配体空轨道作用更稳定,$\Delta_o$ \textbf{增大} (强场)。
    \end{itemize}
    \item \textbf{高自旋 vs 低自旋}:
    \begin{itemize}
        \item 取决于 $\Delta_o$ 与成对能 $P$ (或 $K$) 的竞争。
        \item $\Delta_o > P \to$ 低自旋 (强场,电子优先填 $t_{2g}$)。
        \item $\Delta_o < P \to$ 高自旋 (弱场,电子尽可能分占)。
        \item $d^4 - d^7$ 构型需讨论;$d^1-d^3$ 和 $d^8-d^{10}$ 无选择。
        \item \textbf{记忆}:强场配体 ($CO, CN^-, NO_2^-$) $\to$ 低自旋;弱场配体 ($I^-, Br^-, Cl^-, F^-$) $\to$ 高自旋。
    \end{itemize}
    \item \textbf{磁矩计算}:
    \begin{equation}
    \mu_{eff} = \sqrt{n(n+2)} \text{ B.M.}
    \end{equation}
    其中 $n$ 为未成对电子数。\textbf{注意}:先确定高/低自旋,再数未成对电子。
\end{itemize}
\begin{examplebox}
$Fe^{3+}$ ($d^5$) 高自旋:$(t_{2g})^3(e_g)^2$,5个未成对电子 ($n=5$)。\\
$\mu_{eff} = \sqrt{5(5+2)} = \sqrt{35} \approx 5.92$ B.M.\\
低自旋 $Fe^{3+}$:$(t_{2g})^5(e_g)^0$,仅1个未成对电子,$\mu_{eff} = \sqrt{3} \approx 1.73$ B.M.
\end{examplebox}

\subsection{四面体配合物 ($T_d$)}
\begin{center}
    \includegraphics[width=\linewidth]{attachments/image_4.png}
\end{center}

\textbf{d轨道分裂}:
\begin{itemize}
    \item \textbf{能级顺序}:$e$ (低能, $d_{z^2}, d_{x^2-y^2}$) 和 $t_2$ (高能, $d_{xy}, d_{yz}, d_{xz}$)
    \item \textbf{分裂方向}:与八面体\textbf{相反}(配体位于四面体顶点,与$t_2$轨道方向更接近)
    \item \textbf{分裂能}:$\Delta_t \approx \frac{4}{9} \Delta_o$ (更小)
\end{itemize}

\textbf{电子组态特点}:
\begin{itemize}
    \item \textbf{通常为高自旋}:$\Delta_t$小,难以克服成对能$P$
    \item 电子填充顺序:$e^2 t_2^3 \to e^4 t_2^4$ 等
    \item \textbf{判据}:四面体配合物通常为顺磁性
\end{itemize}

\begin{examplebox}
\textbf{$[CoCl_4]^{2-}$ ($Co^{2+}, d^7$) 四面体}:\\
电子组态:$(e)^4(t_2)^3$,3个未成对电子\\
$\mu_{eff} = \sqrt{3(3+2)} = \sqrt{15} \approx 3.87$ B.M.(顺磁)\\
CFSE = $4(-0.6\Delta_t) + 3(0.4\Delta_t) = -1.2\Delta_t$
\end{examplebox}

\subsection{平面正方形配合物 ($D_{4h}$)}

\textbf{几何特征}:
\begin{itemize}
    \item 4个配体位于正方形四个顶点
    \item 常见于\textbf{$d^8$金属}:$Ni^{2+}, Pd^{2+}, Pt^{2+}$
    \item 可视为八面体沿$z$轴拉伸,失去两个轴向配体
\end{itemize}

\textbf{d轨道分裂}(能量从低到高):
\begin{itemize}
    \item $d_{xy}$ < $d_{xz}, d_{yz}$ < $d_{z^2}$ < $d_{x^2-y^2}$
    \item $d_{x^2-y^2}$能量最高(直接指向配体)
    \item $d_{z^2}$次之(轴向,但配体已移除)
\end{itemize}

\textbf{电子组态与性质}:
\begin{itemize}
    \item \textbf{$d^8$配置}:$(d_{xy})^2(d_{xz},d_{yz})^4(d_{z^2})^2(d_{x^2-y^2})^0$
    \item \textbf{通常为低自旋}:分裂能大,电子全部成对
    \item \textbf{判据}:平面正方形通常为抗磁性(无未成对电子)
    \item 符合16电子规则(8个d电子 + 8个配体电子)
\end{itemize}

\begin{examplebox}
\textbf{$[Ni(CN)_4]^{2-}$ ($Ni^{2+}, d^8$) 平面正方形}:\\
强场配体$CN^-$导致大分裂能\\
电子全部成对,$\mu_{eff} = 0$ B.M.(抗磁)\\
\\
\textbf{对比}:$[NiCl_4]^{2-}$(弱场)为四面体,顺磁
\end{examplebox}

\textbf{四面体 vs 平面正方形判断}:
\begin{itemize}
    \item \textbf{配体场强}:强场配体倾向平面正方形,弱场倾向四面体
    \item \textbf{金属周期}:第二、三过渡系(4d, 5d)更倾向平面正方形
    \item \textbf{磁性判据}:抗磁性$\to$平面正方形,顺磁性$\to$四面体
    \item \textbf{CFSE比较}:平面正方形的CFSE通常更大($d^8$时)
\end{itemize}

\textbf{CFSE 计算示例}:
\begin{examplebox}
\textbf{$d^6$ 配合物比较} ($Fe^{2+}, Co^{3+}$):\\
\textbf{高自旋 $O_h$}: $(t_{2g})^4(e_g)^2$\\
CFSE = $4(-0.4\Delta_o) + 2(0.6\Delta_o) = -0.4\Delta_o$\\
\textbf{低自旋 $O_h$}: $(t_{2g})^6(e_g)^0$ + 成对能 $P$\\
CFSE = $6(-0.4\Delta_o) = -2.4\Delta_o$,但需克服 $2P$\\
净稳定 = $-2.4\Delta_o + 2P$\\
\textbf{判据}:若 $\Delta_o > P$,低自旋更稳定(强场配体)。\\
\textbf{应用}:$[Fe(CN)_6]^{4-}$ (低自旋,抗磁);$[Fe(H_2O)_6]^{2+}$ (高自旋,顺磁)。
\end{examplebox}

\subsection{配体$\pi$相互作用对$\Delta_o$的影响}
\begin{center}
    \includegraphics[width=\linewidth]{attachments/PixPin_2025-12-25_17-32-28.png}
\end{center}
\textbf{光谱化学序列}($\Delta_o$ 从小到大):
\begin{center}
\scriptsize
$I^- < Br^- < S^{2-} < SCN^- < Cl^- < NO_3^- < F^- < OH^- < H_2O < NCS^- < NH_3 < en < NO_2^- < CN^- \approx CO$
\end{center}
\textbf{$\pi$-供体配体}(如 $Cl^-, Br^-, I^-, OH^-$):
\begin{itemize}
    \item 配体有填充的$\pi$轨道(如卤素的$p$轨道)
    \item 与金属$t_{2g}$轨道发生$\pi$-供体作用
    \item $t_{2g}$轨道能量升高(反键作用)
    \item \textbf{结果}:$\Delta_o$ \textbf{减小},成为弱场配体
\end{itemize}

\textbf{$\pi$-受体配体}(如 $CO, CN^-, NO_2^-$):
\begin{itemize}
    \item 配体有空的$\pi^*$反键轨道
    \item 与金属$t_{2g}$轨道发生$\pi$-反馈键(back-bonding)
    \item $t_{2g}$轨道能量降低(成键作用)
    \item \textbf{结果}:$\Delta_o$ \textbf{增大},成为强场配体
\end{itemize}

\textbf{$\sigma$-供体配体}(如 $H_2O, NH_3$):
\begin{itemize}
    \item 仅提供$\sigma$电子对,无显著$\pi$作用
    \item $\Delta_o$中等,取决于配体$\sigma$供电子能力
\end{itemize}

\begin{keypoint}
\textbf{配体场强排序规律}:
\begin{itemize}
    \item $\pi$-受体(强场)> $\sigma$-供体(中等场)> $\pi$-供体(弱场)
    \item $\pi$作用是光谱化学序列的核心原因
    \item 同族元素:I$^-$ < Br$^-$ < Cl$^-$ < F$^-$($\pi$-供体能力递减)
\end{itemize}
\end{keypoint}

\subsection{CFSE计算与构型判断}

\textbf{CFSE计算公式}:
\begin{itemize}
    \item 八面体:CFSE = $(-0.4n_{t_{2g}} + 0.6n_{e_g})\Delta_o + P \times$(成对数)
    \item 四面体:CFSE = $(-0.6n_e + 0.4n_{t_2})\Delta_t$
    \item 平面正方形:需单独计算 $d_{x^2-y^2}$ 轨道能量
\end{itemize}

\textbf{高/低自旋判断准则}:
\begin{itemize}
    \item $\Delta_o > P$(成对能)$\to$ 低自旋
    \item $\Delta_o < P$ $\to$ 高自旋
    \item 强场配体(CN$^-$, CO)$\to$ 大$\Delta_o$ $\to$ 低自旋
    \item 弱场配体(H$_2$O, F$^-$)$\to$ 小$\Delta_o$ $\to$ 高自旋
    \item $d^4 \sim d^7$ 存在高低自旋之分
\end{itemize}


\section{Hückel 分子轨道理论 (HMO)}
\textbf{适用范围}:平面共轭 $\pi$ 体系。\textbf{核心假设}:$\sigma-\pi$ 分离,只考虑 $p_z$ 轨道相互作用。

\textbf{重要结论}:
\begin{itemize}
    \item \textbf{Hückel芳香性规则}:$4n+2$ 个 $\pi$ 电子的环状体系特别稳定(苯$n=1$,环辛四烯阳离子$n=1$)。
    \item \textbf{反芳香性}:$4n$ 个 $\pi$ 电子的环状体系不稳定(环丁二烯,环辛四烯)。
    \item \textbf{成键/反键判据}:$E < \alpha$ 为成键($x < 0$),$E > \alpha$ 为反键($x > 0$)。
    \item \textbf{离域能}:比较共轭体系与孤立双键的能量差,$\beta < 0$ 意味着共轭稳定。
\end{itemize}

\begin{keypoint}
\textbf{久期方程 (Secular Equations):}
\begin{equation}
|H_{ij} - ES_{ij}| = 0
\end{equation}
Hückel 近似:
\begin{itemize}
    \item $S_{ii} = 1, S_{ij} = 0 (i \neq j)$
    \item $H_{ii} = \alpha$ (库仑积分)
    \item $H_{ij} = \beta$ (相邻原子), $0$ (不相邻)
\end{itemize}
变量代换:$x = \frac{\alpha - E}{\beta}$。
\end{keypoint}

\begin{examplebox}
\textbf{乙烯} ($C_2H_4$) $\pi$ 体系:\\
行列式 $\begin{vmatrix} \alpha-E & \beta \\ \beta & \alpha-E \end{vmatrix} = 0$。\\
令 $x = (\alpha-E)/\beta$,$\begin{vmatrix} x & 1 \\ 1 & x \end{vmatrix} = x^2 - 1 = 0 \Rightarrow x = \pm 1$。\\
能量:$E = \alpha \pm \beta$。离域能 = $2(\alpha+\beta) - 2\alpha = 2\beta$ ($\beta < 0$,稳定)。
\end{examplebox}

\textbf{解题技巧}:
\begin{itemize}
    \item 对称分子优先用对称性简化(分块对角化)。
    \item 环状分子直接用 Frost 圆或能级公式 $E_k = \alpha + 2\beta\cos(2k\pi/n)$。
    \item 计算离域能:$E_{\pi}$(共轭)$-E_{\pi}$(孤立双键)。
    \item 成键轨道:$x < 0$ ($E < \alpha$);反键:$x > 0$ ($E > \alpha$)。(注:基于 $x = (\alpha-E)/\beta$ 且 $\beta < 0$)
\end{itemize}

\begin{examplebox}
\textbf{环状多烯 ($C_n H_n$) 通用公式}:
对于 $n$ 个碳原子的单环共轭体系,Hückel 能级公式为:
\begin{equation}
E_k = \alpha + 2\beta \cos\left(\frac{2k\pi}{n}\right), \quad k = 0, \pm 1, \pm 2, \dots
\end{equation}
\textbf{实例:苯 ($C_6H_6$, $n=6$)}:\\
\begin{itemize}
    \item $k=0$: $E = \alpha + 2\beta$ ($A_{2u}$, 最低,成键)
    \item $k=\pm 1$: $E = \alpha + \beta$ ($E_{1g}$, 简并成键)
    \item $k=\pm 2$: $E = \alpha - \beta$ ($E_{2u}$, 简并反键)
    \item $k=3$: $E = \alpha - 2\beta$ ($B_{2g}$, 最高反键)
\end{itemize}
6个 $\pi$ e$^-$ 填充:$A_{2u}^2 E_{1g}^4$,全占满成键 MO。\\
离域能 = $2(\alpha+2\beta) + 4(\alpha+\beta) - 6\alpha = 8\beta$。\\
\textbf{Frost圆}:正六边形内接圆,一顶点在下,圆心在 $\alpha$。
\end{examplebox}

\subsection{环形分子快速计算法}

\textbf{$C_nH_n$ 通用能级公式}:
\begin{equation}
E_k = \alpha + 2\beta\cos\left(\frac{2k\pi}{n}\right), \quad k = 0, \pm1, \pm2, \ldots, \pm\lfloor\frac{n-1}{2}\rfloor
\end{equation}
(对于偶数 $n$,还需包括 $k = n/2$,此能级非简并)

\textbf{快速判断}:
\begin{itemize}
    \item \textbf{Frost圆法}:正$n$边形内接圆,一顶点向下,圆心在$\alpha$
    \item \textbf{基态电子数}:$4n+2$ 芳香(稳定),$4n$ 反芳香(不稳定)
    \item \textbf{离域能}:总能量 - 孤立双键能量
    \item \textbf{能级间距}:$n$ 越大,能级越密集
\end{itemize}

\textbf{计算要点}:
\begin{itemize}
    \item $k=0$ 永远是最低能级($\alpha + 2\beta$)
    \item 奇数$n$:无简并;偶数$n$:成对简并(除$k=0$和$k=n/2$)
    \item 电子填充:由低到高,简并轨道先各填一个(Hund)
    \item 净$\pi$能:$E_\pi = \sum n_i E_i$($n_i$为占据数)
\end{itemize}

\textbf{常见错误}:
\begin{itemize}
    \item \textbf{Hückel$x$的定义}:$x = \frac{\alpha - E}{\beta}$,不是 $\frac{E - \alpha}{\beta}$
    \item \textbf{成键/反键}:$x < 0$ 成键($E < \alpha$),$x > 0$ 反键(因 $\beta < 0$)
    \item \textbf{环状分子$k$范围}:$k$ 可正可负,但总能级数 = $n$
    \item \textbf{简并判断}:$\pm k$ 对应同一能量(除 $k=0$ 和 $k=n/2$)
\end{itemize}

\subsection{典型体系}
\begin{itemize}
    \item \textbf{烯丙基 (Allyl, $C_3$)}:
    \begin{itemize}
        \item 行列式:$\begin{vmatrix} x & 1 & 0 \\ 1 & x & 1 \\ 0 & 1 & x \end{vmatrix} = 0 \Rightarrow x(x^2-2)=0$
        \item 根:$x = 0, \pm\sqrt{2}$。
        \item 能量:$E = \alpha + \sqrt{2}\beta$ (成键), $\alpha$ (非键), $\alpha - \sqrt{2}\beta$ (反键)。
    \end{itemize}
    \item \textbf{环丙烯基 ($C_3$ 环状)}:
    \begin{itemize}
        \item 能量:$\alpha + 2\beta$ (最低,非简并), $\alpha - \beta$ (双重简并)。
        \item $C_3H_3^+$ (环丙烯基阳离子): 2个 $\pi$ 电子填入 $\alpha + 2\beta$,符合 Hückel $4n+2$ 芳香性规则 ($n=0$),非常稳定。
    \end{itemize}
    \item \textbf{丁二烯 (Butadiene)}: $C_{2h}$ 对称性简化计算。
\end{itemize}

\subsection{复杂体系处理: 环状分子 $C_2O_2$ ($D_{2h}$)}
\begin{examplebox}
\textbf{分块对角化 (Block Diagonalization)}:
对于 $D_{2h}$ 平面四元环,利用对称性组合轨道 (SALCs) 将 $4 \times 4$ 久期行列式分解。
\begin{itemize}
    \item \textbf{对称性分析}:4个 $p_z$ 轨道。按 $D_{2h}$ 操作作用于基组,得到可约表示 $\Gamma_\pi$。
    \item \textbf{分解}:$\Gamma_\pi = B_{1u} + B_{2g} + B_{3g} + A_u$ (示例)。
    这意味着久期方程分解为 4 个 $1 \times 1$ 的方程(若无简并),直接得到能量表达式。
    \item \textbf{能量计算}:
    $E_i = \frac{H_{ii}}{S_{ii}} = \alpha + c_i \beta$。
    若涉及异核原子 (O, C),则 $\alpha$ 取值不同 ($\alpha_O = \alpha + \beta, \alpha_C = \alpha$),需在行列式中体现。
\end{itemize}
\end{examplebox}

\end{multicols}

% \clearpage

\noindent\textbf{\Large 附录:常用数据与公式速查}

\begin{multicols}{2}
\scriptsize

\textbf{1. 特征标快速记忆}:
\begin{itemize}
    \item $\chi_{xyz}(E)=3$, $\chi_{xyz}(C_2)=-1$, $\chi_{xyz}(\sigma)=1$, $\chi_{xyz}(i)=-3$
    \item $\chi_{xyz}(C_3)=0$, $\chi_{xyz}(C_4)=1$, $\chi_{xyz}(S_4)=-1$, $\chi_{xyz}(S_6)=0$
\end{itemize}

\textbf{2. 光谱化学序列}($\Delta_o$ 从小到大):
\begin{center}
$I^- < Br^- < S^{2-} < SCN^- < Cl^- < NO_3^- < F^- < OH^- < H_2O < NCS^- < NH_3 < en < NO_2^- < CN^- \approx CO$
\end{center}

\textbf{3. 常见d电子构型磁矩}(仅自旋):
\begin{itemize}
    \item $d^1$: 1.73 BM; $d^2$: 2.83 BM; $d^3$: 3.87 BM
    \item $d^4$高: 4.90 BM, 低: 2.83 BM
    \item $d^5$高: 5.92 BM, 低: 1.73 BM
    \item $d^6$高: 4.90 BM, 低: 0 BM (抗磁)
    \item $d^7$高: 3.87 BM, 低: 1.73 BM
\end{itemize}

\textbf{4. 晶体场分裂能比}:
\begin{itemize}
    \item $\Delta_t \approx \frac{4}{9}\Delta_o$ (四面体约为八面体的4/9)
    \item $\Delta_{sq} > \Delta_o$ (平面正方形大于八面体)
\end{itemize}

\textbf{5. Hückel重要体系能量}:
\begin{itemize}
    \item 乙烯:$\alpha \pm \beta$
    \item 烯丙基:$\alpha \pm \sqrt{2}\beta$, $\alpha$
    \item 丁二烯:$\alpha \pm 1.618\beta$, $\alpha \pm 0.618\beta$
    \item 苯:$\alpha + 2\beta$, $\alpha + \beta$(2), $\alpha - \beta$(2), $\alpha - 2\beta$
    \item 环丁二烯:$\alpha + 2\beta$, $\alpha$(2), $\alpha - 2\beta$
\end{itemize}

\textbf{6. 快速检验技巧}:
\begin{itemize}
    \item \textbf{能量求和}:$\sum E_i = n\alpha$(环状分子)
    \item \textbf{轨道数守恒}:$n$ 个AO $\to$ $n$ 个MO
    \item \textbf{电子数守恒}:总电子数 = $\sum 2 \times$(占据轨道数)
    \item \textbf{对称性检验}:MO必须属于某个不可约表示
    \item \textbf{键级合理性}:单键$\approx$1,双键$\approx$2,三键$\approx$3
    \item \textbf{磁性判断}:有未成对电子$\to$顺磁,全配对$\to$抗磁
    \item \textbf{CFSE正负}:$t_{2g}$ 稳定(负),$e_g$ 不稳定(正)
    \item \textbf{离域能符号}:$\beta < 0$,离域能为负(稳定)
\end{itemize}



\end{multicols}

\end{document}

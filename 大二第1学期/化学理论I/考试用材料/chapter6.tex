% 编译提示:使用 XeLaTeX 编译
% 需要 ctex 宏包支持中文

\documentclass[a4paper,9pt]{article}
\usepackage{amsmath}
\usepackage{amssymb}
\usepackage{geometry}
\usepackage{ctex}
\usepackage{xcolor}
\usepackage{multicol}
\usepackage{titlesec}
\usepackage{fancyhdr}
\usepackage{tcolorbox}

\geometry{left=1cm,right=1cm,top=1cm,bottom=1.8cm}

\setlength{\columnsep}{0.8cm}
\setlength{\columnseprule}{0.4pt}
\renewcommand{\columnseprulecolor}{\color{gray!50}}

% 定义颜色方案
\definecolor{sectioncolor}{RGB}{0,102,204}      % 深蓝色 - 用于章节标题
\definecolor{subsectioncolor}{RGB}{0,153,153}   % 青色 - 用于小节标题
\definecolor{keycolor}{RGB}{204,102,0}          % 橙色 - 用于关键点
\definecolor{derivcolor}{RGB}{100,149,237}      % 浅蓝色 - 用于推导框
\definecolor{boxbg}{RGB}{240,248,255}           % 淡蓝色背景
\definecolor{keybg}{RGB}{255,248,240}           % 淡橙色背景

% 自定义章节标题格式
\titleformat{\section}
  {\normalfont\large\bfseries\color{sectioncolor}}
  {\thesection}{0.5em}{}
\titlespacing*{\section}{0pt}{1ex plus 0.5ex minus 0.2ex}{0.8ex plus 0.2ex}

\titleformat{\subsection}
  {\normalfont\normalsize\bfseries\color{subsectioncolor}}
  {\thesubsection}{0.5em}{}
\titlespacing*{\subsection}{0pt}{0.8ex plus 0.3ex minus 0.1ex}{0.5ex plus 0.1ex}

% 自定义关键点环境
\newenvironment{keypoint}
{\par\vspace{0.3em}\noindent\begin{tcolorbox}[
  colback=keybg,
  colframe=keycolor,
  boxrule=0.5pt,
  arc=2pt,
  left=2pt,
  right=2pt,
  top=2pt,
  bottom=2pt,
  boxsep=0pt]
\small\noindent\textcolor{keycolor}{\textbf{关键点:}}}
{\end{tcolorbox}\par\vspace{0.3em}}

% 自定义推导环境
\newenvironment{derivation}
{\par\vspace{0.3em}\noindent\begin{tcolorbox}[
  colback=boxbg,
  colframe=derivcolor,
  boxrule=0.5pt,
  arc=2pt,
  left=2pt,
  right=2pt,
  top=2pt,
  bottom=2pt,
  boxsep=0pt]
\scriptsize\noindent\textcolor{derivcolor}{\textbf{推导:}}\par}
{\end{tcolorbox}\par\vspace{0.3em}}

% 页眉页脚设置
\pagestyle{fancy}
\fancyhf{}
\fancyfoot[C]{\scriptsize6-\thepage}
\renewcommand{\headrulewidth}{0pt}
\renewcommand{\footrulewidth}{0.4pt}
\renewcommand{\footrule}{\hbox to\headwidth{\color{gray!50}\leaders\hrule height \footrulewidth\hfill}}

% 紧凑标题
\title{\vspace{-2em}\Large\bfseries\color{sectioncolor} Chapter 6 Collections - 氢原子\vspace{-1em}}
\author{}
\date{2025 年 11 月 1 日}

\begin{document}

\maketitle
\thispagestyle{fancy}

\begin{multicols}{2}

\section{氢原子的薛定谔方程}

\subsection{基本方程}

氢原子是唯一可精确求解的多粒子体系。在质心坐标系下,问题简化为单粒子问题。

哈密顿算符:
\begin{equation}
\hat{H} = -\frac{\hbar^2}{2\mu}\nabla^2 - \frac{Ze^2}{4\pi\varepsilon_0 r}
\end{equation}

其中 $\mu = \frac{m_e M}{m_e + M}$ 为约化质量,对氢原子 $\mu \approx m_e$。

球坐标下的拉普拉斯算符:
\begin{equation}
\nabla^2 = \frac{1}{r^2}\frac{\partial}{\partial r}\left(r^2\frac{\partial}{\partial r}\right) + \frac{1}{r^2\sin\theta}\frac{\partial}{\partial\theta}\left(\sin\theta\frac{\partial}{\partial\theta}\right) + \frac{1}{r^2\sin^2\theta}\frac{\partial^2}{\partial\phi^2}
\end{equation}

\subsection{分离变量法}

由于势能仅依赖于 $r$,具有球对称性,可分离变量:
\begin{equation}
\psi(r,\theta,\phi) = R(r)Y(\theta,\phi)
\end{equation}

代入薛定谔方程,可得三个独立的常微分方程,引出三个量子数:
\begin{itemize}
\item $n = 1,2,3,\ldots$(主量子数)
\item $l = 0,1,2,\ldots,n-1$(角量子数)
\item $m = 0,\pm 1,\pm 2,\ldots,\pm l$(磁量子数)
\end{itemize}

\subsection{能量本征值}

\begin{equation}
E_n = -\frac{\mu e^4 Z^2}{32\pi^2\varepsilon_0^2\hbar^2 n^2} = -\frac{Z^2}{n^2}\times 13.6\text{ eV}
\end{equation}

\textbf{重要特征}:能量仅与 $n$ 有关,与 $l, m$ 无关(简并)。

玻尔半径:$a_0 = \frac{4\pi\varepsilon_0\hbar^2}{m_e e^2} = 0.529$ \AA

\section{角度部分 - 球谐函数}

\subsection{方位角方程}

方程 $\frac{d^2\Phi}{d\phi^2} = -m^2\Phi$ 的解为:
\begin{equation}
\Phi_m(\phi) = \frac{1}{\sqrt{2\pi}}e^{im\phi}
\end{equation}

单值性要求 $\Phi(\phi+2\pi) = \Phi(\phi)$,故 $m$ 必须为整数。

\subsection{连带勒让德函数}

极角方程的解为连带勒让德函数 $P_l^m(\cos\theta)$。

\begin{keypoint}
勒让德多项式的罗德里格斯公式:
\begin{equation}
P_l(x) = \frac{1}{2^l l!}\frac{d^l}{dx^l}(x^2-1)^l
\end{equation}
\end{keypoint}

连带勒让德函数定义:
\begin{equation}
P_l^m(x) = (1-x^2)^{|m|/2}\frac{d^{|m|}}{dx^{|m|}}P_l(x)
\end{equation}

重要递推关系:
\begin{align}
(l+1)P_{l+1}(x) &= (2l+1)xP_l(x) \\
&\quad - lP_{l-1}(x) \\
(2l+1)xP_l^m(x) &= (l-m+1)P_{l+1}^m(x) \\
&\quad + (l+m)P_{l-1}^m(x)
\end{align}

示例:$P_0(x) = 1$,$P_1(x) = x$,\\
$P_2(x) = \frac{1}{2}(3x^2-1)$

\subsection{球谐函数}

完整形式:
\begin{equation}
Y_l^m(\theta,\phi) = (-1)^m\left[\frac{(2l+1)}{4\pi}\frac{(l-|m|)!}{(l+|m|)!}\right]^{1/2}P_l^{|m|}(\cos\theta)e^{im\phi}
\end{equation}

\begin{keypoint}
球谐函数是 $\hat{L}^2$ 和 $\hat{L}_z$ 的共同本征函数,构成完备正交归一系。
\end{keypoint}

正交归一性:
\begin{equation}
\int_0^{\pi}\int_0^{2\pi}Y_l^{m*}Y_{l'}^{m'}\sin\theta\,d\theta\,d\phi = \delta_{ll'}\delta_{mm'}
\end{equation}

常用示例(归一化):
\begin{align*}
Y_0^0 &= \frac{1}{2\sqrt{\pi}}, \quad Y_1^0 = \sqrt{\frac{3}{4\pi}}\cos\theta \\
Y_1^{\pm 1} &= \mp\sqrt{\frac{3}{8\pi}}\sin\theta e^{\pm i\phi}, \quad Y_2^0 = \sqrt{\frac{5}{16\pi}}(3\cos^2\theta-1)
\end{align*}

\section{径向波函数}

\subsection{径向方程求解}

径向方程:
\begin{equation}
-\frac{\hbar^2}{2\mu}\frac{1}{r^2}\frac{d}{dr}\left(r^2\frac{dR}{dr}\right) + \left[\frac{l(l+1)\hbar^2}{2\mu r^2} - \frac{Ze^2}{4\pi\varepsilon_0 r}\right]R = ER
\end{equation}

引入 $u(r) = rR(r)$ 和无量纲变量 $\rho = \frac{2Zr}{na_0}$,方程化为:
\begin{equation}
\frac{d^2u}{d\rho^2} = \left[\frac{l(l+1)}{\rho^2} - \frac{n}{Z\rho} + \frac{1}{4}\right]u
\end{equation}

渐近行为分析:$\rho \to \infty$ 时 $u \sim e^{-\rho/2}$;$\rho \to 0$ 时 $u \sim \rho^{l+1}$。

设 $u(\rho) = \rho^{l+1}e^{-\rho/2}F(\rho)$,代入后得到连带拉盖尔方程。

\subsection{连带拉盖尔多项式}

拉盖尔多项式:
\begin{equation}
L_n(x) = e^x\frac{d^n}{dx^n}(x^n e^{-x})
\end{equation}

连带拉盖尔多项式:
\begin{equation}
L_n^k(x) = \frac{d^k}{dx^k}L_{n+k}(x)
\end{equation}

正交性(权函数 $e^{-x}x^k$):
\begin{equation}
\int_0^{\infty}e^{-x}x^k L_n^k(x)L_m^k(x)\,dx = \frac{(n+k)!}{n!}\delta_{nm}
\end{equation}

\subsection{归一化径向波函数}

通解:
\begin{equation}
\begin{split}
R_{nl}(r) &= -\left[\frac{(n-l-1)!}{2n[(n+l)!]^3}\right]^{1/2}\left(\frac{2Z}{na_0}\right)^{3/2}\left(\frac{2Zr}{na_0}\right)^l e^{-Zr/na_0} \\
&\quad \times L_{n-l-1}^{2l+1}\left(\frac{2Zr}{na_0}\right)
\end{split}
\end{equation}

示例($Z=1$):
\begin{align*}
R_{10} &= 2a_0^{-3/2}e^{-r/a_0} \\
R_{20} &= (2a_0)^{-3/2}\frac{1}{\sqrt{2}}\left(2-\frac{r}{a_0}\right)e^{-r/2a_0} \\
R_{21} &= (2a_0)^{-3/2}\frac{1}{\sqrt{24}}\frac{r}{a_0}e^{-r/2a_0}
\end{align*}

径向分布函数:$D(r) = r^2|R_{nl}(r)|^2$(球壳内概率密度)

\section{角动量算符理论}

\subsection{算符定义与对易关系}

经典角动量:$\vec{L} = \vec{r} \times \vec{p}$

量子算符(直角坐标):
\begin{equation}
\hat{L}_x = -i\hbar\left(y\frac{\partial}{\partial z}-z\frac{\partial}{\partial y}\right)
\end{equation}
(循环)

球坐标形式:
\begin{align}
\hat{L}_z &= -i\hbar\frac{\partial}{\partial\phi} \\
\hat{L}^2 &= -\hbar^2\left[\frac{1}{\sin\theta}\frac{\partial}{\partial\theta}\left(\sin\theta\frac{\partial}{\partial\theta}\right)+\frac{1}{\sin^2\theta}\frac{\partial^2}{\partial\phi^2}\right]
\end{align}

\begin{keypoint}
对易关系是角动量理论的核心:
\begin{align}
[\hat{L}_i,\hat{L}_j] &= i\hbar\varepsilon_{ijk}\hat{L}_k \\
[\hat{L}^2,\hat{L}_i] &= 0
\end{align}
\end{keypoint}

\begin{derivation}
证明 $[\hat{L}_x,\hat{L}_y] = i\hbar\hat{L}_z$:

利用 $[\hat{x}_i,\hat{p}_j] = i\hbar\delta_{ij}$,
\begin{align*}
[\hat{L}_x,\hat{L}_y] &= [\hat{y}\hat{p}_z-\hat{z}\hat{p}_y, \hat{z}\hat{p}_x-\hat{x}\hat{p}_z] \\
&= \hat{y}[\hat{p}_z,\hat{z}]\hat{p}_x + \hat{x}[\hat{z},\hat{p}_z]\hat{p}_y \\
&= \hat{y}(-i\hbar)\hat{p}_x + \hat{x}(i\hbar)\hat{p}_y \\
&= i\hbar(\hat{x}\hat{p}_y-\hat{y}\hat{p}_x) = i\hbar\hat{L}_z
\end{align*}
\end{derivation}

\subsection{本征值问题}

由于 $[\hat{L}^2,\hat{L}_z] = 0$,可选择共同本征函数:
\begin{align}
\hat{L}^2Y_l^m &= l(l+1)\hbar^2Y_l^m \\
\hat{L}_zY_l^m &= m\hbar Y_l^m
\end{align}

物理意义:
\begin{itemize}
\item 角动量大小:$|\vec{L}| = \hbar\sqrt{l(l+1)}$(量子化)
\item $z$ 分量:$L_z = m\hbar$(空间量子化)
\item 不确定关系:$\Delta L_x \cdot \Delta L_y \geq \frac{\hbar}{2}|L_z|$($L_x, L_y$ 不能同时确定)
\end{itemize}

\subsection{升降算符方法}

定义阶梯算符:
\begin{equation}
\hat{L}_\pm = \hat{L}_x \pm i\hat{L}_y
\end{equation}

对易关系:
\begin{align}
[\hat{L}_z,\hat{L}_\pm] &= \pm\hbar\hat{L}_\pm \\
[\hat{L}^2,\hat{L}_\pm] &= 0
\end{align}

\begin{derivation}
证明 $\hat{L}_+$ 的作用效果:

设 $\hat{L}_z f = m\hbar f$,作用 $\hat{L}_z$ 于 $\hat{L}_+ f$:
\begin{align*}
\hat{L}_z(\hat{L}_+ f) &= (\hat{L}_+\hat{L}_z + [\hat{L}_z,\hat{L}_+])f \\
&= (\hat{L}_+\hat{L}_z + \hbar\hat{L}_+)f \\
&= (m+1)\hbar(\hat{L}_+ f)
\end{align*}
即 $\hat{L}_+ f$ 是本征值为 $(m+1)\hbar$ 的本征函数。
\end{derivation}

作用规律(含归一化系数):
\begin{align}
\hat{L}_+Y_l^m &= \hbar\sqrt{l(l+1)-m(m+1)}\\
&\quad \times Y_l^{m+1} \\
\hat{L}_-Y_l^m &= \hbar\sqrt{l(l+1)-m(m-1)}\\
&\quad \times Y_l^{m-1}
\end{align}

重要恒等式:
\begin{align}
\hat{L}^2 &= \hat{L}_-\hat{L}_+ + \hat{L}_z^2 + \hbar\hat{L}_z \\
&= \hat{L}_+\hat{L}_- + \hat{L}_z^2 - \hbar\hat{L}_z
\end{align}

\section{算符平均值与不确定性}

\subsection{物理量平均值计算}

对于归一化波函数 $\psi$,物理量 $\hat{A}$ 的平均值:
\begin{equation}
\langle A \rangle = \int \psi^* \hat{A} \psi \, d\tau
\end{equation}

对于氢原子波函数 $\psi_{nlm} = R_{nl}(r)Y_l^m(\theta,\phi)$:
\begin{equation}
\langle A \rangle = \int_0^{\infty}\int_0^{\pi}\int_0^{2\pi} R_{nl}^* Y_l^{m*} \hat{A} (R_{nl}Y_l^m) r^2\sin\theta\,dr\,d\theta\,d\phi
\end{equation}

\begin{derivation}
\textbf{示例}:计算氢原子基态的 $\langle r \rangle$ 和 $\langle r^2 \rangle$。

基态波函数:$\psi_{100} = \frac{1}{\sqrt{\pi a_0^3}}e^{-r/a_0}$

\begin{align*}
\langle r \rangle &= \int_0^{\infty} R_{10}^2 \cdot r \cdot r^2\,dr \\
&= \frac{4}{a_0^3}\int_0^{\infty} r^3 e^{-2r/a_0}\,dr \\
&= \frac{4}{a_0^3} \cdot \frac{3!}{(2/a_0)^4} = \frac{3a_0}{2}
\end{align*}

\begin{align*}
\langle r^2 \rangle &= \frac{4}{a_0^3}\int_0^{\infty} r^4 e^{-2r/a_0}\,dr \\
&= \frac{4}{a_0^3} \cdot \frac{4!}{(2/a_0)^5} = 3a_0^2
\end{align*}

标准偏差:
\begin{equation*}
\Delta r = \sqrt{\langle r^2 \rangle - \langle r \rangle^2} = \frac{\sqrt{3}}{2}a_0
\end{equation*}
\end{derivation}

\subsection{常见算符运算关系}

量子力学中,算符之间存在重要的运算关系。

\begin{keypoint}
海森堡运动方程:
\begin{equation}
\frac{d\hat{A}}{dt} = \frac{1}{i\hbar}[\hat{A},\hat{H}] + \frac{\partial\hat{A}}{\partial t}
\end{equation}
\end{keypoint}

\begin{derivation}
\textbf{示例1}:证明 $\frac{d\langle\vec{r}\rangle}{dt} = \frac{\langle\vec{p}\rangle}{m}$

对位置算符,$\hat{H} = \frac{\hat{p}^2}{2m} + V(\hat{r})$:
\begin{align*}
\frac{d\hat{x}}{dt} &= \frac{1}{i\hbar}[\hat{x},\frac{\hat{p}_x^2}{2m}]
\end{align*}

利用 $[\hat{x},\hat{p}_x] = i\hbar$,可得:
\begin{equation*}
[\hat{x},\hat{p}_x^2] = 2i\hbar\hat{p}_x
\end{equation*}

因此:$\frac{d\hat{x}}{dt} = \frac{\hat{p}_x}{m}$
\end{derivation}

\begin{derivation}
\textbf{示例2}:埃伦费斯特定理 $\frac{d\langle\vec{p}\rangle}{dt} = -\left\langle\nabla V\right\rangle$

对动量算符:
\begin{equation*}
\frac{d\hat{p}_x}{dt} = \frac{1}{i\hbar}[\hat{p}_x,V(\hat{x})]
\end{equation*}

作用于波函数:
\begin{align*}
[\hat{p}_x,V]\psi &= -i\hbar\frac{\partial V}{\partial x}\psi
\end{align*}

故 $\frac{d\hat{p}_x}{dt} = -\frac{\partial V}{\partial x}$
\end{derivation}

\textbf{常用算符对易关系}:
\begin{align}
[\hat{x}_i,\hat{p}_j] &= i\hbar\delta_{ij} \\
[\hat{x},\hat{p}^2] &= 2i\hbar\hat{p} \\
[\hat{r}^2,\hat{p}_r] &= 2i\hbar\hat{r} \\
[\hat{L}_i,\hat{x}_j] &= i\hbar\varepsilon_{ijk}\hat{x}_k \\
[\hat{L}_i,\hat{p}_j] &= i\hbar\varepsilon_{ijk}\hat{p}_k \\
[\hat{L}^2,\hat{r}] &= [\hat{L}^2,\hat{p}] = 0
\end{align}

\subsection{角动量不确定性关系}

\begin{keypoint}
对于不对易算符 $[\hat{A},\hat{B}] = i\hat{C}$,广义不确定关系:
\begin{equation}
\Delta A \cdot \Delta B \geq \frac{1}{2}|\langle C \rangle|
\end{equation}
\end{keypoint}

\begin{derivation}
证明 $\Delta L_x \cdot \Delta L_y \geq \frac{\hbar}{2}|L_z|$:

由 $[\hat{L}_x,\hat{L}_y] = i\hbar\hat{L}_z$:
\begin{equation*}
\Delta L_x \cdot \Delta L_y \geq \frac{\hbar}{2}|\langle L_z \rangle|
\end{equation*}

对于本征态 $Y_l^m$,$\langle L_z \rangle = m\hbar$,故:
\begin{equation*}
\Delta L_x \cdot \Delta L_y \geq \frac{\hbar^2|m|}{2}
\end{equation*}

物理意义:当 $L_z$ 确定时,$L_x$ 和 $L_y$ 完全不确定。
\end{derivation}

\begin{derivation}
\textbf{进一步}:计算角动量分量的涨落。

对于态 $Y_l^m$:
\begin{align*}
\langle L_z \rangle &= m\hbar, \quad \langle L_z^2 \rangle = m^2\hbar^2 \\
\langle L^2 \rangle &= l(l+1)\hbar^2
\end{align*}

由于对称性,$\langle L_x \rangle = \langle L_y \rangle = 0$,且 $\langle L_x^2 \rangle = \langle L_y^2 \rangle$。

从 $\langle L^2 \rangle = \langle L_x^2 \rangle + \langle L_y^2 \rangle + \langle L_z^2 \rangle$:
\begin{equation*}
\langle L_x^2 \rangle = \langle L_y^2 \rangle = \frac{l(l+1)\hbar^2 - m^2\hbar^2}{2}
\end{equation*}

因此:
\begin{equation*}
\Delta L_x = \Delta L_y = \frac{\hbar}{\sqrt{2}}\sqrt{l(l+1)-m^2}
\end{equation*}
\end{derivation}

\section{实轨道与杂化}

对于 $m \neq 0$ 的复数球谐函数,可构造实数线性组合。

\textbf{p 轨道}:
\begin{align}
p_z &= Y_1^0 \propto z/r \\
p_x &= \frac{1}{\sqrt{2}}(Y_1^{-1}-Y_1^1) \propto x/r \\
p_y &= \frac{i}{\sqrt{2}}(Y_1^{-1}+Y_1^1) \propto y/r
\end{align}

\textbf{d 轨道示例}:
\begin{align}
d_{z^2} &= Y_2^0 \propto 2z^2-x^2-y^2 \\
d_{xy} &= \frac{i}{\sqrt{2}}(Y_2^{-2}-Y_2^2) \propto xy
\end{align}

\section{多电子原子}

\subsection{哈密顿算符与近似}

$N$ 电子原子:
\begin{equation}
\hat{H} = \sum_i\left(-\frac{\hbar^2}{2m_e}\nabla_i^2 - \frac{Ze^2}{4\pi\varepsilon_0 r_i}\right) + \sum_{i<j}\frac{e^2}{4\pi\varepsilon_0 r_{ij}}
\end{equation}

\textbf{中心力场近似}:每个电子在原子核和其他电子平均场中运动,
\begin{equation}
\hat{H} \approx \sum_i \hat{h}(i), \quad \psi \approx \prod_i \psi_i
\end{equation}

有效核电荷:$Z_{\text{eff}} = Z - \sigma$

\subsection{泡利原理与电子组态}

\textbf{泡利不相容原理}:两个电子不能占据完全相同的量子态(包括自旋)。

推论:每个轨道 $(n,l,m)$ 最多容纳 2 个自旋相反的电子。

\subsection{LS 耦合与光谱项}

多电子角动量耦合:
\begin{align}
\vec{L} &= \sum_i \vec{l}_i \\
\vec{S} &= \sum_i \vec{s}_i \\
\vec{J} &= \vec{L} + \vec{S}
\end{align}

光谱项符号:$^{2S+1}L_J$

\begin{keypoint}
洪特规则(确定基态):
\begin{enumerate}
\item 最大自旋多重度 $2S+1$ 最低
\item 相同 $S$ 时,$L$ 最大最低
\item 未满壳层:$J = |L-S|$ 最低;超半满:$J = L+S$ 最低
\end{enumerate}
\end{keypoint}

\subsection{选择定则}

电偶极辐射跃迁:
\begin{align}
\Delta l &= \pm 1 \\
\Delta L &= 0, \pm 1 \quad (0 \not\leftrightarrow 0) \\
\Delta J &= 0, \pm 1 \quad (J=0 \not\leftrightarrow J=0) \\
\Delta S &= 0
\end{align}

\section{重要公式速查}

\begin{scriptsize}
\begin{itemize}
\item 玻尔半径:$a_0 = 0.529$ \AA
\item 基态能量:$E_1 = -13.6$ eV
\item 里德伯常数:$R_\infty = 1.097\times 10^7$ m$^{-1}$
\item 能级:$E_n = -\frac{Z^2}{n^2} \times 13.6$ eV
\item 跃迁:$\Delta E = hc R_\infty Z^2\left(\frac{1}{n_1^2}-\frac{1}{n_2^2}\right)$
\item 角动量:$|\vec{L}| = \hbar\sqrt{l(l+1)}$,$L_z = m\hbar$
\item 积分:$\int_0^{\infty} x^n e^{-ax}\,dx = \frac{n!}{a^{n+1}}$
\item 对易子:$[\hat{x},\hat{p}_x] = i\hbar$,$[\hat{L}_x,\hat{L}_y] = i\hbar\hat{L}_z$
\item 不确定关系:$\Delta x \cdot \Delta p_x \geq \frac{\hbar}{2}$
\end{itemize}
\end{scriptsize}

\section{附录:常用数学公式}

\subsection{积分公式}

\textbf{含指数函数}:
\begin{align}
\int_0^{\infty} x^n e^{-ax}\,dx &= \frac{n!}{a^{n+1}} \\
\int_0^{\infty} x^{2n+1} e^{-ax^2}\,dx &= \frac{n!}{2a^{n+1}} \\
\int_0^{\infty} x^{2n} e^{-ax^2}\,dx &= \frac{(2n-1)!!}{2^{n+1}a^n}\sqrt{\frac{\pi}{a}}
\end{align}

$(2n-1)!! = 1\times 3\times 5\times\cdots\times(2n-1)$

\textbf{球谐函数相关}:
\begin{align}
\int_0^{\pi}\int_0^{2\pi} |Y_l^m|^2\sin\theta\,d\theta\,d\phi &= 1 \\
\int_0^{\pi}\int_0^{2\pi} Y_l^{m*}Y_{l'}^{m'}\sin\theta\,d\theta\,d\phi &= \delta_{ll'}\delta_{mm'}
\end{align}

\textbf{径向积分示例}:
\begin{align}
\int_0^{\infty} R_{nl}^2(r) r^2\,dr &= 1 \\
\int_0^{\infty} R_{10}^2 r \cdot r^2\,dr &= \frac{3a_0}{2} \\
\int_0^{\infty} R_{10}^2 \frac{1}{r} \cdot r^2\,dr &= \frac{1}{a_0}
\end{align}

\subsection{微分方程}

\textbf{勒让德方程}:
\begin{equation}
(1-x^2)\frac{d^2y}{dx^2} - 2x\frac{dy}{dx} + l(l+1)y = 0
\end{equation}
解为勒让德多项式 $P_l(x)$

\textbf{连带勒让德方程}:
\begin{equation}
(1-x^2)\frac{d^2y}{dx^2} - 2x\frac{dy}{dx} + \left[l(l+1) - \frac{m^2}{1-x^2}\right]y = 0
\end{equation}

\textbf{拉盖尔方程}:
\begin{equation}
x\frac{d^2y}{dx^2} + (k+1-x)\frac{dy}{dx} + ny = 0
\end{equation}
解为 $L_n^k(x)$

\subsection{常用导数与微分}

\textbf{球坐标梯度}:
\begin{equation}
\nabla f = \frac{\partial f}{\partial r}\hat{e}_r + \frac{1}{r}\frac{\partial f}{\partial\theta}\hat{e}_\theta + \frac{1}{r\sin\theta}\frac{\partial f}{\partial\phi}\hat{e}_\phi
\end{equation}

\textbf{常用函数导数}:
\begin{align}
\frac{d}{dx}(x^n e^{-ax}) &= x^{n-1}e^{-ax}(n - ax) \\
\frac{d}{dx}[e^x f(x)] &= e^x[f(x) + f'(x)]
\end{align}

\subsection{特殊函数值}

\textbf{阶乘与双阶乘}:
\begin{align}
(2n)!! &= 2^n n! \\
(2n-1)!! &= \frac{(2n)!}{2^n n!}
\end{align}

\textbf{Γ函数}:
\begin{align}
\Gamma(n+1) &= n!, \quad \Gamma(1/2) = \sqrt{\pi} \\
\Gamma(n+1/2) &= \frac{(2n-1)!!}{2^n}\sqrt{\pi}
\end{align}

\textbf{三角函数恒等式}:
\begin{align}
\sin^2\theta &= \frac{1-\cos 2\theta}{2} \\
\cos^2\theta &= \frac{1+\cos 2\theta}{2} \\
e^{i\phi} &= \cos\phi + i\sin\phi
\end{align}

\end{multicols}

\end{document}

